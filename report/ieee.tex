\documentclass[conference]{IEEEtran}
\usepackage[pdftex]{graphicx}
\usepackage[cmex10]{amsmath}
\interdisplaylinepenalty=2500
\usepackage{algorithmic}
\usepackage{array}
\usepackage{mdwmath}
\usepackage{mdwtab}
\usepackage{eqparbox}
\usepackage[font=footnotesize]{subfig}
\usepackage{fixltx2e}
\usepackage{url}
\hyphenation{op-tical net-works semi-conduc-tor}

% additional packages and utility commands
\usepackage{minted}
\usepackage{flushend}

% \usepackage{lineno}
% \setlength\linenumbersep{1mm}
% \linenumbers

\begin{document}

% to avoid syntax error highlighting - e.g. foo's not js ;-)
\expandafter\def\csname PY@tok@err\endcsname{}

\title{Lowering the Impact of Intrusion Detection\\
on Resources in Wireless Sensor Networks\\
using a Domain Specific Language\\
and Code Generation Techniques}

\author{\IEEEauthorblockN{Christophe Van Ginneken, Jef Maerien, Christophe
Huygens, Danny Hughes, Wouter Joosen}%
\IEEEauthorblockA{iMinds - DistriNet - KU Leuven\\
B-3001, Leuven, Belgium\\
Christophe.VanGinneken@student.kuleuven.be,\{firstname.lastname\}@cs.kuleuven.be}}

\maketitle

\begin{abstract}
  
Introducing intrusion detection in wireless sensor networks proves to be a
battle for resources. Implementing and optimizing a collection of detection
algorithms to match available resources might very well be an unacceptable
economic burden. This paper introduces a domain specific language to formally
describe such algorithms and proposes the use of code generation techniques to
automate the production of detection software. This automated process allows
for the optimization of resource usage. Specifically the optimization of the
execution time of the algorithms and the use of the energy-costly wireless
radio are prime candidates. A prototype code generator was realized to show
that these techniques can be implemented effectively, combining different
intrusion detection algorithms in such way that they impose less impact on the
resources than the sum of the impact of each algorithm by itself.

\end{abstract}

\section{Introduction}

A wireless sensor network (WSN) is constructed using a large number of
autonomous embedded devices, also called nodes. Their autonomy is rather
absolute, being most of the time battery-powered and deployed in open terrain.
Examples of WSNs include monitoring systems in volcanic regions
\cite{werner2006deploying} or flood areas \cite{hughes2006gridstix}.
Preliminary successes have now even steered researchers towards the
introduction of WSNs in cities and homes, thus creating smart cities
\cite{schaffers2011smart} and smart homes \cite{chan2008review}.

Due to their involvement in more and more personal applications, securing these
nodes should be a priority. When we entrust these autonomous systems with some
of our most intimate information, for example regarding our health
\cite{stankovic2005wireless}, we would of course like them to be able to
protect this information.

Securing WSNs, and especially single nodes, is a daunting task
\cite{perrig2004security}. Mostly due to their autonomous nature, nodes are
most of the time physically accessible. With physical access comes a wide range
of potential, physical attacks \cite{becher2006tampering} that are hard to
detect, let alone prevent. Even if the nodes aren't physically accessible,
their ways of communicating with the outside world through various forms of
wireless communication, offer a plethora of possibilities
\cite{padmavathi2009survey} to attack them, ranging from low-level meddling
with the routing of packets in the network, up to the application level.

Preventing intrusions should be a primary objective. Often this turns out to be
impossible and intrusions are sometimes only noticed post-mortem. When
prevention is not guaranteed, a second line of defence consists in the
detection of intrusions, allowing the introduction of reactive systems to
prevent future problematic situations. These intrusion detection (ID) systems
(IDS) are well-known in classic networked environments, but are not easily
transferred to the world of WSNs \cite{zhang2000intrusion}
\cite{djenouri2005survey}, mostly due to the wireless and ad-hoc nature of the
networks and the lack of a central point where communication can be monitored.

But WSNs also bring another problem to the table: resources. Due to their
vastly deployed numbers and policy to be replaced rather than salvaged, nodes
typically need to be cheap. Their limited functional requirements allow for
them to be built using simple components without redundancy or excess margins.
This introduces an inherent problem for IDS. These systems not only typically
require a lot of resources to store detection information and have to execute
their algorithms over and over again, but they also want to inspect every
single network packet that's passing by. This way they not only would like the
node to be powered-on all the time, they also require constant access to the
wireless radio, which turns out to be the number one energy-consumer of a node.
Introducing ID in WSNs turns out to be a battle for resources.

Finally, besides the technical resources, there is also an economic resource
that plays an important part here. Given the same limited functionality, adding
a complex piece like an IDS, might raise the production cost well above
acceptable levels. The fact that ID algorithms for WSN are currently in a mere
state of research, would currently require a developer to read through many
research papers, making a selection of algorithms and implement them from
scratch. Even if implementations for each of these algorithms would be
available, the chance that they are applicable to the target platform is a big
\emph{if}. Even simply integrating them can be a complex task.

Our contribution to ID in WSNs is the introduction of a domain specific
language (DSL) to formally describe ID algorithms. The DSL is node-oriented
with a strong focus on inter-node communication and aims to allow for the
optimisation of execution of multiple ID algorithms, thus lowering the
sequential and iterative execution of algorithms and reducing the use of the
wireless radio.

Introducing such a formal, platform-independent language to express ID
algorithms and accompanying this language with a code generation framework,
offers an end-to-end solution to many of the problems introduced above.
Research output would be directly applicable in a development environment and a
code generation framework allows developers to simply select algorithms and
have platform-specific code without any effort. Due to the possibility to
optimize the use of resources, the impact of introducing an IDS can be lowered
allowing more algorithms to be added, thus augmenting the barriers, which can
help to reduce the number of undetected intrusions.

The remainder of this paper proceeds as follows: section \ref{section:related}
introduces research regarding ID in WSNs. Section \ref{section:problem}
describes the problem space in detail. Section \ref{section:solution}
introduces our proposed overall solution, further detailed in sections
\ref{section:foo-lang} and \ref{section:prototype}, which respectively present
our domain specific language, FOO-lang, as well as a prototype implementation
of a code generator framework to complement the language. Section
\ref{section:evaluation} evaluates the prototype implementation and determines
if the theoretical merits of introducing FOO-lang, are actually viable. Our
conclusions and proposed topics for future research are presented in section
\ref{section:conclusions}.

\section{Related Work}
\label{section:related}

When securing any network one tries to prevent intrusions. This should be the
primary concern, but not all intrusions can be prevented and we need to fall
back on detecting intrusions. In this section we look at recent contributions
in this field.

Intrusions of computer networks can take many forms. In the case of WSNs this
classification needs to be extended even further. In
\cite{padmavathi2009survey} a good overview of both is presented, showing
clearly that WSNs suffer from their wireless and broadcasting nature. This
causes many attacks based on e.g. eavesdropping to be nearly undetectable while
they are happening. Only when the actual intrusions, based on the covertly
collected information, have happened, the intrusion could possibly be detected
based on the after-effects. So we can't cover the entire spectrum and need to
focus on what is feasible.

Research into ID in WSNs typically focuses on two major topics that complement
the architecture of WSNs: the nodes as single entities and the network as a
group of such nodes. Both are important because the group cannot make a
decision without members that detect malicious behaviour and a group-based
decision is often needed because nodes can easily miss out on certain events
that could indicate intrusions, due to their wireless and not-always-on nature.

\subsection{Detecting Intrusions}
\label{subsection:detecting}

There are different ways to construct a taxonomy for intrusion detection, but
common themes do appear. According to \cite{mishra2004intrusion} and
\cite{ioannis2007towards} there are three major categories: anomaly detection,
signature or misuse detection, and specification-based detection. In
\cite{alrajeh2013intrusion} the authors mostly agree with this topology, but
add the notion of hybrid intrusion detection systems and cross layer intrusion
detection systems as recent advances in research. Although these additional
categories offer very interesting prospects, the authors have to admit that the
impact on the resource-constrained sensor nodes might still be too high.

The meshed network topology of WSNs and its routing protocols have produced
many related attacks and each of these attacks has been the focus of many
research papers presenting algorithms to detect them. In the following
paragraphs we take a look at some popular ones.

In a wormhole attack, an artificial low-latency link is created between two
distant nodes in the network. This causes the routing protocol to divert more
traffic through this link, offering the attacker an abundance of packets to
inspect. In \cite{maheshwari2007detecting} an algorithm is proposed to allow
nodes to determine if a wormhole is actively present. In essence, the algorithm
relies on the exchange of neighbour lists to allow nodes to determine if
unexpected substructures are present in the graph representing the network.

Another attack that is related to the wormhole is the sinkhole. When launching
a sinkhole attack, as described in \cite{krontiris2008launching}, the attacker
first needs to compromise an existing node. Being in control of the node, the
attacker can make his captured node look more \emph{attractive} to the
surrounding nodes and thus draw more traffic to itself. The sinkhole attack is
typically a foundation for other routing level attacks, such as selective
forwarding, modification of packets or even dropping them altogether. In its
simplest form, the sinkhole offers the attacker with lots of packets to analyze
and use to gather information about the network and its functionality.

Detecting sinkholes is very hard and is often only possible based on other
attacks that are supported by the sinkhole. When the sinkhole is used to
implement selective forwarding, \cite{ngai2006intruder} proposes an algorithm
that allows a base station, the aggregating master node to which all nodes
typically send their information, to observe missing data from nodes in the
same area in a statistical way.

More typical attacks such as flooding, the sybil attack, rushing,\dots are
amongst others presented in \cite{wood2002denial} and \cite{djenouri2005survey}.

\subsection{Cooperative Decision Making}
\label{subsection:coorperative}

As in many other situations, a group is often stronger than the sum of its
individual components. This is surely the case for WSNs. Because not all nodes
are always actively participating in the network, some nodes might miss clues
that would otherwise lead them to detect intrusions. It is hardly impossible
for a single node to detect a complex attack by itself. Therefore a second
important research topic consists of cooperative algorithms to combine
information from single nodes into a group-based decision about alleged
intrusions.

In \cite{krontiris2009cooperative}, the authors first present a theoretical
foundation to analyze cooperative algorithms. Given these foundations they
continue to present an algorithm consisting of 5 phases. It essentially
implements a voting system based on the Guy Fawkes protocol
\cite{anderson1998new} to identify an intruder. It enables the exchange of
suspected intruder information and allows distributed authentication of those
\emph{votes}. Based on these authenticated votes, a distributed decision can be
made about the commonly identified intruder.

A distributed, cooperative algorithm can sometimes be implemented locally. This
is illustrated by the reputation-based detection algorithm introduced in
\cite{ganeriwal2008reputation}. Using this algorithm, nodes can exchange
information about the reputation of other nodes. Combining this information
allows them to decide about their trust in a given node, based on more than
their own, often partial, observations.

\subsection{Software Attestation}
\label{subsection:attestation}

A research topic that exists in parallel to the detection and cooperation
duality is software attestation. Software attestation offers algorithms for
exchanging information about the software that is running on a node, and aims
to identify nodes that can no longer prove that their content is unaltered.
Examples of evolving algorithms to implement this functionality include SWATT
\cite{seshadri2004swatt}, ICE/SCUBA \cite{seshadri2006scuba} and SAKE
\cite{seshadri2008sake}.

Software attestation is hard and many details can cause an algorithm to fail.
Interesting is the discussion surrounding this topic that was started by
\cite{castelluccia2009difficulty}, in response to the previously mentioned
papers. In \cite{perrig2010refutation} the authors of the original papers
counter many of the objections made to their work, but the general feeling is
that even in the best conditions, there is always a way to circumvent even the
most ingenious algorithm to perform software-based attestation.

\section{Problem Analysis}
\label{section:problem}

The problem of introducing ID in WSNs is much broader than simply the
implementation of a software component. It starts at the very foundations of
WSNs. In contrast with for example SNORT \cite{roesch1999snort}, the de facto
standard with respect to IDS in classic networks, there is currently no
community based collection of algorithms that can be implemented. The reason is
mostly due to the lack of a central/external location where all network traffic
can be analyzed out-of-band. It's not possible to create a single entity that
will monitor the entire WSN as is the case in a classic wired network.
Therefore, all algorithms typically evolve independently, without any cohesion.

Even more, all of these algorithms lack a common, formal notation. In the case
of SNORT, a detection language is provided and new signatures are added on a
regular basis, creating an ever growing rule base, capable of detecting even
the newest attacks out there.

If implementers of a WSN want to add ID to their network, and therefore to the
nodes in the network, they are facing no small task: they need to gather
research papers, from which they need to extract the proposed algorithms. Even
if the papers would come with an implementation of their algorithm, the chance
that the implementation matches the target platform of the newly created WSN is
slim.

Simply implementing the different algorithms in sequence, or reusing existing
implementations if they would be available, is also no valid option. All of
these algorithms typically perform the same actions: first they analyze each
incoming packet and collect information about nodes. Secondly, at given
intervals, they iterate all known nodes, checking the aggregated information
about them and making decisions about the trust to put in them. The algorithms
typically also exchange information with other nodes to complement their own
findings.

If such algorithms would simply be combined in a sequential way, the resulting
code would be far from optimal and would consume many of the resources of the
node it runs on: the repetitive parsing of the received packets would result in
much longer processing times than needed, while the chattiness of the
inter-node communication would cause the wireless radio to be on more often
than wanted.

The cost to implement multiple algorithms over and over again, due to the need
to optimize the code's organisation, would soon outweigh the available budget
of any WSN implementation project. If not, the risk to make mistakes due to
misinterpreting the often complex algorithms or the lack of flexibility to
change sets of detection algorithms on a regular basis, or simply add a new
algorithm in an easy way, are all very valid reasons to not go down this road.
Then, how can we add some form of an IDS to WSNs?

\section{Proposed Solution}
\label{section:solution}

Different approaches are possible. A major contribution would consist in the
creation of a software framework that accommodates some of the typical usage
patterns: first, a generic payload parser could be envisaged, allowing
algorithms to hook in using callbacks, thus making sure that the parsing is
only done once and not repeated for each algorithm. Secondly, the framework
could collect all outgoing messages and combine them in a single outgoing
packet at the end of a cycle of the node's event loop.

Such a framework comes with guidelines on how to use it, but still relies on
good behaviour of its users. Also, the integration of the algorithms with this
framework would still be manual work and reconfiguration would typically impact
this part of the implementation. Finally, although such a framework does
support the implementation, it still requires thorough analysis of research
material and extraction of the relevant algorithms.

In the case of SNORT \cite{roesch1999snort}, or any other classic network IDS
for that matter, the central detection engine accepts formal descriptions of
signatures of attack patterns. Because it's not possible to create such a
central component once, there has been no urge to describe detection algorithms
in a formal way. By introducing code generation techniques to convert a formal,
platform-independent description of detection algorithms, and generating this
central component, we can bridge the remaining gap.

Code generation allows for the creation of code that implements the actual
algorithm. The algorithm can be described in a platform-independent way,
therefore allowing reuse of the algorithm on multiple platforms. The generated
code can further access a minimal software framework that covers common tasks
and/or platform specific functionality.

Given formal descriptions of the algorithms, an implementer of a new WSN could
simply select a set of algorithms, feed their formal description to the code
generator and obtain platform-specific code, optimized for execution and
communication.

A formal description of ID algorithms would not only be beneficiary to
implementers, but also to researchers. Formal descriptions of the algorithms
allow for automated generation of code, but the same descriptions can also be
loaded into simulators and be investigated in combination with other algorithms
\cite{mernik2005and}. Another interesting option would be that of formal
analysis of the algorithms. Further, describing the algorithms in a
platform-independent way, would increase their usefulness and allow for more
complex algorithms to be abstracted.

\section{Introducing FOO-lang}
\label{section:foo-lang}

The central component of our proposed solution is a formal description of
intrusion detection algorithms. This formal description can be realized using a
domain specific language (DSL). Before actually introducing a new language, it
is important to ensure the reasons to do so are justified. Too often languages
are implemented too quickly, where coding conventions or frameworks with well
thought-out APIs fit the bill quite nicely.

The line can not be drawn in a black and white fashion and DSLs do have
benefits, but don't come for free \cite{mernik2005and}. Many good reasons why
to implement a DSL, or not, can be found in \cite{mernik2005and},
\cite{van2000domain} and \cite{fowler2010domain}. Based on these, we try to
justify our choice to propose a DSL for this solution.

\subsection{Justification of the Use of a DSL}
\label{subsection:justification}

The primary goal for the proposed solution is to avoid the creation of
inefficient code that drains the limited resources of a node. Two basic
examples are presented to illustrate this unwanted behaviour: repetitive
execution of the same tasks, such as parsing or iterating a set of known nodes
and the use of the wireless radio.

The latter can be covered by a framework, as illustrated before. It comes with
an obligation to actually use the framework. The former is a bit trickier.
Imagine using the C language as a lingua franca to complement the framework,
that hides the platform specific parts. This could be valid, but offering a
complete language would soon result in loops that break with the guidelines and
would undermine the conceptual idea of optimizing the organisation of the
functionality.

Restricting the formal description to a subset that doesn't allow for
constructions that violate the goals, is in this case a valid reason to tend
towards a DSL. Introducing a DSL further also allows for more functional ways
of handling the concepts that are part of the domain. In this case the domain
clearly consists of interactions between nodes and local processing of
algorithms that typically deal with sets of nodes. Although that C comes close
to a lingua franca amongst both researchers and implementers, it is still a
low-level language with very little support for concise operations such as
pattern matching, event-driven-ness,\dots.

If C doesn't fit the profile, maybe another language does. During the early
days of our research, we looked at a language that seemed to fit the required
functionality like a glove: Erlang \cite{armstrong1993concurrent}. It comes
with strong communication paradigms, has pattern matching, is concurrent and
event-driven by nature,\dots and according to \cite{wong1998compiling} it can
even be translated to C. This road looked very promising, but as with any other
general purpose language, it proved to be too expressive and would allow for
constructions that would thwart the envisaged optimizations.

Our final conclusion is that a real domain-functionality inspired DSL, that
permits researchers to express the detection algorithms without overhead and
without risk of introducing ill behaviour, is the way to move forward.

\subsection{Design Principles for FOO-lang}
\label{subsection:design}

Designing a language is an exciting task, but at the same time comes with great
responsibility. To paraphrase Albert Einstein: it is important to make the
language ``as simple as possible, but not simpler''. The language should come
with enough support to write concise code, but not become obfuscated. The
language constructs it offers should be applicable in generic combinations with
each other and be especially transparent in use. On the other hand, it is also
important that the language restricts the user without limiting his
expressiveness or making certain expressions unintuitive.

Starting from this last requirement, one of the most important things that
should not be available are \emph{loops}. If we want to control the way lists
of data are handled, we need control over the loops targeting that data. In
case of this specific domain, data is centralized around \emph{nodes}, so loops
are typically targeting \emph{nodes}. To meet this restriction but still offer
the user a way to functionally handle nodes, scheduling and events are
introduced. Instead of explicitly constructing a loop, functionality can be
scheduled or attached to certain events that happen on the collection of nodes.
This way, the intended loop is actually abstracted to its functional meaning
and can be handled and combined with other execution strategies.

This is the actual core of the language and it is also where its name derived
from: Function Organization Optimization.

A second important feature of the language is that it should feel natural.
Luckily, dealing with embedded systems and WSNs, both researchers and
implementers share a common language. Most development on these systems is done
using C, and both parties know this language well. We therefore chose to mimic
C for a lot of the general syntax.

FOO-lang also borrows basic syntax from object-oriented (OO) languages, in that
respect that it bears the concept of methods that can be called upon
\emph{objects}. The language doesn't provide any means to define or instantiate
objects though. The \emph{nodes} domain controls the availability of these
objects, creating them and providing them to the functions that are defined.

To avoid typical constructions that require loops, the concept of pattern
matching is introduced. A pattern of non-variables and variables can be
provided to functions that support them. If such a function can match the
non-variable part of the pattern, the variables in the pattern take on the
corresponding values of the data against which the match was performed. It is
obvious that this will typically be used to analyze incoming data when
communicating between nodes. List literals are often used in conjunction with
this functionality and allow to combine different (non-)variables; these are
therefore also available in FOO-lang.

To allow complex code to be introduced, FOO-lang provides a way to import
external functionality. This feature is provided through a statement that
allows to define the prototype of the imported function, which can then be used
as any other function.

One final important feature is type inference. FOO-lang tries to limit the need
for typing information. Based on declaration and usage, most types can be
inferred. In doing so, the language tries to focus on the functionality and not
on the technical implementation. This mainly targets the researchers who now
can focus on the bare essence of their algorithms. Also, typing can be
platform-dependent and we want the descriptions to be platform-independent.

\subsection{The Actual Language}
\label{subsection:language}

In the previous paragraphs the design principles of FOO-lang were introduced.
In this section we briefly introduce some of the syntax of FOO-lang using an
example implementation.

Based on the algorithms that were introduced in many of the consulted papers, a
common structure for these algorithms can be detected: (1) when new data is
received the algorithm wants to process it, (2) at regular intervals the
algorithm wants to validate the aggregated information on nodes and (3) the
algorithm might want to communicate with other nodes to exchange information.
Given this structure, we now introduce the \emph{hello world} example for
FOO-lang: \emph{heartbeat}.

It can be considered the most elementary ID algorithm possible, still
containing all aspects of a generic algorithm. It deals with availability: as
long as we receive regular updates from a node, we trust it. When it fails to
produce a \emph{heartbeat}, we no longer trust it. The resulting FOO-lang code
is presented in listing \ref{lst:heartbeat}.

Following the design principles, most of this example code should be easy to
understand. FOO-lang tries to be ``readable'', especially to users with a C
background and common knowledge about other languages and programming
paradigms. Most language constructs are borrowed from other languages, like
annotations using the `@' symbol or atoms with the `\#' prefix.

Defining a language is one thing. It should of course also be possible to
generate code from it that actually addresses the problems we identified
before. In case of a language, the real proof is in building an actual code
generator.

\vspace{1.2cm}

\inputminted[linenos,xleftmargin=2mm,numbersep=1mm,frame=lines,framesep=2mm,fontsize=\footnotesize,fontfamily=courier]{js}
  {heartbeat.foo}
\captionof{listing}{Example implementation of a \emph{heartbeat}.\label{lst:heartbeat}}

\section{Prototype Code Generator}
\label{section:prototype}

We have built a prototype code generator for FOO-lang using Python and ANTLR
\cite{antlr3}. The implementation tries to be as generic as possible and is
based on transformations using the visitor pattern of abstract syntax trees
(AST) and a semantic model \cite{fowler2010domain}.

The FOO-lang code is parsed using a parser generated by the ANTLR tool. This
produces an AST, which in its turn is transformed into a semantic model using a
visitor. The semantic model represents the actual meaning of the
implementation. From a project point of view, the DSL is merely a textual
representation of the semantic model and the parser and first visitor allow us
to easily populate the model.

The initial model is an exact representation of the FOO-lang code that was
parsed. This means for example that not all types are well defined. A second
visitor, now targeting the semantic model, is used to check all types and tries
to infer those that are still marked \emph{unknown}.

With all types known, the model is semantically complete and we can move to the
next phase: construction of a code model. A code model can actually be
considered an AST that can be persisted as code again. The initial code model
that is constructed from the semantic model typically contains all the language
features available in FOO-lang. The construction is done partly by the generic
generator and partly by the \emph{nodes} domain.

The code generator aims to be language and platform independent. Both language
and platform are exchangeable plug-ins to the generator. With the initial code
model based on the full set of language constructs of FOO-lang, it is clear
that the transformations that need to be performed by these plug-ins are the
migration of features that are not literally available in the target language
or platform.

The resulting code model is an AST that can be emitted in the desired language
and conforming to the target platform.

\section{Evaluation}
\label{section:evaluation}

To evaluate the prototype we chose to use a system based on the Atmel
ATMEGA1284p micro-controller \cite{datasheet:atmega1284p} and the Digi XBee S2
Zigbee module \cite{manual:xbee}. To drive the hardware a minimalistic library
was used, wrapping the technical calls to the components in more easy to use
functions.

The reason for this at first sight unusual setup is simple: relying on basic
hardware and software allows us to validate that the generator is capable of
generating code even for the most basic environment available. Any step up,
both in hardware or software, would offer better and higher abstractions that
would make it easier to generate code for that situation. It shows that the
requirements of the generator towards the platform are minimal, don't rely on
advanced frameworks or operating systems and can therefore be applied in any
environment, targeting any platform.

For the evaluation we constructed a small WSN consisting of three nodes: an
end-device, a router and a coordinator. The end-device and the router were
completely autonomous nodes, while the coordinator consisted of an XBee module,
directly connected to a computer via a USB break-out board, allowing serial
access via a terminal.

We started from a basic application that measures light intensity and reports
it to the coordinator. The application was written both manually and generated.
On top of this baseline we added two intrusion detection related algorithms: a
heartbeat, that allows nodes to validate each other's continuous presence, and
a reputation-building algorithm that checks if a parent node is cooperative and
actually forwards messages with a destination further down the
network\cite{ganeriwal2008reputation}.

Three criteria were evaluated: the image size of the resulting compiled code,
the network usage in number of frames and bytes and the time required to
perform once cycle of the event loop. These metrics were collected in four
situations: without ID algorithms, with a heartbeat, with reputation-tracking
and with both algorithms implemented.

In case of the manual implementation, both algorithms were constructed as
standalone modules and sequentially called from the base-application's
event loop. The code generator was provided with FOO-lang descriptions of the
algorithms and generated all four cases.

Before reviewing the results, it is very important to put these results in a
proper perspective. Comparing manually written code to generated code is no
exact science. One can always argue that the quality of the two code bases is
not comparable and that the manual code always can be improved.

Still it might be interesting to look at the numbers, given that both code
bases are constructed with the same intentions and practices. We believe we
have achieved this and have taken honest decisions while constructing the
implementations, making them comparable. Numerical analysis of the
implementation offers us a preliminary estimation of the effect of our proposed
solution.

corollaryTables \ref{tbl:manual} and \ref{tbl:generated} show the collected data
respectively for the manual and the generated implementation. The base case is
presented in absolute values, while the implementations with the added
algorithms are presented as relative values.

\begin{table}[H]
  \centering
  \begin{tabular}{lrrrr}
  \hline
      & base & heartbeat & reputation & both\\
  \hline
  size (bytes) & 10500 & 148\% & 127\% & 175\%\\
  frames & 20 & 255\% & 160\% & 315\%\\
  bytes & 476 & 406\% & 181\% & 487\%\\
  time ($\mu$s) & 48 & 196\% & 183\% & 310\%\\
  \hline
  \end{tabular}
  \caption{Results for the manual implementation.}
  \label{tbl:manual}
\end{table}

\begin{table}[H]
  \centering
  \begin{tabular}{lrrrr}
  \hline
         & base & heartbeat & reputation & both\\
  \hline
  size (bytes) & 10496 & 175\% & 156\% & 200\%\\
  frames & 20 & 245\% & 160\% & 275\%\\
  bytes & 476 & 399\% & 186\% & 454\%\\
  time ($\mu$s) & 48 & 252\% & 252\% & 288\%\\
  \hline
  \end{tabular}
  \caption{Results for the generated implementation.}
  \label{tbl:generated}
\end{table}

From these raw results, we learn that adding algorithms manually to the
base-application, results in a cumulative increase. The impact of both
algorithms is simply the sum of the impact of each algorithm by itself. In the
case of the time required to execute one cycle of the event loop, the total
time is even a bit higher than simply the sum.

In the case of the generated implementation, this no longer holds: although
that even the individual increases due to adding a single algorithm are higher
than in the manual case, the result for the implementation of both algorithms
is less than the sum of both. Here we clearly see the effect of reusing the
framework that comes with FOO-lang. The same goes for all other metrics. Maybe
most remarkable is the effect on the time of one event loop cycle: both
algorithms by itself add 150\% with respect to the base case, but when
combined, the impact hardly increases. Here we learn that the introduced
framework is responsible for the initial overhead, but it clearly pays for
itself when adding more algorithms.

Table \ref{tbl:summary} compares the two situations by subtracting the manual
case from the generated case, showing the impact of the code generation. A
relative value for the entire implementation with both algorithms is also
presented.

\begin{table}[H]
  \centering
  \begin{tabular}{lrrrrr}
  \hline
                & base & heartbeat & reputation & both  & result \\
  \hline
  size (bytes)  & -4    & 2822     & 3070       & 2664  & 115\%  \\
  frames        & 0     & -2       & 0          & -8    & 87\%   \\
  bytes         & 0     & -36      & 24         & -156  & 93\%   \\
  time ($\mu$s) & 0     & 27       & 33         & -11   & 93\%   \\
  \hline
  \end{tabular}
  \caption{Comparison of both implementations.}
  \label{tbl:summary}
\end{table}

When comparing the results of both implementations we first can conclude that
the generator produces exactly the same code for the base case. We couldn't
measure any real differences between both implementations.

Secondly, we see that the framework that comes with the generated code adds to
the size of the resulting image. But the cost is almost constant and is
relatively lower when algorithms are combined. With roughly 3KB of additional
overhead, or 15\% in this simple case, this impact is affordable.

From a functional point of view, the generated code lives up to the
expectactions: both the number of frames as the bytes sent benefit from well
organized code and reuse of a common framework.

Finally we see that both algorithms by itself add an equal amount of time to
the processing of a single event loop cycle, but when combined, the event loop
is about 7\% faster compared to the manual case.

Without any effort to optimize the framework code, nor leveraging knowledge
about the specific algorithms, this very basic generated code shows that the
proposed principles are not only theoretically feasible but actually result in
interesting improvements.

\section{Conclusions and Future Work}
\label{section:conclusions}

WSN nodes typically respond to events in their environment or perform their
tasks at clearly scheduled intervals. Applying a language that has these
concepts at its core, allows focus on the bare essence and reuse of best
practices at the generation level. The rather limited functional domain of ID
in WSNs proofs to be a wonderful candidate for applying code generation
techniques.

The ideas proposed in this paper and the prototype are foundations. They show
that code generation techniques can be used to combine different, independent
algorithms in an automated way, while offering better resource usage than
simply sequentially combining the algorithms.

These preliminary steps open up a broad range of possible tracks to explore. On
one hand our goal is to collect as many relevant research papers as possible
and extract the algorithms from them, implementing them using FOO-lang. This
will undoubtedly lead to extensions to the language and its core libraries. On
the code generation front itself, the generator has to be lifted from the
prototype level up to production quality. Next, many optimizations can be added
to it: functional code inlining, continuation-passing style code
structuring,\dots

On a higher level, other challenges are also prominently present: having
formal, platform-independent descriptions of algorithms, allows for simulation
and thorough analysis of their behaviour, both by themselves and in
combinations. Finding ways to combine as many algorithms as possible in a
single generated solution will enable implementers of WSNs to add a decent
level of intrusion detection to their solutions.

Another interesting track is that of policy-based generation and deployment.
Different nodes in the network might benefit from different configurations of
intrusion algorithms. When trying to detect a combination of selective
forwarding and a sinkhole attack \cite{ngai2006intruder}, this algorithm should
only be deployed on coordinators or base stations, not on end-nodes.

Finally we hope that this paper might act as a call-for-participation by all
ID in WSNs researchers and will lead to an ever growing collection of formal
descriptions available to those that need them.

\section*{Acknowledgements}
\label{section:acknowledgements}

We would like to thank the reviewers for their thoughtful and helpful comments
that enhanced the readability of this paper. Our gratitude and respect also
goes out to all members of the WSN work group at KU Leuven for creating the
nurturing environment where these ideas could grow.

\bibliographystyle{IEEEtran}
\bibliography{referenties}

\end{document}

