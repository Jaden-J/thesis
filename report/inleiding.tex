%!TEX root=masterproef.tex

\chapter{Inleiding}
\label{inleiding}

\TODO

\section{Draadloze sensornetwerken}

\TODO

\subsection{Sensorknopen}

\TODO

\section{Inbraakdetectie}

\TODO

\section{Probleemstelling}

\TODO

\section{Doelstelling}

In de literatuur betreffende ``inbraakdetectie in draadloze sensornetwerken'',
ligt de nadruk in hoofdzaak op het detecteren van specifieke aanvallen of het
vaststellen van anomalie\"en in het verwachte gedrag van sensorknopen en/of het
netwerk dat hen verbindt.

Deze werken stellen tevens dat het een nagenoeg onmogelijke taak is om alle
benodigde detectiemechanismen effectief te implementeren. Dit is logisch
gegeven het beperkte aanbod aan middelen die sensorknopen typisch ter hunnen
beschikking hebben. Zo zou bv. een exhaustieve lijst van aanvalspatronen
slechts in sensorknopen met zeer grote hoeveelheden geheugen kunnen opgeslagen
worden en zouden de berekeningen die nodig zijn om verschillende anomalie\"en
te detecteren gewoonweg te veel energie verbruiken.

Als in deze fase van onderzoek naar systemen om inbraken te detecteren het niet
mogelijk is om een sluitend intrusie detectie systeem voor draadloze
sensornetwerken te ambi\"eren, lijkt het opportuun om een stap terug te nemen
en de focus eerst te leggen op de middelen die nodig zijn om de reeds
beschreven en mogelijk ook toekomstige oplossingen te realiseren. Is het
mogelijk om een raamwerk te cre\"eren dat een implementator van een sensorknoop
in staat stelt om een selectie van de in de literatuur beschreven oplossingen
te implementeren, zonder zich zorgen te moeten maken over de onderliggende
interactie met andere knopen, het vergaren en opvragen van informatie op
systeem-niveau,...?

Deze thesis wil zulk een raamwerk ontwerpen, implementeren en de impact ervan
bepalen. Daartoe zal eerst een lijst gemaakt worden van de verwerkte
oplossingen, waaruit de functionele en technische vereisten voor het raamwerk
gedistilleerd kunnen worden. Vervolgens zal een architectuur voorgesteld worden
die aan deze vereisten voldoet. Aan de hand van een implementatie van deze
architectuur zal tot slot nagegaan worden wat de impact is van dit raamwerk met
betrekking tot geheugen en rekenkracht.

De voordelen van zulk een raamwerk zijn legio: een herbruikbaar raamwerk neemt
zorgen, gemeenschappelijk aan de verschillende oplossingen weg en kan zorgen
voor een optimale implementatie. Door middel van een goedgekozen technische
architectuur kan tevens platform-onafhankelijheid nagestreefd worden.
