%!TEX root=masterproef.tex

\begin{preface}

Het is niet iedereen gegeven om op 40-jarige leeftijd een masterproef te mogen
of kunnen maken. De keuze om opnieuw te gaan studeren maak je niet alleen en
vraagt van veel mensen een bijzondere inspanning. Familie en vrienden, maar ook
professoren en assistenten worden geconfronteerd met een ongewone situatie.

Deze masterproef is het resultaat van veel meer dan de afgelopen drie jaren van
hernieuwde studie aan de universiteit van Leuven. Oude en nieuwe ervaringen
gaan hand in hand en hebben geleid tot vragen en antwoorden die groter zijn dan
de som van de afzonderlijke delen.

Dat ik deze masterproef heb kunnen realiseren met de hulp van een fantastisch
team is slechts het topje van de ijsberg. Dat de mensen in dit team ook nog
eens een band hebben met het verleden onderstreept de ondertoon van deze
masterproef.

Professor dr. ir. Wouter Joosen en prof. dr. ir. Christophe Huygens staan
onrechtstreeks weer aan de bakermat van mijn toekomst. Hun passie en
gedrevenheid zijn een bron van inspiratie geweest om dit moeilijke onderwerp
aan te pakken en het tot een goed einde te brengen.

Ofschoon de ongewone situatie soms voor spanningen zorgde, was Jef Maerien de
juiste begeleider, die mijn vlot op deze woelige rivier dikwijls met een klein
manoeuvre op koers wist te houden. Sorry voor mijn soms arrogante attitude en
bedankt voor het behouden vertrouwen. Jouw kritische opmerkingen hebben mij
meer dan eens op de tippen van mijn tenen gehouden en hebben geleid tot betere
resultaten.

Het team bleef niet beperkt tot promotor, co-promotor en begeleider. Tal van
andere professoren en assistenten stonden mij met raad en daad bij wanneer ik
hen benaderde met mijn vragen. Ook buiten de universiteit kon ik steeds rekenen
op mijn vertrouwde klankborden om de juiste weg te vinden. Koen, misschien is
dit slechts het zoveelste fijne project geweest; ik vond het toch extra
speciaal.

Ook al draagt dit werk ogenschijnlijk alleen mijn naam, deze tekst zou nooit
geworden zijn wat hij nu is zonder de talloze uren die mijn team van lezers -
Erik \& Annemie, Bart \& Ans, Pieter \& Else - er hebben in gestoken. Van het
eerste artikel tot de laatste referentie en elke overtollige zinsnede ertussen,
hebben ze gewikt en gewogen en meestal er voor gezorgd dat de tekst er meer dan
vooruit op ging.

Naast mijn persoonlijk gekozen lezers, wil ik ook mijn beide assessoren, \TODO,
bedanken voor de tijd die ze namen om dit werk te evalueren. Ik hoop werkelijk
dat wij later de gelegenheid zullen vinden om alsnog van gedachten te wisselen.
Uw bemerkingen kunnen mij immers helpen om bepaalde aspecten nog beter uit te
werken. Want voor mij is deze masterproef geen eindbestemming.

Maar ondanks de ongelooflijke steun van heel dit team, was deze masterproef
nooit gelukt zonder een team op het thuisfront. Opnieuw drie jaar gaan studeren
overstijgt een klassieke dagtaak en legt een veel grotere druk op een gezin dan
we voor aanvang hadden kunnen inschatten. Zo heeft mijn \emph{soulmate}
Kristien elke seconde van de afgelopen drie jaar, elke druppel zweet en tranen
mee doorgemaakt, hebben ouders en schoonouders op de meest onmogelijke momenten
ingesprongen om mij tijd en ruimte te geven om dit huzarenstukje te realiseren
en moesten we meer dan eens beroep doen op vrienden en buren om allerhande
elementaire karweien voor ons te doen. Het is hartverwarmend om te ervaren
hoeveel mensen de afgelopen drie jaar meegeleefd hebben.

Tot slot zijn er nog twee mensen die echt een hele dikke knuffel verdienen. Van
iedereen begrijpen zij misschien het minst wat papa gedaan heeft de afgelopen
jaren en waarom er zo weinig tijd voor hen overbleef. Eline en Arjen, mijn
lieve schatten, bedankt dat jullie soms zo begripvol waren en me dikwijls
opnieuw moed hebben gegeven om toch door te gaan.

\bigskip

Bedankt!

\end{preface}
