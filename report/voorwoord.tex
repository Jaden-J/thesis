%!TEX root=masterproef.tex

\begin{preface}

Het is niet iedereen gegeven om op 40 jarige leeftijd een masterproef te mogen,
of eerder kunnen, maken. De keuze om opnieuw te gaan studeren maak je niet
alleen en vraagt van veel mensen een bijzondere inspanning. Familie en
vrienden, maar ook professoren en assistenten worden plots uit hun vertrouwde
omgeving weggerukt en worden geconfronteerd met een ongewone situatie.

Net zoals snel zal blijken, is deze masterproef het resultaat van veel meer dan
de afgelopen drie jaren van hernieuwde studie aan de universiteit van Leuven.
Oude en nieuwe ervaringen gaan hand in hand en hebben geleid tot vragen en
antwoorden die samen groter zijn dan de som van de afzonderlijke delen.

Dat ik deze masterproef heb kunnen realiseren met de hulp van een ongelooflijk
team is slechts het topje van de ijsberg. Dat de mensen in dit team ook nog
eens een band hebben met het verleden typeert meer dan eens de ondertoon van
deze masterproef.

Professor dr. ir. Wouter Joosen en prof. dr. ir. Christophe Huygens beseffen
het misschien niet ten volle, maar staan eens te meer aan de bakermat van mijn
toekomst. Hun passie en gedrevenheid zijn opnieuw een ongelooflijke bron van
inspiratie geweest om dit moeilijke onderwerp aan te pakken en het tot een goed
einde te brengen.

Ofschoon de ongewone situatie soms zorgde voor spanningen, was Jef Maerien
tevens de enige juiste begeleider, die mijn vlot op deze woelige rivier
dikwijls met een klein manoeuvre op koers wist te houden. Sorry voor mijn soms
arrogante attitude en bedankt voor het behouden vertrouwen. Jouw kritische
opmerkingen hebben mijn meer dan eens op de tippen van mijn tenen gehouden en
hebben op verschillende vlakken geleid tot betere resultaten.

TODO: assessoren

Maar ondanks de ongelooflijke steun van dit team, was deze masterproef nooit
gelukt zonder een nog groter team op het thuisfront. Drie jaar terug gaan
studeren overstijgt een klassieke dagtaak en legt een veel grotere dan verwacht
druk op een gezin. Kristien, mijn echtgenote, heeft elke seconde van de
afgelopen drie jaar, elke druppel zweet en tranen mee door gemaakt, ouders en
schoonouders hebben op de meest onmogelijke momenten ingesprongen om mij tijd
en ruimte te geven om dit huzarenstukje te realiseren. Meer dan eens moesten we
beroep doen op vrienden en buren om allerhande elementaire karwijen in mijn
plaats op te lossen. Het is hallucinant en hartverwarmend om te ervaren hoe
zoveel mensen meegeleefd hebben met mij.

Tot slot zijn er nog twee mensen die echt een hele dikke knuffel verdienen. Van
iedereen begrijpen zij misschien het minst wat papa gedaan heeft de afgelopen
jaren en waarom er zo weinig tijd voor hen overbleef. Eline en Arjen, mijn
lieve schatten, bedankt dat jullie soms zo begripvol waren en me dikwijls
opnieuw moed hebben gegeven om toch door te gaan.

\bigskip

Bedankt!

\end{preface}
