\documentclass[DIV=calc,paper=a4,fontsize=11pt,twocolumn,draft]{scrartcl}

\usepackage[dutch]{babel}
\usepackage[protrusion=true,expansion=true]{microtype}
\usepackage{amsmath,amsfonts,amsthm,amssymb}
\usepackage[svgnames]{xcolor}
\usepackage[hang,small,labelfont=bf,up,textfont=it,up]{caption}
\usepackage{booktabs}
\usepackage{fix-cm}
\usepackage{sectsty}
\allsectionsfont{\usefont{OT1}{phv}{b}{n}}

\usepackage{fancyhdr}
\pagestyle{fancy}
\usepackage{lastpage}

\lhead{\usefont{OT1}{phv}{m}{n} \footnotesize Deze ochtend vond ik een hacker in mijn koelkast}
\chead{}
\rhead{\usefont{OT1}{phv}{m}{n} \footnotesize TECHNOLOGIE \& BEVEILIGING} 
\lfoot{}
\cfoot{\usefont{OT1}{phv}{b}{n}\thepage}
\rfoot{}

\renewcommand{\headrulewidth}{0.0pt}
\renewcommand{\footrulewidth}{0.0pt}

\fancypagestyle{firststyle}
{
   \fancyhf{}
   \fancyfoot[C]{\usefont{OT1}{phv}{b}{n}\thepage}
}

\definecolor{VeryDarkGrey}{RGB}{64,64,64}
\definecolor{BaseColor}{RGB}{30,142,148}

\usepackage{lettrine}
\newcommand{\initial}[1]{
\lettrine[lines=4,lhang=0.3,nindent=0em]{
\color{VeryDarkGrey}
{\textsf{#1}}}{}}

\usepackage{titling}

\newcommand{\HorRule}{\color{BaseColor} \rule{\linewidth}{1pt}}

\pretitle{\vspace{-30pt} \begin{flushright} \HorRule \fontsize{50}{50}
\usefont{OT1}{phv}{b}{n} \color{BaseColor} \selectfont}

\title{Deze ochtend\\vond ik een hacker\\in mijn koelkast}

\posttitle{\par\end{flushright}\vskip 0.5em}

\preauthor{\begin{flushright}\large \lineskip 0.5em \usefont{OT1}{phv}{m}{n}
door }\author{Christophe Van Ginneken}
\postauthor{\end{flushright}\HorRule}

\date{}

\usepackage{ifdraft}
\usepackage{lineno}
\ifdraft{
  \setlength{\columnsep}{1cm}
  \linenumbers
}{}

\newcommand{\heading}[1]{
\subsubsection*{#1}
\vspace{-2mm}
}

\begin{document}
\twocolumn[
\begin{@twocolumnfalse}
\maketitle
\thispagestyle{firststyle}
\vspace{-1.5cm}
\begin{abstract}
\usefont{OT1}{phv}{m}{n}
De opkomst van het ``internet van de dingen'' is niet meer te stoppen en weldra
zal elk ding op aarde voorzien zijn van een eigen plek op het internet. Dit is
leuk, want zo zullen we van op het werk kunnen nakijken hoeveel melk er nog in
onze koelkast staat of kunnen we de verwarming van de badkamer alvast een
graadje hoger zetten terwijl we nog in de file zitten op een druilerige
vrijdagavond. Maar terwijl wij uitkijken naar een lekker warm bad, staat de
voordeur van ons geautomatiseerd huis ook open voor anderen, met minder
aangename bedoelingen en een massa aan mogelijkheden om al onze online
\emph{dingen} aan te vallen. Zullen we ons kunnen verdedigen, of is de strijd
bij voorbaat al verloren?

\end{abstract}
\vspace{0.5cm}
\end{@twocolumnfalse}
]

\initial{D}\emph{eze ochtend vond ik een hacker in mijn koelkast}. Het lijkt
een deel van een dialoog uit een slechte science fiction film. Als we echter de
evolutie van technologie van naderbij bekijken, zien we dat de realiteit
misschien sneller dan verwacht de fictie zal inhalen.

Nu het internet er voor gezorgd heeft dat elke computer - en ondertussen ook
bijna elke telefoon - ter wereld online is, voelen we reeds de hete adem van de
volgende revolutie in onze nek. Opnieuw wordt de schaal weer kleiner en wil men
kost wat kost elk ding ter wereld aansluiten op internet.

Elk ding mogen we hier eigenlijk zelfs letterlijk nemen. Onder de noemer van
het ``internet van de dingen'' wordt al enkele jaren getracht om elk toestel in
ons huis te voorzien van een aansluiting naar het internet. De eerste stap was
de introductie van digitale televisie. Deze ``digiboxen'' zijn in essentie
kleine computers verbonden met de televisiedistributeur via het internet. Via
interactieve programmagidsen en spelletjes, verrijken ze onze
televisie-ervaring met een ``druk op de rode knop''.

De makers van televisietoestellen konden niet achter blijven en al snel zat er
ook in onze televisietoestellen een netwerkaansluiting en konden we surfen
zonder ook maar van scherm te veranderen. Via diezelfde televisietoestellen
vernemen we vandaag dat we dankzij onze electriciteitsleverancier nu ook het
energieverbruik van elk toestel in ons huis kunnen controleren en indien nodig
onze hond even een lesje leren.

Het succes van deze mogelijkheden werkt duidelijk aanstekelijk en van
verschillende kanten hoor je verhalen over hoe fijn het zou zijn als we ons
hele leven zouden kunnen besturen via het internet. Waarom zouden we immers
onze koelkast niet op het internet aansluiten en deze de mogelijkheid bieden om
ons te laten weten dat we nog melk moeten halen of dat die schimmelkaas nu echt
wel ... aan vervanging toe is.

\heading{Een wereld vol microcontrollers en sensoren}

De onderliggende technologie die dit alles mogelijk maakt is de
microcontroller. Een microcontroller is in essentie een zeer kleine computer in
de vorm van \'e\'en enkele chip. Zo bevat hij alles wat nodig is om er software
op te plaatsen en uit te voeren: rekenkracht, geheugen en aansluitingspunten
voor allerhande externe componenten.

Microcontrollers worden in eerste plaats ontwikkeld om andere toestellen te
besturen en trachten daarom zo weinig mogelijk aanspraak te maken op externe
bronnen. De belangrijkste bron voor elk toestel is natuurlijk energie. Een laag
energieverbruik gaat echter wel gepaard met een lagere rekenkracht, maar dit is
doorgaans geen probleem omdat ze ingezet worden voor een goed gedefini\"eerde
taak.

De schaal waarop microcontrollers werken zorgt er ook voor dat ze relatief
goedkoop zijn en in grote getale worden ingezet. In een luxewagen zitten
vandaag al snel enkele honderden microcontrollers, die elk instaan voor \'e\'en
van de vele snufjes die onze rit aangenamer en veiliger maken.

Maar microcontrollers op zich zijn echter nutteloos. Ze hebben invoer nodig om
hun taken te kunnen vervullen en in tegenstelling tot hun grote broer in onze
computer thuis, beschikken ze niet over een toetsenbord en een muis om ze te
vertellen wat ze moeten doen. Daarom beschikken microcontrollers over
aansluitingspunten voor externe componenten die hun toelaten om informatie uit
de omringende omgeving op te nemen en toestellen aan te sturen.

De externe componenten die microcontrollers in staat stellen om hun omgeving
waar te nemen zijn sensoren. Deze elektronische schakelingen kunnen
omgevingsfactoren detecteren en omzetten in elektrische signalen. De
microcontroller kan vervolgens deze elektrische signalen meten en hieruit een
waarde bepalen voor bijvoorbeeld de intensiteit van het licht of de vochtigheid
in een kamer.

Het is deze combinatie van grote hoeveelheden sensoren en microcontrollers die
de evolutie van de automobielsector vooruit stuwt en ons reeds vandaag auto's
aanbiedt die autonoom kunnen parkeren, of zelfs remmen wanneer we dit zelf
dreigen te laat te doen.

\heading{Draadloze sensornetwerken}

De ontwikkelingen omtrent microcontrollers en sensoren zijn de laatste jaren
enorm ge\"evolueerd. Een zelfde evolutie kunnen we optekenen bij draadloze
technologie\"en. Mobiel internet is een dagdagelijks gegeven geworden en we
voelen ons bijna naakt als we 's morgens op weg naar werk niet even onze
Facebook status kunnen aanpassen of het gedrag van anderen kunnen
\emph{tweeten}.

De combinatie van deze twee werelden heeft geleid tot de ontwikkeling van
draadloze sensornetwerken. Dit zijn grote hoeveelheden microcontrollers met
sensoren die dankzij draadloze technologie met elkaar kunnen communiceren en zo
een netwerk cre\"eren van zgn. sensorknopen.

De inzetbaarheid kent vele vormen. Zo kunnen overstromingsgebieden in het oog
gehouden worden of kan men het trek- en kuddegedrag van dieren op de voet
volgen. Dankzij hun kleine vormgeving en zeer lage energienoden, kunnen deze
sensorknopen soms wel tot meer dan een jaar functioneren aan de hand van
\'e\'en enkele batterij. Dankzij hun lage kostprijs worden ze dan ook beschouwd
als een wegwerpbaar goed en zullen ze typisch overvloedig ingezet worden. Zo
zal het uitvallen van \'e\'en knoop impact hebben op de algemene werking van
het netwerk, omdat de overige knopen zorgen voor voldoende redundantie en de
taken van de uitgevallen knoop gewoon kunnen overnemen.

Het hoeft dus geen betoog dat draadloze sensornetwerken een interessante
basiscomponent bieden om nog meer technologische luxe te kunnen bouwen.

\heading{De realiteit achtervolgt de fictie}

We schrijven midden augustus 2013. Ergens in de Verenigde Staten van Amerika,
leggen twee jonge ouders hun kind te slapen onder het alziende oog van hun
nieuwe draadloze, met het internet verbonden, babyfoon. Wanneer zij enige tijd
later de kamer van het kind opnieuw betreden, horen ze een onbekende stem
obscene woorden ten berde brengen langs deze babyfoon, tot groot jolijt van de
kleine spruit.

Ondertussen wordt in Europa duchtig verder gesleuteld aan de draadloze
pacemaker. Een wonderbaarlijk stukje technologie dat artsen toegang geeft tot
het hart van hun pati\"ent, waar deze zich ook bevindt. Het lijkt wel een scene
uit een fictie-serie waarin een hacker zich toegang verschaft tot zo'n
pacemaker en zo de drager ervan vermoordt. Fictie? Dick Cheney denkt het in
ieder geval niet. In oktober 2013 heeft hij immers het draadloze aspect van
zijn pacemaker laten verwijderen, om een mogelijke terroristische aanval te
vermijden.

Toen we in 1995 genoten van de eerste acteerprestaties van Angelina Jolie in de
cultfilm ``Hackers'', leek het kunnen besturen van verkeerslichten een prachtig
staaltje science fiction. Bijna 20 jaar later, zijn verkeerslichten en
draadloze sensornetwerken in wetenschappelijke publicaties alvast dikke
vrienden en is men klaar om onder het mom van zgn. ``slimme steden'', elk van
deze lichten een autonoom leven te bieden. De grote omarming door het
``internet van de dingen'' is nu reeds alomtegenwoordig.

\heading{De hacker uit de koelkast halen}

Misschien lijkt het dat een hacker weinig kwaad kan doen in onze koelkast. Maar
wat indien we morgenvroeg plots merken dat de yoghurt die we aan onze kinderen
geven eerder een groene kleur vertoont omdat onze koelkast nagelaten heeft ons
te verwittigen dat de vervaldatum verstreken was of dat er eigenlijk nog
voldoende flessen melk waren en dat we er onderweg naar huis dus geen hadden
moeten kopen nadat we een bericht hadden ontvangen van onze koelkast dat de
voorraad op was? Toevallig was er die dag tevens een super interessante
promotie van een nieuw merk van zuivelproducten dat je dan toch maar eens kon
proberen. Eens deze technologie zijn intrede doet in ons dagdagelijks leven,
zullen we er stilaan op vertrouwen en zal de kleinste fout grote gevolgen
hebben.

Zelfs buurman Jan kan niet voorbij aan de voorbeelden en stelt zich ondertussen
terecht de vraag: ``Ok, en wat nu?''. Als we toch willen genieten van al dat
moois, maar we ook willen dat we onze kinderen een veilig ontbijt kunnen
aanbieden is de beveiliging van deze draadloze sensornetwerken een noodzaak.

Onder beveiliging verstaan we meestal de eerste doelstelling: het voorkomen dat
iets misgaat. Maar beveiliging gaat verder dan dat. Niet alles kan voorkomen
worden. In het geval van een inbraak zal men soms genoegen moeten nemen met het
in staat zijn om de inbraak vast te stellen om zo toch nog na de feiten
reactieve maatregelen te nemen. In de digitale wereld is dit schering en inslag
- denken we maar aan de recente onthullingen inzake de spionage in
verschillende telecombedrijven - en specifieke inbraakdetectiesystemen worden
dan ook in grote getale ingezet om op zijn minst de inbraak vast te stellen en
de schade te kunnen opmeten.

Dit is zonder meer ook de situatie bij draadloze sensornetwerken. Ook hier
zullen we soms genoegen moeten nemen met het kunnen vaststellen van een
inbraak. Maar het huwelijk van sensorknopen en inbraakdetectie blijkt al snel
te stranden op basis van tegenstrijdige belangen.

Een sensorknoop is nagenoeg gedurende zijn volledige levensloop aan zijn lot
overgelaten. De enige link met de buitenwereld is het draadloze netwerk dat hem
toelaat te communiceren met andere knopen. Diezelfde knopen zijn tevens zijn
enige communicatiekanaal met de wereld buiten het netwerk omdat communicatie
over langere afstanden doorheen het netwerk van knopen vloeit.

Gedurende deze hele periode dient de knoop energie te halen uit \'e\'en enkele
batterij en moet dus zeer spaarzaam omspringen met deze bron. Omdat knopen
tevens in grote getalen worden ingezet en ze feitelijk beschouwd dienen te
worden als wegwerpproducten, dient hun kostprijs zo laag mogelijk gehouden te
worden. Ze zijn dan ook voorzien van maar net genoeg geheugen en
verwerkingskracht om hun veelal beperkte taak uit te voeren.

Inbraakdetectie daarentegen vraagt opvolging. De meerderheid van de alarmen die
door een dergelijk systeem worden gegenereerd moeten eigenlijk bijna altijd nog
ge\"interpreteerd worden. Aangezien bijna elke aanval verschillend is, zijn de
mogelijkheden om te detecteren ook vaak eindeloos en wil men een
inbraakdetectiesysteem constant laten werken. Om zijn werk te kunnen doen,
dient het bij voorkeur ook te beschikken over veel gegevens. Deze gaan van
aanvalspatronen tot modellen van normaal gedrag om anomalie\"en te kunnen
detecteren. Tot slot is, vanuit het oogpunt van een knoop, het detecteren van
inbraakpogingen een niet-functionele, bijkomende belasting.

Het is snel duidelijk dat de afweging tussen knoop en detectie neerkomt op een
gebruik van de middelen waarover een knoop beschikt. In een ideale wereld zou
een inbraakdetectiesysteem voor draadloze sensorknopen de levensduur van de
batterij van de knoop niet mogen be\"invloeden, maar dat is spijtig genoeg
echte science fiction.

Onderzoeksliteratuur omtrent inbraakdetectie in draadloze sensornetwerken
beschrijft onnoemelijk veel manieren om specifieke aanvallen te detecteren. Een
enkele uitzondering waagt zich aan een combinatie van patronen of stelt een
raamwerk voor om enkele patronen te combineren. De algemene conclusie is echter
unaniem: gegeven de beperkte mogelijkheden van een knoop, is het onmogelijk om
een volledige dekking te bieden betreffende inbraakdetectie.

Meer zelfs, omdat een aanvaller nagenoeg te allen tijde in staat is om een
knoop fysiek te benaderen, het geheugen van de knoop te raadplegen en te
wijzigen, is er per definitie geen enkele mogelijkheid om elke poging tot
inbraak te verijdelen zonder fysieke uitbreiding van een knoop met bijkomende,
specifieke hardware. Deze zou echter de prijs van een knoop meerdere malen
vermenigvuldigen en zich zelf daarmee uit de markt prijzen.

Het beveiligen van draadloze sensornetwerken komt daarmee neer op een afweging
van risico's. Hierbij zal een inschatting moeten gemaakt worden welke mogelijke
aanvallen we willen onderscheppen om zoveel mogelijke barri\`eres op te werpen
om de meerderheid van de aanvallers te ontmoedigen.

\heading{In de praktijk}

Indien het probleem onoverkomelijk is, rest er het draaglijker maken van de
pijn. Indien we in staat zijn om de impact van de inbraakdetectie te
verlichten, kan er op meer aanvallen gecontroleerd worden en kan de maker van
het sensornetwerk een ruimere keuze maken uit de bestaande
detectiemogelijkheden, waardoor de drempel voor de aanvaller toch weer een
beetje hoger wordt.

De ontwikkelaar van de software van een draadloze sensorknoop kan zelf
onderzoeksliteratuur raadplegen en \'e\'en of meerdere van de beschreven
detectiemechanismen trachten te implementeren. Hierbij zal hij typisch, voor
elk van deze detectoren, een gelijkaardige blok programmacode maken dat
enerzijds actief wordt bij het ontvangen van nieuwe communicatie uit het
netwerk en anderzijds tussendoor de verzamelde informatie evalueert en
beslissingen neemt omtrent de opgetekende situatie.

Aangezien \'e\'en enkele knoop zelden een aanval of inbraak kan detecteren, is
communicatie met de andere knopen een noodzakelijk kwaad. Communicatie tussen
knopen is tevens de belangrijkste bron van informatie voor de knoop om zich een
beeld te vormen van wat er zich rondom hem afspeelt. Nagenoeg elk
detectiesysteem zal bij ontvangst van gegevens via het netwerk de inhoud ervan
moeten inspecteren om zich van nieuwe informatie over zijn omgeving te
vergewissen.

Anderzijds dient ook op regelmatige tijdstippen een inventaris gemaakt te
worden van de verzamelde gegevens. Dit kan gaan van het controleren of er geen
andere knopen zijn die reeds geruime tijd geen activiteit hebben vertoond, tot
het berekenen van een reputatie van een knoop op basis van de verschillende
gebeurtenissen in het netwerk en de informatie die andere knopen hieromtrent
gedeeld hebben.

Het is evident dat deze zeer gelijklopende structuur voor elk
detectiemechanisme onherroepelijk kan leiden tot veel dezelfde programmacode,
maar ook tot meerdere malen het uitvoeren van diezelfde code en talrijke
berichtenuitwisselingen over het netwerk tussen de knopen, en dit voor elk van
de detectoren afzonderlijk. Veelvuldige uitvoering leidt tot langere
verwerkingstijden en veel communicatie leidt tot meer gebruik van het draadloze
netwerk.

Voor eenvoudige detectoren lijkt dit een artificieel probleem. Elke zichzelf
respecterende ontwikkelaar zal dit opmerken en zal de code zo structureren dat
deze problemen weggewerkt worden. Inderdaad, voor eenvoudige systemen \'en
indien de ontwikkelaar de volledige inbraakdetectie zelf bouwt, is dat het
geval. Maar inbraakdetectie is een niet-functioneel gegeven voor de
ontwikkelaar. Om inbraakdetectie schaalbaar inzetbaar te maken, zal men ook
hier zoveel mogelijk gebruik willen maken van bestaande implementaties. Op dat
ogenblik heeft de ontwikkelaar niet langer de luxe om de code anders te
organiseren en zullen de langere uitvoertijden en het veelvuldige
netwerkgebruik effectief een overmatige belasting worden op de beperkte
mogelijkheden van de sensorknoop.

Men kan echter nog verder argumenteren dat de de bijkomende verwerkingstijden
bijna verwaarloosbaar zijn voor microcontrollers die amper 0.4 milli-amp\`ere
stroom verbruiken wanneer ze actief zijn - een peulschil in vergelijking met
hun grote broer in onze computer die al gauw 10 volledige amp\`eres vraagt, een
25000-voud. Dit is correct, maar in dat geval mag men niet uit het oog
verliezen dat een typische draadloze radio al snel 40mA verbruikt bij het
verzenden en ontvangen van communicatie. Dit is een factor 100 ten opzichte van
de microcontroller. Een veelgebruikte oplaadbare batterij, zoals deze in GSM
toestellen, biedt gangbaar ongeveer 1700mAh. Een actieve draadloze radio
verbruikt al snel 40mA en zal de energie van deze volledige batterij op iets
minder dan 2 dagen opgebruiken. Het beperken van het gebruik ervan is dus een
zeer belangrijke noodzaak.

\heading{Code die code genereert}

De ontwikkelaar van software voor een sensorknoop zou dus bij voorkeur een
groot aantal bestaande detectiesystemen willen verzamelen en deze zo eenvoudig
mogelijk en goed georganiseerd opnemen in zijn eigen code.

Ondanks de enorme ontwikkelingen op het vlak van taaltechnologie, is dit echter
nog steeds nagenoeg onmogelijk. De verschillen in stijl van programmeren van
verschillende ontwikkelaars en de brede waaier aan mogelijkheden die aangewend
kunnen worden, gecombineerd met de complexheid van de algoritmen, maakt dat het
technisch niet realistisch is om bestaande programmacode van verschillende
detectiemechanismen te nemen en automatisch om te zetten tot beter
georganiseerde code.

Een andere aanpak bestaat er in om de beschrijving van een detector te
realiseren aan de hand van een domeinspecifieke taal. Zo'n taal is opgebouwd
rond een specifiek domein, in dit geval dat van inbraakdetectie, en voorziet
daarvoor een typisch en vertrouwd jargon. Dit type van talen is, in
tegenstelling tot normale, generieke programmeertalen, beperkt in zijn
expressiviteit en bevat bv. niet de mogelijkheid om een klassieke iteratie te
beschrijven.

Deze beperking leidt er toe dat de beschrijving van een detector nog nauwelijks
op meerdere manieren kan gerealiseerd worden - of alvast niet op een ongunstige
manier - waardoor een geautomatiseerde verwerking sterk vereenvoudigd wordt.
Hierdoor wordt \'e\'en van de basisproblemen ontweken en ligt de weg open om
verschillende mechanismen te combineren en optimaal te organiseren. Dit kan
gebeuren door middel van specifieke codegeneratiesoftware die de
domeinspecifieke taal analyseert en omzet in effectieve programmacode.

Het aspect van een domeinspecifieke taal is hier wel een essentieel gegeven.
Net door de taal toe te spitsen op \'e\'en specifiek domein wordt het mogelijk
om de taal zo te beperken dat de resulterende beschrijvingen er in nagenoeg
eenduidig zijn. Codegeneratie kan al veel van de taken van programmeurs uit
handen nemen, doch is vandaag nog niet in staat om algemene software te
produceren op basis van een algemene functionele beschrijving.

Door aan de hand van codegeneratie de code optimaal te organiseren, wordt het
mogelijk om de verschillende detectoren parallel te laten werken en niet langer
sequentieel. Dit leidt ook tot de mogelijkheid om de communicatie die de
verschillende mechanismen versturen te bundelen in \'e\'en bericht, waardoor
het herhaaldelijk gebruik van het draadloze netwerk tot een minimum gereduceerd
wordt en de broodnodige optimalisatie van het energieverbruik kan gerealiseerd
worden.

Het samennemen van verschillende detectoren en het combineren van communicatie
zijn slechts twee van de mogelijkheden die zich aanbieden. Zo kan ook het
geheugengebruik geoptimaliseerd worden door het delen van veranderlijke
gegevens tussen detectoren mogelijk te maken. Ook wordt het dankzij
codegeneratie mogelijk om een aantal aspecten om te zetten in meer technische
code, waar een menselijke programmeur dikwijls voorrang geeft aan leesbaarheid
en onderhoudbaarheid.

\heading{De toekomst}

De noden en mogelijkheden van inbraakdetectie enerzijds en van draadloze
sensornetwerken anderzijds, zijn elkaars concurrenten. Draadloze sensorknopen
zijn enorm beperkt in hun mogelijkheden en de impact van een degelijke
inbraakbeveiliging hypothekeert nagenoeg de volledige functionaliteit van de
knoop.

Ofschoon de fysieke toegankelijkheid van sensorknopen leidt tot een situatie
waar het feitelijk onmogelijk is om een inbraak te detecteren, blijft het
belangrijk om deze netwerken toch te voorzien van een vorm van inbraakdetectie.
Hoe meer aanvallen een netwerk van zulke knopen kan detecteren, hoe moeilijker
het wordt voor een aanvaller om zich vlot meester te maken van het netwerk en
er malafide praktijken mee te ondersteunen.

Naast het onderzoeken van aanvallen en het beschrijven van mechanismen om deze
te detecteren, is er nog een andere piste die bewandeld kan worden.
Domeinspecifieke talen en codegeneratie kunnen aangewend worden om de
consequenties die de implementatie van inbraakbeveiliging met zich meebrengt te
verlichten.

Aan de hand van codegeneratie kunnen de algoritmen onafhankelijk beschreven
worden en toch zo georganiseerd worden dat de uitvoeringstijd geminimaliseerd
wordt. Verder wordt ook het gebruik van de netwerkcommunicatie gegroepeerd in
een enkele gebundelde zending van de informatie van alle verschillende
algoritmen samen.

Zo resulteert deze aanpak in een automatisering van de implementatie van
inbraakdetectiemechanismen, waardoor de flexibiliteit om code te organiseren
behouden blijft en het mogelijk wordt om de impact van deze bijkomende
niet-functionele code te minimaliseren. De onafhankelijkheid van de
domeinspecifieke taal biedt verder tevens de mogelijkheid om
platformonafhankelijke beschrijvingen van inbraakdetectiealgoritmes te hanteren
binnen het heterogene landschap van de draadloze sensornetwerken.

Zo heeft de ontwikkelaar van de software van de sensorknopen die weldra
alomtegenwoordig zullen zijn in onze huizen, alvast de mogelijkheid om snel en
effectief meerdere inbraakdetectiemechanismen toe te voegen aan zijn eigen
functionele code. Laat ons hopen dat we op die manier morgen met een gerust
hart een pak melk uit de koelkast kunnen nemen. $\color{BaseColor}\blacksquare$

\vspace{0.6cm}

\bf{\emph{Christophe Van Ginneken studeert computerwetenschappen aan de
universiteit van Leuven.}}

\end{document}