%!TEX root=masterproef.tex

\subsection{Attesteren van software}
\label{subsection:attestation}

Het voorbeeld uit bijlage \ref{appendix:node-capture} toont aan dat zelfs het
vluchtige geheugen van een knoop niet veilig is. Als een aanvaller in staat is
om ongemerkt de programma-code van een knoop te bekomen, alsook alle gegevens
die zich alleen tijdens uitvoering in het geheugen bevinden, dan kan deze
aanvaller deze code aanpassen zodat de werking ogenschijnlijk ongewijzigd is,
maar dat hij toch controle heeft over de werking en zo het hele netwerk kan
be\"invloeden.

Een zeer logische onderzoeksvraag dient zich al snel aan: ``\emph{Is het
mogelijk om wijzigingen aan het programma van een knoop in het netwerk vast te
stellen?}''. Deze vraag wordt onderzocht binnen het domein van software
attestatie.

\subsubsection*{Werking}

Alle bestaande vormen van software attestatie maken gebruik van een protocol
gebaseerd op het challenge response principe. Als men de integriteit van een
knoop wil vast stellen, zal men aan deze knoop een verzoek sturen om een unieke
samenvatting te maken van zijn inhoud door middel van een cryptografische
hashfunctie, een \emph{checksum}.

De vaststeller beschikt zelf over een versie van de inhoud van de knoop en kan
dezelfde unieke samenvatting berekenen. Door in het initi\"ele verzoek een
\'e\'enmalig te gebruiken code mee te geven, een zgn. \emph{nonce}, en deze
deel te laten uitmaken van de inhoud, kunnen verschillende verzoeken telkens
met een ander, unieke samenvatting beantwoord worden en wordt kan deze
samenvatting niet op voorhand gekend en berekend worden.

De inhoud waarvan een samenvatting gemaakt wordt is typisch de programma code
die op de knoop ge\"installeerd werd. Indien een aanvaller deze code kon
wijzigen, zou de samenvatting niet langer overeenkomen met die opgesteld door
de vaststeller en kan deze laatste besluiten om deze gewijzigde code niet te
vertrouwen en de knoop uit te sluiten.

Bijlage \ref{appendix:attestation} belicht een voorbeeld van een software
attestatie algoritme, ICE, en evalueert het met betrekking tot zijn
mogelijkheden en beperkingen. Ook wordt kort ingegaan op de zgn. ``Trusted
Platform Module'' (TPM), op het concept van gedistribueerde attestatie en wordt
geconcludeerd dat het attesteren van software mogelijk is, maar met de nodige
omzichtigheid moet aangepakt worden. Veelal zal blijken dat bijkomende
infrastructuur zal moeten voorzien worden.

Deze masterproef laat het attesteren van software buiten beschouwing en gaat
uit van een situatie waar een knoop op \'e\'en of andere manier \emph{veilig}
kan beschouwd worden op dit vlak. Het is echter wel belangrijk om deze
concepten op zijn minst te kaderen in het volledige spectrum.
