%!TEX root=masterproef.tex
\subsection{Attesteren van software}
\label{subsection:attestation}

Het voorbeeld uit sectie \ref{section:node-capture} toonde al aan dat zelfs het
vluchtige geheugen van een knoop niet veilig is. Als een aanvaller in staat is
om ongemerkt de programma-code van een knoop te bekomen, alsook alle gegevens
die alleen tijdens uitvoering in het geheugen, dan kan deze aanvaller deze code
aanpassen zodat de werking ogenschijnlijk ongewijzigd is, maar dat hij toch
controle heeft over de werking en zo het hele netwerk kan be\"invloeden.

Een zeer logische onderzoeksvraag dient zich al snel aan: ``\emph{Is het
mogelijk om wijzigingen aan het programma van een knoop in het netwerk vast te
stellen?}''. Deze vraag wordt onderzocht binnen het domein van software
attestatie.

\cite{castelluccia2009difficulty}
