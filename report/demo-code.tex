%!TEX root=masterproef.tex

\chapter{FOO-lang broncode van selectie detectiealgoritmen}
\label{appendix:demo-code}

De in FOO-lang beschreven voorbeelden van detectiealgoritmen vormen de basis
voor de evaluatie van het prototype dat in deze masterproef ge\"implementeerd
werd.

\section{\emph{Heartbeat}}
\label{section:demo-code-heartbeat}

\inputminted[linenos,frame=lines,framesep=2mm,fontsize=\footnotesize]{js}{../src/foo-lang/examples/heartbeat.foo}
\vspace{-5mm}
\captionof{listing}{\ttt{heartbeat.foo}
  \label{lst:heartbeat.foo}}

\section{Reputatie}
\label{section:demo-code-reputation}

\inputminted[linenos,frame=lines,framesep=2mm,fontsize=\footnotesize]{js}{../src/foo-lang/examples/reputation.foo}
\vspace{-5mm}
\captionof{listing}{\ttt{reputation.foo}
  \label{lst:reputation.foo}}

\section{Samenwerking}
\label{section:demo-code-cooperation}

De volgende FOO-lang broncode is een experimenteel document dat diende als
oefening om na te gaan hoeveel aanpassingen aan FOO-lang zouden moeten gebeuren
om een derde algoritme te kunnen beschrijven.

De broncode bevat aanwijzingen naar en bedenkingen omtrent bepaalde
oplossingen. Veel van de opmerkingen hebben betrekking op de implementatie van
de code generator. Slechts een minderheid aan echte aanpassingen noodzaken een
wijziging aan FOO-lang.

\inputminted[linenos,frame=lines,framesep=2mm,fontsize=\footnotesize]{js}{../src/foo-lang/examples/cooperation.foo}
\vspace{-5mm}
\captionof{listing}{\ttt{cooperation.foo}
  \label{lst:cooperation.foo}}
