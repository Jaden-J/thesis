%!TEX root=masterproef.tex

\begin{abstract}

Het introduceren van inbraakdetectie in draadloze sensornetwerken resulteert al
snel in een gevecht voor middelen: een draadloze sensorknoop beschikt over een
beperkte autonomie en moet zijn energie optimaal benutten. Bijkomende
niet-functionele inbraakbeveiliging vraagt veel van deze middelen en bedreigt
daarmee zijn eigen kans om opgenomen te worden in het uiteindelijke ontwerp van
elk nieuwe draadloze sensorknoop.

Omdat draadloze sensornetwerken met rasse schreden onze persoonlijke
levenssfeer binnentreden, moeten we echter een degelijke beveiliging eisen.
Preventie is een eerste stap, maar niet alle inbraken kunnen verijdeld worden.
Soms moeten we genoegen nemen het kunnen detecteren van inbraken om ons er in
toekomstige situaties eventueel beter tegen te wapenen.

Indien het probleem niet kan vermeden worden, moeten we trachten het
draaglijker te maken. Deze masterproef wil zowel de druk op de middelen van de
sensorknopen verlichten als de bijkomende economische druk op de ontwikkeling,
die de introductie van inbraakdetectie met zich meebrengt, zoveel mogelijk
wegnemen.

Om dit te realiseren stelt deze masterproef daarom een domeinspecifieke taal
voor die onderzoekers in staat stelt om algoritmen voor inbraakdetectie op een
formele en platformonafhankelijke manier te defini\"eren. Deze eerste stap
ontslaat ontwikkelaars van nieuwe sensornetwerken van de taak om
onderzoeksliteratuur te doorworstelen en te trachten de algoritmen uit deze
teksten te puren.

Een formele beschrijving laat ook toe om deze op een geautomatiseerde wijze te
benaderen. Zo wordt het mogelijk om door middel van codegeneratie de algoritmen
automatisch om te zetten in platformspecifieke programmacode. Codegeneratie
stelt ons vervolgens in staat om de programmacode op zo'n manier te organiseren
dat de middelen van de sensorknoop zo optimaal mogelijk benut worden.

Dankzij het vrijwaren van de middelen van de sensorknoop en het reduceren van
de economische kost, wordt het mogelijk om meer inbraakpogingen te detecteren.

\end{abstract}
