%!TEX root=masterproef.tex

\begin{abstract}

Draadloze sensornetwerken treden met rasse schreden onze persoonlijke
levenssfeer binnen. Een afdoende beveiliging tegen inbraken moet garanties
kunnen bieden dat deze vooruitgang zelf geen bedreiging wordt. Preventie is de
eerste stap, maar niet alle inbraken kunnen verijdeld worden. Soms moeten we
genoegen nemen met het kunnen detecteren van inbraken om ons in de toekomst er
beter tegen te wapenen.

Het introduceren van inbraakdetectie in draadloze sensornetwerken resulteert al
snel in een gevecht om middelen: een draadloze sensorknoop beschikt over een
beperkte autonomie en moet zijn energie optimaal benutten. Hieraan
inbraakbeveiliging toevoegen vraagt veel van de beschikbare middelen en
bedreigt daarmee de kans om opgenomen te worden in het uiteindelijke ontwerp
van elk nieuwe draadloze sensorknoop.

Indien het probleem niet kan vermeden worden, moeten we trachten het
draaglijker te maken. Deze masterproef wil zowel de druk op de middelen van de
sensorknopen verlichten als de bijkomende economische druk op de ontwikkeling,
die de introductie van inbraakdetectie met zich meebrengt, reduceren.

Om dit te realiseren wordt een domeinspecifieke taal voorgesteld die
onderzoekers in staat stelt om algoritmen voor inbraakdetectie op een formele
en platformonafhankelijke manier te defini\"eren. Deze eerste stap ontslaat
ontwikkelaars van nieuwe sensornetwerken van de taak om onderzoeksliteratuur te
doorworstelen en algoritmen uit deze teksten te puren.

Een formele beschrijving laat verder toe om deze op geautomatiseerde wijze te
benaderen. Zo wordt het mogelijk om door middel van codegeneratie de algoritmen
automatisch om te zetten in platformspecifieke programmacode, z\'o
georganiseerd dat de middelen van de sensorknoop zo optimaal mogelijk benut
worden.

De initi\"ele testen met een prototype codegenerator zijn veelbelovend. Ze
bevestigen de intu\"itie dat een goede organisatie van verschillende
detectiealgoritmen kan leiden tot een beter gebruik van de middelen van een
sensorknoop \'en dat dit volledig geautomatiseerd kan gebeuren. Dankzij het
vrijwaren van de middelen van de sensorknoop en het reduceren van de
economische kost, wordt het zo mogelijk om meer inbraakpogingen te detecteren.

Zowel de domeinspecifieke taal als de code generator bieden opportuniteiten tot
verder onderzoek. Testen met detectiealgoritmen en platformen moeten op grotere
schaal uitgewerkt worden. De realisatie van een ecosysteem rond de
geformaliseerde detectiealgoritmen is een andere belangrijke richting die
nagestreefd moet worden en waar vooral het onderzoeksdomein baat bij heeft.

\end{abstract}
