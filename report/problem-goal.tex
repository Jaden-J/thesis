\documentclass[12pt,a4paper]{article}

\usepackage[margin=1in]{geometry}
\usepackage{graphicx}
\usepackage[english]{babel}
\usepackage{minted}
\usepackage{caption}
\usepackage{subcaption}

\usepackage{color}

\usepackage{url}

\usepackage{longtable}

\newcommand{\TODO}{\textbf{\color{red}TODO}}

\usepackage{enumitem}
\setlist{nolistsep}

% prefer dashes over ball-bullets
\renewcommand{\labelitemi}{$-$}

\usepackage{fancyhdr}
\pagestyle{fancy}
\fancyhead{}
\fancyfoot{}
\renewcommand{\headrulewidth}{0mm}
\rfoot{\thepage}    
\cfoot{}
\lfoot{Inbraakdetectie in draadloze sensornetwerken: Probleem- en doelstelling}

\author{Christophe Van Ginneken\\\url{Christophe.VanGinneken@student.kuleuven.be}}

\title{Inbraakdetectie in draadloze sensornetwerken: Probleem- en doelstelling}
\date{\today}

% http://tex.stackexchange.com/questions/299/how-to-get-long-texttt-sections-to-break
\newcommand*\justify{%
  \fontdimen2\font=0.4em% interword space
  \fontdimen3\font=0.2em% interword stretch
  \fontdimen4\font=0.1em% interword shrink
  \fontdimen7\font=0.1em% extra space
  \hyphenchar\font=`\-% allowing hyphenation
}

% http://en.wikibooks.org/wiki/LaTeX/Customizing_LaTeX
\newcommand{\ttt}[1]{{\tt \justify{#1}}}

\begin{document}

\maketitle

\section*{Inleiding}

Wanneer gerefereerd wordt naar de context van deze thesis, nl. inbraakdetectie
voor draadloze sensornetwerken, wordt meestal snel verwezen naar de anonieme
hacker die zich toegang kan verlenen tot de weerloze draadloze sensorknoop, hem
omtovert tot zijn eigendom en hiermee de goede werking van het netwerk in het
gedrag brengt. Ofschoon volledig correct, blijft het voor buurman Jan slechts
een abstract gegeven en blijft hij achter met de vraag ``Ja, ok, en dan?''. En
ofschoon de zwaarte van dit probleem nog enkele jaren in de toekomst ligt, zijn
er vandaag spijtig genoeg toch reeds voorbeelden te vinden.

\subsection*{De realiteit achtervolgt de fictie}

We schrijven midden augustus 2013. Ergens in de Verenigde Staten van Amerika,
leggen twee jonge ouders hun kind te slapen, onder het alziende oog van hun
nieuwe draadloze, met het internet verbonden babyfoon. Wanneer zijn enige tijd
later de kamer van het kind opnieuw betreden, horen zij dat langs deze babyfoon
een onbekende obscene woorden ten berde brengt.

Ondertussen wordt in Europa duchtig gesleuteld aan de eerste draadloze
pacemaker. Een wonderbaarlijk stukje technologie dat doktors toegang geeft tot
het hart van hun pati\"ent, waar zij zich ook bevinden. Het lijkt wel een scene
uit een fictie-serie waarin een hacker zich toegang verschaft tot zo'n
pacemaker en zo de drager ervan vermoord. Fictie? Dick Cheney denkt het in
ieder geval niet. In oktober 2013 heeft hij immers het draadloze aspect van
zijn pacemaker laten verwijderen, om een mogelijke terroristische aanval te
vermijden.

Toen we in 1995 genoten van de eerste acteerprestaties van Angelina Jolie in
Hackers, leek het kunnen besturen van verkeerslichten een prachtig staaltje
science fiction. Bijna 20 jaar later, zijn verkeerslichten en draadloze
sensornetwerken in wetenschappelijke publicaties alvast dikke vrienden en is
men klaar om onder het mom van de zgn. ``Smart City'', elk van deze lichten een
autonoom leven te bieden. De grote omarming door het ``Internet of Things'' is
nu reeds alom tegenwoordig.

Zelfs buurman Jan kan niet voorbij aan deze voorbeelden en stelt zich
ondertussen terecht de vraag: ``Ok, en wat nu?''. De feiten wat ze zijn en
weldra zal elk \emph{ding} op aarde wel een eigen IP adres hebben. Hoe ziet de
wereld er dan eigenlijk uit?

\section*{Probleemstelling}

Aan de ene kant hebben we de draadloze sensorknopen en aan de andere kan hebben
we inbraakdetectie. Ofschoon de ene de andere duidelijk nodig heeft om zich -
en uiteindelijk ons - te beschermen, blijkt snel dat ze tegenstrijdige belangen
dienen.

Een knoop is nagenoeg gedurende zijn volledige levensloop aan zijn lot
overgelaten. De enige link met de buitenwereld is zijn draadloze radio die hem
toelaat te communiceren met andere knopen en zo langs dit netwerk met zijn
thuis. Gedurende deze hele periode dient de knoop ook zijn energie te halen uit
\'e\'en enkele batterij en moet dus zeer spaarzaam omspringen met deze bron.
Omdat knopen tevens in grote getale worden ingezet en ze feitelijk beschouwd
dienen te worden als wegwerp producten, dient hun kostprijs laag gehouden te
worden. Ze zijn dan ook voorzien van net genoeg geheugen en verwerkingskracht
om hun veelal beperkte taak uit te voeren.

Inbraakdetectie, aan de andere kant, vraagt opvolging. Een meerderheid van de
alarmen die door een dergelijk systeem worden gegeneerd zijn nog steeds te
valideren door een menselijke analist. Aangezien dat daarnaast elke aanval wel
verschillend kan zijn, zijn de mogelijkheden om te detecteren ook vaak
eindeloos en wil men bij voorkeur een inbraakdetectie systeem constant laten
werken. Om zijn werk te kunnen doen, dient het bij voorkeur ook te beschikken
over veel gegevens. Deze gaan van aanvalspatronen tot modellen van normaal
gedrag om anomalie\"en te kunnen detecteren. Tot slot is het detecteren op zich
natuurlijk van uit het oogpunt van een knoop een niet-functionele bijkomende
belasting in het algemeen.

Het is snel duidelijk dat de afweging tussen knoop en detectie neerkomt op een
gebruik van de middelen waarover een knoop beschikt. In een ideale wereld zou
een inbraakdetectie systeem voor draadloze sensorknopen de levensduur van de
batterij van de knoop niet mogen be\"invloeden, maar dat is spijtig genoeg
echte science fictie.

\subsection*{Het onderzoeksdomein}

De literatuur omtrent inbraakdetectie in draadloze sensornetwerken beschrijft
onnoemelijk veel manieren om welbepaalde aanvallen te detecteren. Een
uitzondering waagt zich aan een combinatie van patronen of stelt een raamwerk
voor om meerdere patronen te combineren. De algemene conclusie is echter
unaniem: gegeven de beperkte mogelijkheden van een knoop, is het onmogelijk om
een volledige dekking te bieden betreffende inbraakdetectie. Nog meer zelfs,
omdat een aanvaller nagenoeg ten allen tijde in staat is om een knoop fysiek te
benaderen, het geheugen van de knoop te raadplegen en te wijzigen, is er per
definitie geen enkele mogelijkheid om een inbraak te verijdelen - zonder
fysieke uitbreidingen van een knoop met bijkomende specifieke hardware, welke
de prijs van een knoop zou vermenigvuldigen en zich zelf daarmee uit de markt
van de mogelijke oplossing prijst.

\section*{Doelstelling}

Hoe kan een master-thesis dan bijdragen aan dit probleem? Indien het probleem
onoverkomelijk is, rest er het draaglijker maken van het lijden. Indien we in
staat zijn om de impact van de inbraakdetectie te verlichten, kan er op meer
aanvallen gecontroleerd worden en kan de maker van het sensornetwerk een
ruimere keuze maken uit de bestaande detectie algoritmen, waardoor de drempel
voor de aanvaller weer een beetje hoger wordt.

\subsection*{Situatie}

Om een mogelijk bijdrage te defini\"eren aan deze situatie, bekijken we eerst
de situatie op het slagveld. De ontwikkelaar van de software van een draadloze
sensorknoop kan zelf de literatuur raadplegen en \'e\'en of meerdere van de
beschreven detectie-mechanismen trachten te implementeren. Hierbij zal hij
typisch, voor elk van deze detectoren, een zeer gelijkaardige blok code
produceren.

Aangezien een knoop zelden op zich een aanval kan detecteren, is communicatie
met de andere knopen een noodzakelijk kwaad. Tevens is activiteit op het
netwerk typisch de belangrijkste bron van informatie voor de knoop betreffende
wat er zich rondom hem afspeelt. Nagenoeg elk detectie algoritme zal bij
ontvangst van een netwerkpakket de inhoud ervan moeten inspecteren om zich van
nieuwe informatie over zijn omgeving te vergewissen.

Anderzijds dient elk algoritme ook op regelmatige tijdstippen een inventaris te
maken van de geaccumuleerde situatie. Dit kan gaan van het controleren of er
geen andere knopen zijn die reeds geruime tijd geen activiteit hebben vertoond,
tot het berekenen van een reputatie van een knoop op basis van de verschillende
gebeurtenissen in het netwerk en de informatie die andere knopen hieromtrent
hebben gedeeld.

Het is evident dat deze zeer gelijklopende structuur voor elk
detectie-mechanisme onherroepelijk kan leiden tot code duplicatie, maar ook tot
meerdere sequenti\"ele iteraties en talrijke communicatie tussen de knopen voor
elk van de algoritmen afzonderlijk. Sequenti\"ele uitvoering leidt tot langere
uitvoertijden en meerdere communicaties leidt tot meer gebruik van het netwerk.

\subsection*{Probleemevaluatie}

Voor eenvoudige algoritmen lijkt dit een artificieel probleem. Elke zichzelf
respecterende ontwikkelaar zal dit opmerken en zal de code zo structureren dat
deze redundanties weggewerkt worden. Inderdaad, voor eenvoudige algoritmen \'en
indien de ontwikkelaar zijn volledige inbraakdetectie zelf bouwt, is dat het
geval. Maar inbraakdetectie is een niet-functioneel gegeven voor deze
ontwikkelaar. Om inbraakdetectie schaalbaar inzetbaar te maken, zal ook hier
zoveel mogelijk gebruik gemaakt willen worden van bestaande implementatie. Op
dat ogenblik, heeft de ontwikkelaar niet langer de luxe om de code anders te
organiseren en zullen de langere uitvoertijden en het veelvuldige
netwerkgebruik effectief problemen worden.

Men kan echter nog verder argumenteren dat de de bijkomende verwerkingstijden
bijna verwaarloosbaar zijn voor microcontrollers die amper 0.4mA stroom
verbruiken wanneer ze actief zijn. Dit is waar. Maar in dat geval mag men niet
uit het oog verliezen dat een typische draadloze radio al snel 40mA verbruikt
bij verzenden en ontvangen. Dit is een factor 100 ten opzichte van de
microcontroller. Een veelgebruikte oplaadbare batterij, zoals deze in GSM
toestellen, biedt gangbaar ongeveer 1700mAh. Een actieve draadloze radio
verbruikt deze volledige batterij op minder dan anderhalve dag. Het beperken
van het gebruik ervan is dus een zeer belangrijke noodzaak.

\subsection*{Aanpak}

In deze thesis stel ik voor om de beschrijving van een detectie-algoritme te
realiseren aan de hand van een domein-specifieke taal. Deze taal is beperkt in
zijn expressiviteit en bevat bv. niet de mogelijkheid om een klassieke iteratie
te beschrijven. Hierdoor wordt \'e\'en van de basisproblemen ontweken en ligt
de weg open om op geautomatiseerde wijze verschillende van deze algoritme te
combineren.

Immers indien hergebruik van bestaande code niet mogelijk is omwille van de
hoger vermelde problemen, dient er op een andere manier een combinatie van
detectie-mechanismen gerealiseerd te worden. Deze thesis stelt daarom voor om
op basis van de beschrijvingen in de domein-specifieke taal, door middel van
code generatie de nodige code voor inbraakdetectie te genereren op maat van de
ontwikkelaar, net zoals hij dit zou doen. Op deze manier kan nog steeds de
beschrijving opnieuw gebruikt worden, zonder dat dit tot redundant werk leidt.

\section*{Conclusie}

De noden en mogelijkheden van inbraakdetectie en draadloze sensornetwerken
kunnen niet meer concurrenten zijn van elkaar. Draadloze sensorknopen zijn
enorm beperkt in hun mogelijkheden en de impact van een degelijke
inbraakbeveiliging hypothekeert nagenoeg de volledige functionaliteit van de
knoop.

Ofschoon de fysieke toegankelijkheid van sensorknopen leidt tot een situatie
waar het feitelijk onmogelijk is om een inbraak te detecteren, blijft het
belangrijk om deze netwerken toch te voorzien van een vorm van inbraakdetectie.
Hoe meer aanvallen een netwerk van zulke knopen kan detecteren, hoe moeilijker
het wordt voor een aanvaller om zich vlot meester te maken van het netwerk en
er malafide praktijken mee te ondersteunen.

Naast het onderzoeken van aanvallen en het beschrijven van algoritmen om deze
te detecteren, is er nog een andere piste die bewandeld kan worden. In deze
thesis wordt een mogelijkheid voorgesteld om de problemen die de implementatie
van inbraakbeveiliging met zich meebrengt te verlichten.

Aan de hand van code generatie kunnen de algoritmen onafhankelijk beschreven
worden en toch mooi georganiseerd worden zodat de uitvoeringstijd
geminimaliseerd wordt en dat het gebruik van het netwerk gegroepeerd wordt in
\'e\'en enkele verzending van de informatie van de verschillende algoritmen.

\end{document}
