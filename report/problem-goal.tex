\documentclass[12pt,a4paper,draft]{article}

\usepackage[margin=1in]{geometry}
\usepackage{graphicx}
\usepackage[dutch]{babel}
\usepackage{minted}
\usepackage{caption}
\usepackage{subcaption}

\usepackage{color}

\usepackage{url}

\usepackage{longtable}

\usepackage{lineno}

\newcommand{\TODO}{\textbf{\color{red}TODO}}

\usepackage{enumitem}
\setlist{nolistsep}

% prefer dashes over ball-bullets
\renewcommand{\labelitemi}{$-$}

\usepackage{fancyhdr}
\newcommand{\changefont}{%
    \fontsize{9}{11}\selectfont
}
\pagestyle{fancy}
\fancyhead{}
\fancyfoot{}
\renewcommand{\headrulewidth}{0mm}
\rfoot{\thepage}    
\cfoot{}
\lfoot{\changefont Inbraakdetectie in draadloze sensornetwerken: Probleem- en doelstelling}

\author{Christophe Van Ginneken\\\url{Christophe.VanGinneken@student.kuleuven.be}}

\title{Inbraakdetectie in draadloze sensornetwerken: Probleem- en doelstelling}
\date{\today}

% http://tex.stackexchange.com/questions/299/how-to-get-long-texttt-sections-to-break
\newcommand*\justify{%
  \fontdimen2\font=0.4em% interword space
  \fontdimen3\font=0.2em% interword stretch
  \fontdimen4\font=0.1em% interword shrink
  \fontdimen7\font=0.1em% extra space
  \hyphenchar\font=`\-% allowing hyphenation
}

% http://en.wikibooks.org/wiki/LaTeX/Customizing_LaTeX
\newcommand{\ttt}[1]{{\tt \justify{#1}}}

\linenumbers
\begin{document}

\maketitle

\section*{Inleiding}

In de context van deze thesis, inbraakdetectie voor draadloze sensornetwerken,
wordt bij wijze van voorbeeld nogal snel beroep gedaan op de \emph{anonieme
hacker} die zich toegang verleent tot de weerloze draadloze sensorknoop, hem
omschoolt tot zijn volgeling en hierdoor de goede werking van het netwerk in
het gedrag brengt. Ofschoon volledig correct, blijft het voor buurman Jan
slechts een abstract gegeven en blijft hij achter met de vraag ``Ja, ok, en
dan?''. Waar het echte probleem waarschijnlijk nog enkele jaren in de toekomst
ligt, zijn er vandaag spijtig genoeg toch reeds voorbeelden te vinden.

\subsection*{De realiteit achtervolgt de fictie}

We schrijven midden augustus 2013. Ergens in de Verenigde Staten van Amerika,
leggen twee jonge ouders hun kind te slapen, onder het alziende oog van hun
nieuwe draadloze, met het internet verbonden, babyfoon. Wanneer zij enige tijd
later de kamer van het kind opnieuw betreden, horen ze een onbekende stem
obscene woorden ten berde brengen langs deze babyfoon, tot groot jolijt van de
kleine spruit.

Ondertussen wordt in Europa duchtig verder gesleuteld aan de draadloze
pacemaker. Een wonderbaarlijk stukje technologie dat artsen toegang geeft tot
het hart van hun pati\"ent, waar deze zich ook bevindt. Het lijkt wel, en is
ook, een scene uit een fictie-serie waarin een hacker zich toegang verschaft
tot zo'n pacemaker en zo de drager ervan vermoordt. Fictie? Dick Cheney denkt
het in ieder geval niet. In oktober 2013 heeft hij immers het draadloze aspect
van zijn pacemaker laten verwijderen, om een mogelijke terroristische aanval te
vermijden.

Toen we in 1995 genoten van de eerste acteerprestaties van Angelina Jolie in de
cultfilm ``Hackers'', leek het kunnen besturen van verkeerslichten een prachtig
staaltje science fiction. Bijna 20 jaar later, zijn verkeerslichten en
draadloze sensornetwerken in wetenschappelijke publicaties alvast dikke
vrienden en is men klaar om onder het mom van de zgn. ``Smart City'', elk van
deze lichten een autonoom leven te bieden. De grote omarming door het
``Internet of Things'' is nu reeds alomtegenwoordig.

Zelfs buurman Jan kan niet voorbij aan deze voorbeelden en stelt zich
ondertussen terecht de vraag: ``Ok, en wat nu?''. De feiten zijn wat ze zijn en
weldra zal elk \emph{ding} op aarde wel een eigen netwerkadres hebben. Hoe zal
die wereld er dan eigenlijk uit zien?

\section*{Probleemstelling}

Aan de ene kant hebben we de draadloze sensorknopen en aan de andere kant
hebben we de beveiliging ervan, meer specifiek een inbraakdetectiesysteem.
Ofschoon de ene de andere duidelijk nodig heeft om zich - en uiteindelijk ons -
te beschermen, blijkt snel dat hun huwelijk strandt op basis van tegenstrijdige
belangen.

Een sensorknoop is nagenoeg gedurende zijn volledige levensloop aan zijn lot
overgelaten. De enige link met de buitenwereld is zijn draadloze radio die hem
toelaat te communiceren met andere knopen. Diezelfde knopen zijn tevens zijn
enige communicatiekanaal met de wereld buiten het netwerk. Gedurende deze hele
periode dient de knoop energie te halen uit \'e\'en enkele batterij en moet dus
zeer spaarzaam omspringen met deze bron. Omdat knopen tevens in grote getalen
worden ingezet en ze feitelijk beschouwd dienen te worden als wegwerpproducten,
dient hun kostprijs zo laag mogelijk gehouden te worden. Ze zijn dan ook
voorzien van net genoeg geheugen en verwerkingskracht om hun veelal beperkte
taak uit te voeren.

Inbraakdetectie, aan de andere kant, vraagt opvolging. De meerderheid van de
alarmen die door een dergelijk systeem worden gegenereerd zijn nog steeds te
valideren. Aangezien bijna elke aanval verschillend is, zijn de mogelijkheden
om te detecteren ook vaak eindeloos en wil men een inbraakdetectiesysteem
constant laten werken. Om zijn werk te kunnen doen, dient het bij voorkeur ook
te beschikken over veel gegevens. Deze gaan van aanvalspatronen tot modellen
van normaal gedrag om anomalie\"en te kunnen detecteren. Tot slot is, vanuit
het oogpunt van een knoop, het detecteren van inbraakpogingen een
niet-functionele, bijkomende belasting.

Het is snel duidelijk dat de afweging tussen knoop en detectie neerkomt op een
gebruik van de middelen waarover een knoop beschikt. In een ideale wereld zou
een inbraakdetectiesysteem voor draadloze sensorknopen de levensduur van de
batterij van de knoop niet mogen be\"invloeden, maar dat is spijtig genoeg
echte science fiction.

\subsection*{Het onderzoeksdomein}

De literatuur omtrent inbraakdetectie in draadloze sensornetwerken beschrijft
onnoemelijk veel manieren om specifieke aanvallen te detecteren. Een enkele
uitzondering waagt zich aan een combinatie van patronen of stelt een raamwerk
voor om enkele patronen te combineren. De algemene conclusie is echter unaniem:
gegeven de beperkte mogelijkheden van een knoop, is het onmogelijk om een
volledige dekking te bieden betreffende inbraakdetectie. Meer zelfs, omdat een
aanvaller nagenoeg te allen tijde in staat is om een knoop fysiek te benaderen,
het geheugen van de knoop te raadplegen en te wijzigen, is er per definitie
geen enkele mogelijkheid om elke poging inbraak te verijdelen zonder fysieke
uitbreiding van een knoop met bijkomende, specifieke hardware. Deze zou echter
de prijs van een knoop meerdere malen vermenigvuldigen en zich zelf daarmee uit
de markt prijzen. Het beveiligen van draadloze sensornetwerken komt daarmee
neer op een afweging van risico's. Hierbij zal een inschatting gemaakt moeten
moeten worden van welke mogelijke aanvallen we willen onderscheppen om zoveel
mogelijke barri\`eres op te werpen om de meerderheid van de aanvallers te
ontmoedigen.

\section*{Doelstelling}

Hoe kan een masterthesis dan bijdragen tot dit probleem? Indien het probleem
onoverkomelijk is, rest er het draaglijker maken van de pijn. Indien we in
staat zijn om de impact van de inbraakdetectie te verlichten, kan er op meer
aanvallen gecontroleerd worden en kan de maker van het sensornetwerk een
ruimere keuze maken uit de bestaande detectiealgoritmen, waardoor de drempel
voor de aanvaller toch weer een beetje hoger wordt.

\subsection*{Situatie}

Om een mogelijke bijdrage tot verbetering van deze probleemstelling te
defini\"eren bekijken we eerst de situatie op het slagveld: De ontwikkelaar van
de software van een draadloze sensorknoop kan zelf de literatuur raadplegen en
\'e\'en of meerdere van de beschreven detectiemechanismen trachten te
implementeren. Hierbij zal hij typisch, voor elk van deze detectoren, een zeer
gelijkaardige blok code produceren.

Aangezien een enkele knoop zelden een aanval kan detecteren, is communicatie
met de andere knopen een noodzakelijk kwaad. Tevens is activiteit op het
netwerk typisch de belangrijkste bron van informatie voor de knoop om zich een
beeld te vormen van wat er zich rondom hem afspeelt. Nagenoeg elk
detectiealgoritme zal bij ontvangst van een netwerkpakket de inhoud ervan
moeten inspecteren om zich van nieuwe informatie over zijn omgeving te
vergewissen.

Anderzijds dient elk algoritme ook op regelmatige tijdstippen een inventaris te
maken van de geaccumuleerde situatie. Dit kan gaan van het controleren of er
geen andere knopen zijn die reeds geruime tijd geen activiteit hebben vertoond,
tot het berekenen van een reputatie van een knoop op basis van de verschillende
gebeurtenissen in het netwerk en de informatie die andere knopen hieromtrent
hebben gedeeld.

Het is evident dat deze zeer gelijklopende structuur voor elk
detectiemechanisme onherroepelijk kan leiden tot code duplicatie, maar ook tot
meerdere sequenti\"ele iteraties en talrijke communicaties tussen de knopen
voor elk van de algoritmen afzonderlijk. Sequenti\"ele uitvoering leidt tot
langere uitvoertijden en meerdere communicaties leidt tot meer gebruik van het
draadloze netwerk.

\subsection*{Probleemevaluatie}

Voor eenvoudige algoritmen lijkt dit een artificieel probleem. Elke zichzelf
respecterende ontwikkelaar zal dit opmerken en zal de code zo structureren dat
deze redundanties weggewerkt worden. Inderdaad, voor eenvoudige algoritmen \'en
indien de ontwikkelaar zijn volledige inbraakdetectie zelf bouwt, is dat het
geval. Maar inbraakdetectie is een niet-functioneel gegeven voor de
ontwikkelaar. Om inbraakdetectie schaalbaar inzetbaar te maken, zal men ook
hier zoveel mogelijk gebruik willen maken van bestaande implementaties. Op dat
ogenblik heeft de ontwikkelaar niet langer de luxe om de code anders te
organiseren en zullen de langere uitvoertijden en het veelvuldige
netwerkgebruik effectief een overmatige belasting worden op de beperkte
mogelijkheden van de sensorknoop.

Men kan echter nog verder argumenteren dat de de bijkomende verwerkingstijden
bijna verwaarloosbaar zijn voor microcontrollers die amper 0.4mA stroom
verbruiken wanneer ze actief zijn. Dit is correct, maar in dat geval mag men
niet uit het oog verliezen dat een typische draadloze radio al snel 40mA
verbruikt bij verzenden en ontvangen. Dit is een factor 100 ten opzichte van de
microcontroller. Een veelgebruikte oplaadbare batterij, zoals deze in GSM
toestellen, biedt gangbaar ongeveer 1700mAh. Een actieve draadloze radio
verbruikt al snel 40mA en zal de energie van deze volledige batterij op iets
minder dan 2 dagen opgebruiken. Het beperken van het gebruik ervan is dus een
zeer belangrijke noodzaak.

\subsection*{Aanpak}

In deze thesis stel ik voor om de beschrijving van een detectiealgoritme te
realiseren aan de hand van een domeinspecifieke taal. Deze taal is beperkt in
zijn expressiviteit en bevat bv. niet de mogelijkheid om een klassieke iteratie
te beschrijven. Hierdoor wordt \'e\'en van de basisproblemen ontweken en ligt
de weg open om op geautomatiseerde wijze verschillende van deze algoritmen te
combineren.

Immers, indien hergebruik van bestaande code niet mogelijk is omwille van de
hoger vermelde problemen, dient er op een andere manier een combinatie van
detectiemechanismen gerealiseerd te worden. In deze thesis wordt daarom
voorgesteld om op basis van de beschrijvingen in de domeinspecifieke taal, door
middel van codegeneratie, de nodige code voor inbraakdetectie te genereren op
maat van de ontwikkelaar, net zoals wanneer hij dit manueel zou doen. Op deze
manier kan nog steeds de beschrijving opnieuw gebruikt worden, echter zonder
dat dit tot redundant werk leidt.

Door aan de hand van codegeneratie de code optimaal te organiseren, wordt het
mogelijk om de verschillende algoritmen parallel te laten werken en niet langer
sequentieel. Dit leidt ook tot de mogelijkheid om de communicatie die de
verschillende algoritmen versturen te bundelen in \'e\'en packet, waardoor het
herhaaldelijk gebruik van de draadloze radio tot een minimum gereduceerd wordt.

Het samennemen van verschillende uitvoeringslussen en het combineren van
communicatie zijn slechts twee van de mogelijkheden die zich aanbieden. Zo kan
ook het vluchtige geheugengebruik geoptimaliseerd worden door delen van
variabelen tussen algoritmen. Ook wordt het dankzij codegeneratie mogelijk om
een aantal aspecten om te zetten in meer technische code, waar een menselijke
programmeur dikwijls voorrang geeft aan leesbaarheid en onderhoudbaarheid.
Naast de eerste twee hoger vermelde doelstellingen, willen we in deze thesis
ook andere van deze technieken evalueren en de opportuniteiten in kaart brengen.

De implementatie van dit codegeneratie raamwerk zal in eerste instantie
volledig onafhankelijk van enige andere technologie ontwikkeld worden. Deze
aanpak laat toe om het resultaat later in verschillende andere omgevingen in te
bedden. Minstens \'e\'en andere omgeving zal uitgewerkt worden. Hierbij zal
getracht worden om het resultaat op te splitsen in enerzijds een basiscomponent
die op elke knoop voorzien kan worden en anderzijds een gegenereerd deel dat
apart ge\"installeerd kan worden.

\section*{Conclusie}

De noden en mogelijkheden van inbraakdetectie en draadloze sensornetwerken
kunnen geen grotere concurrenten zijn van elkaar. Draadloze sensorknopen zijn
enorm beperkt in hun mogelijkheden en de impact van een degelijke
inbraakbeveiliging hypothekeert nagenoeg de volledige functionaliteit van de
knoop.

Ofschoon de fysieke toegankelijkheid van sensorknopen leidt tot een situatie
waar het feitelijk onmogelijk is om een inbraak te detecteren, blijft het
belangrijk om deze netwerken toch te voorzien van een vorm van inbraakdetectie.
Hoe meer aanvallen een netwerk van zulke knopen kan detecteren, hoe moeilijker
het wordt voor een aanvaller om zich vlot meester te maken van het netwerk en
er malafide praktijken mee te ondersteunen.

Naast het onderzoeken van aanvallen en het beschrijven van algoritmen om deze
te detecteren, is er nog een andere piste die bewandeld kan worden. In deze
thesis wordt codegeneratie als mogelijke oplossing voorgesteld om de
consequenties die de implementatie van inbraakbeveiliging met zich meebrengt te
verlichten.

Aan de hand van codegeneratie kunnen de algoritmen onafhankelijk beschreven
worden en toch zo georganiseerd worden dat de uitvoeringstijd geminimaliseerd
wordt. Verder wordt ook het gebruik van het netwerk gegroepeerd in een enkele
verzending van de informatie van alle verschillende algoritmen samen.

Zo resulteert deze aanpak in een automatisering van de implementatie van
inbraakdetectiemechanismen, waardoor de flexibiliteit om code te organiseren
behouden blijft en het mogelijk wordt om de impact van deze bijkomende
niet-functionele code te minimaliseren. De onafhankelijkheid van de
domeinspecifieke taal biedt verder tevens de mogelijkheid om
platformonafhankelijke beschrijvingen van inbraakdetectiealgoritmes te
hanteren binnen het zeer heterogene landschap van de draadloze sensornetwerken.

\end{document}
