%!TEX root=masterproef.tex

\chapter{Architectuur}
\label{chapter:architectuur}

Ofschoon het domein van inbraakdetectie goed afgelijnd lijkt, werd uit de
analyse van de probleemstelling snel duidelijk dat in het geval van draadloze
sensornetwerken, de situatie enigszins complexer blijkt te zijn dan men op het
eerste zicht zou mogen verwachten. De conclusie van de probleemstelling is dat
vooral het onderzoek naar inbraakdetectie in draadloze sensornetwerken
feitelijk nog in zijn kinderschoenen staat en dat de stap naar
industrialisering nog niet genomen is.

De daarbij horende gestructureerde manier van werken en het streven naar een
geformaliseerde manier van werken ontbreken nog duidelijk en noodzaken daarom
een holistische aanpak. De oplossingsstrategie vertrekt van dit gegeven en
wordt in sectie \ref{section:strategy} uitgewerkt tot een totaaloplossing.

Deze strategie wordt in secties \ref{section:funcarch} en
\ref{section:techarch} uitgewerkt in de vorm van respectievelijk een
functionele en technische architectuur.

\section{Oplossingsstrategie}
\label{section:strategy}

\TODO

\section{Functionele architectuur}
\label{section:funcarch}

In een functionele analyse gaan we in essentie op zoek naar de manier waarop de
beoogde functionaliteit kan bereikt worden. In dit geval betreft het het zoeken
naar een manier waarop het mogelijk is om detectiealgoritmen van verschillende
bronnen te combineren tot een geheel dat minder eisen stelt tov. een
sensorknoop, dan wanneer mijn de verschillende algoritmen sequentieel hun ding
laat doen.

\subsection{Manuele programmatie}

In zijn eenvoudigste vorm kan een oplossing voor het probleem bestaan in het
defini\"eren van regels waar men zich moet aan houden bij het implementeren van
detectiealgoritmes. Zo is het in theorie perfect mogelijk voor een programmeur
om de extra belasting die de implementatie van verschillende detectiealgortimen
met zich meebrengt te minimaliseren door voorafgaand een goede analyse te maken
en op systematische wijze deze om te zetten in goed georganiseerde code.

Gedistribueerde afspraken zonder centrale controle werken zelden. De
ongebondenheid laat nog steeds toe om terug te vallen op klassieke
implementaties.

\subsection{Code generatie}

\TODO

\subsection{Semantisch model}

\TODO

\citep{fowler2010domain}

\subsection{Domein specifieke taal}



\section{Technische architectuur}
\label{section:techarch}

\TODO

\subsection{Python}

\TODO

\subsection{ANTLR}

\TODO

