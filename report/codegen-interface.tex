%!TEX root=masterproef.tex

\subsection{Interfaces}
\label{subsection:devel-codegen-interfaces}

Voor we het SM en het CM in detail bekijken, kijken we eerst naar de interfaces
die de codegenerator ter beschikking stelt.

\subsubsection{foo.py}

Op het hoogste niveau biedt de generator een interface via de opdrachtprompt
(\emph{Command-Line Interface}) (CLI) aan in de vorm van een Python script:
\ttt{foo.py}. Codevoorbeeld \ref{lst:foo.py-help} toont de uitvoering van het
script en geeft een overzicht van de mogelijkheden.

\begin{listing}[ht]
  \begin{minted}[linenos,frame=lines,framesep=2mm,fontsize=\footnotesize]{console}
$ source setpath.sh
$ ./foo.py --help
usage: foo.py [-h] [-v] [-c] [-i] [-g FORMAT] [-o OUTPUT] [-l LANGUAGE]
              [-p PLATFORM]
              [sources [sources ...]]

Command-line tool to interact with foo-lang and its code generation
facilities.

positional arguments:
  sources               the source files in foo-lang

optional arguments:
  -h, --help            show this help message and exit
  -v, --verbose         output info on what's happening
  -c, --check           perform model checking
  -i, --infer           perform model type inferring
  -g FORMAT, --generate FORMAT
                        output format (choices: none, ast, ast-dot, sm-dot,
                        foo, code / default: none)
  -o OUTPUT, --output OUTPUT
                        output directory (default: .)
  -l LANGUAGE, --language LANGUAGE
                        when format=code: target language (choices: c /
                        default: c)
  -p PLATFORM, --platform PLATFORM
                        when format=code: target platform (choices: moose,
                        demo / default: moose)
  \end{minted}
  \vspace{-5mm}
  \caption{Informatie over de werking van \ttt{foo.py}}
  \label{lst:foo.py-help}
\end{listing}

De CLI biedt toegang tot alle aspecten van de generator: modelcontrole
(\ttt{check}), typedeductie (\ttt{infer}), het uitvoerformaat (\ttt{format}),
plaats van de uitvoer (\ttt{output}), de programmeertaal (\ttt{language}) en
voor welk platform de generatie moet gebeuren (\ttt{platform}).

De lijst van mogelijke uitvoerformaten bestaat uit: \ttt{none}, \ttt{ast},
\ttt{ast-dot}, \ttt{sm-dot}, \ttt{foo} en \ttt{code}. Formaat \ttt{ast} toont
een hi\"erarchisch overzicht van de AST op het scherm in tekstuele vorm. De
uitvoer van \ttt{ast-dot} zagen we eerder in figuur \ref{fig:devel-ast}. De
uitvoer is code die als invoer kan dienen voor GraphViz \citep{url:graphviz},
een openbronproject dat zich specialiseert in het visualiseren van
graafgeori\"enteerde gegevens, zoals deze AST. Door middel van het
\ttt{dot}-commando kan van deze code een visuele voorstelling gemaakt worden.

Overeenkomstig bestaat er de mogelijkheid om een visuele voorstelling te maken
van het SM, door middel van het \ttt{sm-dot} formaat. Om controles te doen
betreffende de goede verwerking van de FOO-lang broncode kan een ingelezen set
van modules ook opnieuw als FOO-code uitgevoerd worden.

Tot slot is er nog het \ttt{code} formaat, dat de generator vraagt om code te
genereren. Hierbij dienen dan ook de overige opties voorzien te worden:
uitvoerlocatie, taal en platform.

\subsubsection{API}

Het \ttt{foo.py} Python-script is slechts een dunne schil rond de Python-API.
Deze biedt alle functionaliteit aan in de vorm van een Python-module met een
imperatieve interface. Codevoorbeeld \ref{lst:codegen-api} toont de interface
van deze module.

\begin{listing}[ht]
  \begin{minted}[linenos,frame=lines,framesep=2mm,fontsize=\footnotesize]{python}
def create_model():
  ...
  return model

def parse(string, noprint=False):
  ...
  return parser

def infer(model, silent=False):
  ...

def check(model, silent=False):
  ...

def generate(model, args):
  ...

def load(string, model=None):
  ...
  return model
  \end{minted}
  \vspace{-5mm}
  \caption{API van de codegenerator}
  \label{lst:codegen-api}
\end{listing}

De verschillende fasen uit het generatieproces bestaan uit het aanmaken van een
(leeg) model, het parsen van de broncode, het deduceren van onbekende types,
het controleren of een model volledig in orde is en het genereren van code. De
bijkomende \ttt{load}-functie combineert de \ttt{create\_model} en
\ttt{parse}-functionaliteit in \'e\'en handige functie.

De API laat toe om de generator vanuit Python aan te spreken en eventueel
verder te integreren in een uitgebreider compilatieproces, of om andere
interfaces te voorzien (visuele gebruikersinterfaces zoals bv. een
webinterface\dots).

Verder biedt de API toegang tot de entiteiten op het hoogste niveau, zoals de
parser, de model-entiteit uit het SM\dots De volledige openheid van Python-code
laat toe om dieper door te dringen en elk aspect van het model te ondervragen
en te wijzigen.

Beide onderliggende modellen kunnen tevens volledig benaderd worden aan de hand
van een \emph{visitor}. Deze wordt door de generator veelvuldig gebruikt, zelfs
voor kleine operaties en biedt een aantrekkelijkere interface om met de
modellen te werken dan het direct ondervragen van eigenschappen en het
aanroepen van methoden in het model.
