%!TEX root=masterproef.tex

\subsection{Interfaces}
\label{subsection:devel-codegen-interfaces}

Voor we het SM en het CM in detail bekijken, kijken we eerst naar de interfaces
die de code generator ter beschikking stelt.

\subsubsection{foo.py}

Op het hoogste niveau biedt de generator een commandolijn interface (CLI) aan
in de vorm van een Python script: \ttt{foo.py}. Codevoorbeeld \ref{lst:foo.py-help}
toont de uitvoering van het script die een overzicht geeft van de mogelijkheden.

\begin{listing}[ht]
  \begin{minted}[linenos,frame=lines,framesep=2mm,fontsize=\footnotesize]{console}
$ source setpath.sh
$ ./foo.py --help
usage: foo.py [-h] [-v] [-c] [-i] [-g FORMAT] [-o OUTPUT] [-l LANGUAGE]
              [-p PLATFORM]
              [sources [sources ...]]

Command-line tool to interact with foo-lang and its code generation
facilities.

positional arguments:
  sources               the source files in foo-lang

optional arguments:
  -h, --help            show this help message and exit
  -v, --verbose         output info on what's happening
  -c, --check           perform model checking
  -i, --infer           perform model type inferring
  -g FORMAT, --generate FORMAT
                        output format (choices: none, ast, ast-dot, sm-dot,
                        foo, code / default: none)
  -o OUTPUT, --output OUTPUT
                        output directory (default: .)
  -l LANGUAGE, --language LANGUAGE
                        when format=code: target language (choices: c /
                        default: c)
  -p PLATFORM, --platform PLATFORM
                        when format=code: target platform (choices: moose,
                        demo / default: moose)
  \end{minted}
  \vspace{-5mm}
  \caption{Informatie over de werking van \ttt{foo.py}}
  \label{lst:foo.py-help}
\end{listing}

De CLI biedt toegang tot alle aspecten van de generator: model controle
(\ttt{check}), type deductie (\ttt{infer}), het uitvoerformaat, waar de uitvoer
moet geplaatst worden, welke taal gebruikt moet worden en voor welk platform de
generatie moet gebeuren.

De lijst van mogelijke uitvoerformaten bestaat uit: \ttt{none}, \ttt{ast},
\ttt{ast-dot}, \ttt{sm-dot}, \ttt{foo} en \ttt{code}.

Formaat \ttt{ast} toont een hierarchisch overzicht van de AST op het scherm in
tekstuele vorm, zoals weergegeven in codevoorbeeld \ref{lst:foo.py-ast}. De
uitvoer van \ttt{ast-dot} zagen we in essentie reeds eerder in figuur
\ref{fig:devel-ast}. De uitvoer is feitelijk code die als invoer kan dienen
voor GraphViz \citep{url:graphviz}, een open bron project dat zich
specialiseert in het visualiseren van graafgeori\"enteerde gegevens, zoals deze
AST met een boomstructuur. Door middel van het \ttt{dot} commando kan
vervolgens van deze code een visuele voorstelling gemaakt worden.

\begin{listing}[ht]
  \begin{minted}[linenos,frame=lines,framesep=2mm,fontsize=\footnotesize]{console}
$ source setpath.sh
$ ./foo.py -g ast examples/hello.foo 
ROOT
  MODULE
    IDENTIFIER
      hello
    CONST
      IDENTIFIER
        interval
      UNKNOWN_TYPE
      INTEGER_LITERAL
        1000
    EXTEND
...
  \end{minted}
  \vspace{-5mm}
  \caption{Tekstuele uitvoer van een AST}
  \label{lst:foo.py-ast}
\end{listing}

Overeenkomstig bestaat er ook de mogelijkheid om een visuele voorstelling te
maken van het SM, door middel van het \ttt{sm-dot} formaat. Om controles te
doen betreffende de goede verwerking van de FOO-lang broncode kan een ingelezen
set van modules ook opnieuw als FOO-code uitgevoerd worden.

Tot slot is er nog het \ttt{code} formaat, dat de generator vraagt om
effectieve code te genereren. Hierbij dienen dan ook de overige opties
eventueel ingevuld te worden: uitvoerlocatie, taal en platform.

\subsubsection{API}

Het \ttt{foo.py} Python script is slechts een CLI-verpakking rond de Python
API. Deze biedt alle functionaliteit aan in de vorm van een Python module met
een imperatieve interface. Codevoorbeeld \ref{lst:codegen-api} toont de interface
van deze module.

\begin{listing}[ht]
  \begin{minted}[linenos,frame=lines,framesep=2mm,fontsize=\footnotesize]{python}
def create_model():
  ...
  return model

def parse(string, noprint=False):
  ...
  return parser

def infer(model, silent=False):
  ...

def check(model, silent=False):
  ...

def generate(model, args):
  ...

def load(string, model=None):
  ...
  return model
  \end{minted}
  \vspace{-5mm}
  \caption{API van de code generator}
  \label{lst:codegen-api}
\end{listing}

In volgorde zien we de verschillende fasen uit het generatie proces: het
aanmaken van een (leeg) model, het parsen van de broncode, het deduceren van
onbekende types, het controleren of een model volledig in orde is en
uiteindelijk het genereren van de code. De bijkomende \ttt{load} functie
combineert de \ttt{create\_model} en \ttt{parse} functionaliteit in \'e\'en
handige functie.

De API laat toe om de generator vanuit Python aan te spreken en eventueel
verder te integreren in een uitgebreider compilatieproces, of om andere
interfaces te voorzien (visuele gebruikersinterfaces zoals bv. een
webinterface,\dots).

De API biedt toegang tot de entiteiten op het hoogste niveau, zoals de parser,
de model-entiteit uit het SM,\dots De volledige openheid van Python code laat
verder toe om dieper door te dringen en elk aspect van het bv. model te
ondervragen of zelfs te wijzigen.

Beide onderliggende modellen zijn echter volledig ondervraagbaar aan de hand
van een \emph{visitor}. Deze worden door de generator veelvuldig gebruikt,
zelfs voor kleine operaties en bieden een veel aantrekkelijkere interface om
met de modellen te werken dan het ruwweg volgen van eigenschappen en methoden
doorheen het model.
