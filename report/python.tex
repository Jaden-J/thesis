%!TEX root=masterproef.tex

\section{Python}
\label{section:devel-python}

Als programmeertaal voor het prototype werd geopteerd voor Python, een
ge\"interpreteerde taal met dynamische typering die tevens verschillende
programmeerparadigma ondersteunt: imperatief, object-geori\"enteerd en
functioneel. Dit maakt het een zeer veelzijdige taal met veel mogelijkheden.

Python is volledig open in zijn structuur. Alles is toegankelijk en niets wordt
verborgen. Dit laat toe om elk aspect van een gegevensstructuur te manipuleren,
wat heel handig is, maar ook kan leiden tot onverwachte neveneffecten.

Alle functionaliteit, klassen of gewone functies, worden verzameld in een
\emph{module} en andere modules kunnen vervolgens deze functionaliteit
importeren. Door de volledige transparantie en dankzij introspectie, kan de
implementatie van een module zelfs dynamisch aangepast worden. Dit werd o.a.
toegepast voor het implementeren van het \emph{visitor}-patroon, in meer detail
besproken in bijlage \ref{appendix:visitor}.

Verder beschikt Python over een zeer rijke verzameling van kant-en-klare
modules, die toelaten om enkele basistaken vlot te implementeren. De
flexibiliteit en de mix van zowel imperatief als object-geori\"enteerd als
functioneel programmeren liet meermaals toe om bepaalde zaken op creatieve
manier te implementeren.
