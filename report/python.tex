%!TEX root=masterproef.tex

\section{Python}
\label{section:devel-python}

Als programmeertaal voor het prototype werd geopteerd voor Python. Python is
een ge\"interpreteerde taal met dynamische typering die verschillende
programmeerparadigma ondersteunt: imperatief, object-geori\"enteerd en
functioneel. Dit maakt het een zeer veelzijdige taal die veel mogelijkheden
biedt.

Python is ook volledig open in zijn structuur. Alles is toegankelijk en niets
wordt verborgen. Dit laat toe om elk aspect van een gegevensstructuur of object
te manipuleren. Dit kan heel handig zijn, maar kan ook leiden tot onverwachte
neveneffecten.

Alle functionaliteit, klassen of gewone functies, worden verzameld in een
\emph{module} en andere modules kunnen vervolgens deze functionaliteit
importeren. Door de volledige transparantie en dankzij ver doorgedreven
mogelijkheden tot introspectie, kan de implementatie van een module dynamisch
aangepast worden. Dit werd o.a. uitvoerig toegepast voor het implementeren van
het \emph{visitor} patroon, verder besproken in sectie
\ref{subsubsection:devel-visitor-pattern}.

Ofschoon mijn ervaring met Python beperkt was, is Python zeker geen slechte
keuze voor de implementatie van deze software. Zo beschikt de taal over een
zeer rijke verzameling van kant-en-klare modules, die toelaten om enkele
basistaken snel te implementeren. De flexibiliteit en de mix van zowel
imperatief als object-geori\"enteerd als functioneel programmeren liet
meermaals toe om bepaalde zaken op creatieve manier te implementeren.

Het feit dat in essentie een nieuwe taal was, bracht ook een bijkomende
leercurve met zich mee. Ook zorgde voortschrijdend inzicht voor stapsgewijze
verbeteringen aan bepaalde constructies, die echter soms door tijdbeperkingen
niet voor alle overige code konden bijgewerkt worden.
