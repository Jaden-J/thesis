%!TEX root=masterproef.tex

\chapter{Oplossingsstrategie}
\label{chapter:oplossingsstrategie}

Uit de probleemstelling moeten we concluderen dat er weinig in het voordeel van
het bouwen van een IDS voor een WSN spreekt. Op elk niveau, van de hardware van
de knoop en de elementaire software die het netwerk vormgeeft, tot de
onderzoeker, ontwikkelaar en uitbater van het netwerk, zijn er obstakels te
identificeren. Dit hoofdstuk volgt opnieuw de verschillende stappen in de keten
en tracht een antwoord aan te bieden dat tegemoet komt aan de
ge\"identificeerde problemen.

Sectie \ref{section:solution-node-wsn} stelt dat het ontbreken van een dominant
hardwareplatform er toe leidt dat de oplossing flexibel moet zijn t.o.v.
verschillende platformen.

Sectie \ref{section:solution-software} veralgemeent deze eis naar het niveau
van de software, omdat er bv. ook geen overheersend besturingssysteem
beschikbaar is.

Sectie \ref{section:solution-proglang} evalueert de mogelijkheid om de
programmeertaal C te hanteren als gemeenschappelijke en gestandaardiseerde taal
om detectiealgoritmen te beschrijven.

Sectie \ref{section:solution-library} introduceert softwarebibliotheken als een
oplossing voor gemeenschappelijke logica en het centraliseren van iteratieve
processen.

Sectie \ref{section:solution-dsl} argumenteert dat een domeinspecifieke taal
een oplossing kan zijn om aan de hand van een formele en platformonafhankelijke
beschrijving van algoritmen o.a. een brug te slaan tussen onderzoek en
ontwikkeling.

Sectie \ref{section:solution-codegen} tot slot stelt dat codegeneratie een
geformaliseerd proces kan automatiseren en kan tegemoet komen aan de overige
problemen in het kader van het ontwikkelingsproces.

\vspace{-3mm}

\section{Hardware en netwerk}
\label{section:solution-node-wsn}

Het feit dat sensorknopen, en ge\"integreerde systemen in het algemeen, fysiek
toegankelijk zijn en daarom inherent vatbaar voor fysieke aanvallen, is een
probleem waaraan op zich weinig kan gedaan worden. Zolang de ontwikkeling van
sensorknopen in zijn kinderschoenen staat en de focus nog op de basisbehoeften
ligt (energiebesparing, grootte,...) en zolang er geen echte standaard voor
deze hardware bestaat\footnote{Standaarden zullen zonder twijfel ontstaan en de
eerste schuchtere pogingen zijn reeds te zien in de vorm van bv. het Zigduino
platform (\url{http://www.logos-electro.com/store/zigduino-r2}) of de Waspmote
van Libellum (\url{http://www.libelium.com/products/waspmote/}).}, zal het nog
even duren voor de focus verschuift naar de beveiliging van de hardware.

Vanuit het oogpunt van de hardware en het netwerk houden we daarom een
belangrijk criterium voor ogen: de oplossing moet flexibel genoeg zijn om met
verschillende soorten hardware, netwerkprotocollen\dots om te gaan en mag dus
niet inherent afhankelijk zijn van beperkende veronderstellingen op dit niveau.

\vspace{-3mm}

\section{Elementaire software}
\label{section:solution-software}

Onder elementaire software verstaan we elke vorm van software die nodig is om
de basisbehoeften van het systeem te vervullen. Daarbij denken we aan een
besturingssysteem, maar dat verschilt van de klassieke opvatting in het geval
van een sensorknoop, bv. in grootte en mogelijkheden. Maar het kan ook zo
eenvoudig zijn als een eindeloze herhaling van dezelfde instructies.

Een besturingssysteem voor een sensorknoop biedt een dunne abstractielaag die
de harde realiteit van de programmatie van een naakte \mcu verlicht door de
introductie van processen, componenten\dots Gegeven de beperkte voorzieningen
van een sensorknoop, focust de ontwikkeling van deze systemen zich op het
beperken van de eigen impact en zal daarom zoveel mogelijk balast trachten te
vermijden.

De meeste van deze systemen zijn openbronsoftware en integratie is mogelijk
door diepgaande aanpassingen. Aanpassingen zijn dan wel nodig voor elk
besturingssysteem en net zoals bij de hardware, is er nog geen de facto
standaard. TinyOS en Contiki zijn zonder twijfel sterke spelers, maar de
\emph{Windows} onder de besturingssystemen voor sensorknopen moet nog opstaan.

\vspace{-3mm}

\section{Programmeertalen}
\label{section:solution-proglang}

De keuze van een programmeertaal gaat in het geval van een WSN vooral samen met
de taal die gebruikt wordt door het besturingssysteem. Omdat dit systeem
typisch niet op zich staat en de toepassing samen met het systeem tot een
geheel verwerkt wordt, is de integratie tussen de twee zeer sterk.

De ontwikkelaar van de hardware levert bij zijn sensorknopen, of zelfs al bij
een \mcu, typisch \'e\'en compiler en derhalve ook \'e\'en taal. Het aanbod is
uitermate beperkt en bestaat in grote lijnen uit ``C''. Er bestaan
alternatieven, maar die zijn schaars, bv. de SunSpot van Oracle, die Java
aanbiedt.

De programmeertaal C staat zeer dicht bij de hardware en is duidelijk de eerste
stap in de ontplooiing van softwareontwikkeling op ge\"integreerde systemen. De
voorzichtige pogingen om nieuwe programmeertalen voor te stellen tonen aan dat
op dit vlak een evolutie in ontwikkeling is. Maar het zal nog enige tijd duren
vooraleer een nieuwe, dominante taal opstaat en C van de troon stoot.

Op dit ogenblik is C de lingua franca voor alle spelers in dit segment: zowel
onderzoeker als ontwikkelaar hanteren deze taal, ook al moet de kanttekening
gemaakt worden dat in veel gevallen onderzoekers zich nog beroepen op
simulaties waarbij C ontweken wordt.

\vspace{-3mm}

\section{Softwarebibliotheek}
\label{section:solution-library}

In sectie \ref{section:problem-develop} zagen we reeds dat \'e\'en van de
pijnpunten is dat de verschillende algoritmen typisch dezelfde acties
ondernemen. Indien de algoritmen als onafhankelijke modules zouden worden
ontwikkeld en sequentieel achter elkaar opgeroepen worden, zullen zij veel
dubbel werk leveren. Dat gaat echter ten koste van de energievoorziening en dus
de levensduur van de knoop.

Dit is een klassiek softwareprobleem en wordt normaal opgevangen door het
gebruik van een softwarebibliotheek. Deze biedt een verzameling van functies
die toelaten om gemeenschappelijke functionaliteit slechts \'e\'enmaal te
defini\"eren en vervolgens vanuit specifieke toepassingen of algoritmen aan te
spreken.

De introductie van een softwarebibliotheek lost een aantal van de technische
problemen op, maar zal weinig dekking geven voor de overige aspecten. Zo moeten
bij gebruik van een softwarebibliotheek de regels ervan ook effectief gevolgd
worden door alle partijen. De analyse van onderzoeksdocumenten en het koppelen
aan de gebruikte softwarebibliotheek blijft nog steeds een dure en
foutgevoelige taak.

Zelfs indien men bij onderzoek gebruik zou maken van dezelfde
softwarebibliotheek, zou dit nog niet platformonafhankelijk zijn, zou de taal
vastliggen en zou ook bv. het netwerkprotocol of zelfs de sensorknoop
vastliggen. Buiten het feit dat de ontwikkeling makkelijker en beter zou kunnen
gebeuren, levert het geen enkel voordeel op voor de onderzoekers.

\vspace{-3mm}

\section{Domeinspecifieke taal}
\label{section:solution-dsl}

Indien de programmeertaal en een softwarebibliotheek nog te veel vrijheid laten
en geen integratie van onderzoek en ontwikkeling kunnen bewerkstelligen, kan er
gekeken worden naar een andere gemeenschappelijke taal om de algoritmen in uit
te drukken. In klassieke softwaremethodologie\"en wordt hiervoor bv.
\emph{Unified Modeling Language} (UML) \citep{url:uml} of een andere
analysetaal gebruikt.

In het geval van inbraakdetectie voor een WSN, zou de keuze kunnen vallen op
een domeinspecifieke taal (\emph{Domain Specific Language})
(DSL)\citep{van2000domain, mernik2005and, fowler2010domain}. Het domein,
inbraakdetectie in draadloze sensornetwerken, is immers duidelijk afgelijnd en
is beperkt in functionaliteit: knopen kunnen berichten ontvangen en versturen
alsook metingen doen aan de hand van hun sensoren.

Domeinspecifieke talen kunnen veel vormen aannemen. Zo zijn er \emph{inwendige}
talen die binnen een bestaande programmeertaal kunnen gebruikt worden. Ze
gebruiken de mogelijkheden van de omkaderende taal en omgeving om een
bijkomende taal te realiseren die voordelen biedt bovenop het gebruik van de
basistaal. Het is duidelijk dat de keuze voor een inwendige DSL samengaat met
de mogelijkheden van de overkoepelende programmeertaal. De implementatie van
een inwendige DSL in C is mogelijk, maar zal slechts een beperkte meerwaarde
bieden. Daartegenover staat trouwens dat we met een inwendige DSL geen
beperkingen kunnen opleggen aan de omkaderende taal. Alle mogelijkheden van C
zouden ter beschikking blijven en ongewenste constructies zouden nog steeds
kunnen binnendringen.

Een tweede groep van talen zijn de zgn. \emph{uitwendige} talen. Dit zijn talen
die onafhankelijk van een programmeertaal opgebouwd worden en dus volledig vrij
zijn in het bepalen van restricties en mogelijkheden. Een uitwendige DSL kan in
het geval van inbraakdetectie een oplossing zijn om de beschrijving van
detectiealgoritmen te formaliseren. Zo'n beschrijving kan platformonafhankelijk
zijn, waardoor de resultaten van onderzoek direct herbruikbaar en breed
toepasbaar worden.

Door de DSL te laten aansluiten bij C en uit te breiden met constructies die
toelaten om op een hoger niveau met gegevensstructuren om te gaan, ontstaat een
omgeving die zeer nauw aansluit bij zowel onderzoekers als ontwikkelaars.

\vspace{-3mm}

\section{Codegeneratie}
\label{section:solution-codegen}

Eens men beschikt over een formele beschrijving van een algoritme, is de stap
naar codegeneratie niet meer groot. Codegeneratie is geen noodzakelijke stap,
maar blijkt in dit kader toch een meerwaarde.

Zo kan een uitbater van een WSN zelf de software voor zijn netwerk combineren
met een door hem samengestelde selectie aan inbraakdetectiealgoritmen, zonder
beroep te moeten doen op een ontwikkelaar.

De generator staat verder ook in voor het genereren van geoptimaliseerde
technische code. Een voorbeeld hiervan is de zgn. \emph{inlining} van
programmacode, waarbij opgeroepen code uit een functie in de plaats van de
eigenlijke oproep gekopieerd wordt. Dit vraagt misschien enkele bytes
programmacode extra, maar het uitsparen van de functieoproep kan de \mcu
ontlasten.

Dat deze toepassing valabel is, wordt aangetoond door TinyOS
\citep{levis2005tinyos}. De oorspronkelijke nesC code wordt immers herschreven
tot standaard C code, welke vervolgens gecompileerd wordt. Bij deze omzetting
worden nagenoeg alle kleine functieoproepen \emph{inline} geplaatst
\citep{gay2007software}.

Een functionele codegenerator kan hierin echter veel verder gaan en niet louter
op syntactisch/technisch vlak aan \emph{inlining} doen, maar ook op functioneel
vlak. Indien ook de softwarebibliotheek, net zoals de DSL als invoer van de
codegenerator wordt beschouwd, zal deze code inherent mee geoptimaliseerd
kunnen worden en afgestemd zijn op de eigenlijke algoritmen.

\vspace{-3mm}

\section{Dekking probleemdefinitie}
\label{section:problem-coverage}

Met deze oplossingsstrategie dekken we de in sectie
\ref{section:problem-definition} beschreven probleemdefinitie af. Door het
mogelijk te maken om m\'e\'er algoritmen met dezelfde middelen te
implementeren, wordt tegemoet gekomen aan de noodzaak tot detectie. Geen
gebruik maken van een standaardplatform zorgt ervoor dat de oplossing vrij
blijft van externe afhankelijkheden. De domeinspecifieke taal laat toe om op
formele wijze detectiealgoritmen te beschrijven. Samen met de codegeneratie
kunnen de kosten van de ontwikkeling geminimaliseerd worden en kan men
effectief op economisch verantwoorde wijze een IDS toevoegen aan een WSN. Tot
slot biedt de volledig geautomatiseerde oplossing een flexibele oplossing die
kan inspelen op wijzigende situaties.
