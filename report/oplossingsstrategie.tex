%!TEX root=masterproef.tex

\chapter{Oplossingsstrategie}
\label{chapter:oplossingsstrategie}

Uit de probleemstelling in hoofdstuk \ref{chapter:probleemstelling} moeten we
spijtig genoeg concluderen dat er weinig in het voordeel van het bouwen van een
IDS voor een WSN spreekt. Op elk niveau, van de hardware van de knoop en de
elementaire software die het netwerk vormgeeft, tot de onderzoeker,
ontwikkelaar en uitbater van het netwerk, zijn er obstakels te identificeren.

Het probleem is zelfs zo fundamenteel van aard dat het principieel onmogelijk
is om een sluitende oplossing te realiseren, en toch is het belangrijk om aan
inbraakdetectie te doen, zij het niet om toekomstige aanvallen te verijdelen,
maar gewoon al om te weten dat er mogelijk problemen zijn en hiermee rekening
te houden bij de interpretatie van de gegevens die het DSN ons aanbiedt.

Wanneer we de probleemstelling trachten samen te vatten en te komen tot een
oplossingsstrategie, merken we eerst op dat een groot probleem schuilt in de
zwakke verbondenheid van de verschillende niveaus. Het feit dat er geen
draagvlak is voor het formaliseren van het resultaat van het onderzoek naar
inbraakdetectiealgoritmen, resulteert in een gebrekkige doorstroming van deze
kennis doorheen de volledige keten. Het resultaat daarvan laat zich voelen in
het feit dat op dit ogenblik er nagenoeg louter informatie over het onderwerp
kan gevonden worden in academische literatuur.

We mogen ons gelukkig prijzen, want aan de andere kant van de keten is het
probleem nog niet zichtbaar. Er zijn, op enkele grappige incidenten, nog geen
wereldschokken inbraken gepleegd in DSN - of toch niet dat we weten. De
introductie van DSN in onze samenleving is inderdaad al bezig, maar heeft nog
geen kritische massa bereikt.

Maar dat betekent niet dat we vanuit een academische wereld, naast het
aanleveren van losse detectiealgoritmen en gedeeltelijke raamwerken, er niet
voor moeten zorgen dat al dit werk ook effectief bruikbaar is in een
niet-academische context. Als weldra de realiteit de fictie heeft ingehaald en
de eerste berichten van inbraken opduiken, moeten ontwikkelaars en uitbaters
van DSN in staat zijn om op een economisch verantwoorde manier een IDS toe te
voegen aan hun netwerken. Inbraakdetectie is een re\"eel probleem en in de
realiteit spelen economische factoren wel degelijk een rol. Zo fundamenteel het
onderzoek gevoerd wordt, zo fundamenteel moeten de antwoorden die uit dat
onderzoek voortvloeien beantwoorden aan de realiteit waar ze in spelen.

Daarom is het van groot belang dat naast de detectiealgoritmen op zich, er ook
een draagvlak gecre\"eerd wordt, een methodologie naar voor geschoven wordt,
waarop deze detectiealgoritmen kunnen gedijen.

In dit hoofdstuk volgen we opnieuw de verschillende stappen in de keten en
trachten een antwoord op te bouwen dat tegemoet komt aan de ge\"identificeerde
problemen. Elk deel van het antwoord moet een wezenlijke bijdrage leveren en
moet trachten de impact op elk aspect van de keten te minimaliseren.

\section{Hardware en netwerk}
\label{section:solution-node-wsn}

Het feit dat sensorknopen, en embedded systemen in het algemeen, toegankelijk
zijn en daarom inherent vatbaar zijn voor fysieke aanvallen is een probleem
waar op zich weinig aan kan gedaan worden. Zeker zolang de ontwikkeling van
sensorknopen in zijn kinderschoenen staat en de focus nog op de basisbehoeften
ligt (energiebesparing, grootte,...), zolang er geen echte standaard voor deze
hardware bestaat\footnote{Standaarden zullen zonder twijfel ontstaan en de
eerste schuchtere pogingen zijn reeds te zien in de vorm van bv. het Zigduino
platform (\url{http://www.logos-electro.com/store/zigduino-r2}) of de Waspmote
van Libellum (\url{http://www.libelium.com/products/waspmote/}).}, zal het nog
even duren voor de focus verschuift naar de beveiliging van de hardware.

Maar dat tijdperk mogen we niet afwachten en zelfs met een enorme diversiteit
aan hardware en weinig opties op beveiliging op het niveau van hardware, blijft
het belangrijk om alleen al op het vlak van de software te kijken hoe het
niveau van de beveiliging toch kan opgetrokken worden, zodat er een basisniveau
van beveiliging mogelijk is.

Vanuit het oogpunt van de hardware en het netwerk, houden we een belangrijk
criterium voor ogen: de oplossing moet flexibel genoeg zijn om met
verschillende soorten hardware, netwerkprotocollen \dots om te gaan en mag dus
niet inherent afhankelijk zijn van beperkende veronderstellingen op dit niveau.

\section{Elementaire software}
\label{section:solution-software}

Onder elementaire software verstaan we elke vorm van software die nodig is om
de basisbehoeften van het systeem te vervullen. In eerste plaats denken we
daarbij natuurlijk aan een besturingssysteem, maar dat verschilt in het geval
van een sensorknoop een beetje van de klassieke opvatting, op zijn minst al in
grootte en mogelijkheden.

Een besturingssysteem voor een sensorknoop biedt een dunne abstractielaag die
de harde realiteit van de programmatie van een naakte \mcu verlicht door de
introductie van processen, componenten \dots Gegeven de beperkte voorzieningen
van een sensorknoop, focust de ontwikkeling van zulke systemen zich op het
beperken van de eigen impact en zal daarom zoveel mogelijk belast trachten te
vermijden.

Het is daarentegen perfect mogelijk om bovenop zulk een systeem een IDS te
construeren. De meeste systemen zijn open-bron-software en integratie is
mogelijk door diepgaande aanpassingen. Zulke aanpassingen zijn dan wel nodig
voor elk besturingssysteem en opnieuw, zoals bij de hardware, is er nog geen de
facto standaard. TinyOS en Contiki zijn zonder twijfel sterke spelers, maar de
\emph{Windows} onder de besturingssystemen voor sensorknopen moet nog opstaan.

Daarnaast zijn er echter ook nog veel alternatieve oplossingen, gaande van
intern ontwikkelde softwarebibliotheken die dienst doen als besturingssysteem,
tot een eenvoudige zgn. \emph{event loop} die zonder franjes elke
functionaliteit keer op keer aanroept en uitvoert.

Net zoals bij de hardware, moeten we een ook op het vlak van de elementaire
software de behoefte identificeren om flexibel om te springen met dit aspect en
geen van de mogelijkheden of eigenschappen van deze systemen als gegeven te
beschouwen in de oplossing.

\section{Programmeertalen}
\label{section:solution-proglang}

De keuze van een programmeertaal hangt in het geval van een DSN vooral vast aan
de taal die gebruikt wordt door het besturingssysteem. Omdat dit systeem
typisch niet op zich staat en de toepassing samen met het systeem tot een
geheel verwerkt wordt, is de integratie tussen de twee zeer sterk.

De ontwikkelaar van de hardware levert bij zijn sensorknopen, of in
elementairdere zin de \mcu, typisch \'e\'en compiler en derhalve ook \'e\'en
taal. Het aanbod is uitermate beperkt en bestaat in grote lijnen uit ``C''. Er
bestaan alternatieven, maar die zijn schaars. Zo is er bv. de SunSpot van
Oracle, die Java aanbiedt.

De C programmeertaal staat zeer dicht bij de hardware en is duidelijk de eerste
stap in de ontplooiing van softwareontwikkeling op embedded systemen. De
voorzichtige pogingen om nieuwe programmeertalen voor te stellen voor deze
omgeving tonen aan dat er een evolutie op dit vlak in ontwikkeling is, maar
opnieuw zal dit nog tijd vragen vooraleer er een dominante taal C van de troon
zal stoten.

Op dit ogenblik is C de lingua franca voor alle spelers in dit segment: zowel
onderzoek als ontwikkelaar hanteren deze taal, ook al moet de kanttekening
gemaakt worden dat in veel gevallen onderzoekers zich nog beroepen op
simulaties waarbij C ontweken wordt.

Dat op dit ogenblik C de dominante taal is in de wereld van DSN, biedt een
opportuniteit om een oplossing louter te focussen op deze taal. Indien de
oplossing ook taal-onafhankelijk kan zijn, is dit een extra pluspunt.

\section{Softwarebibliotheek}
\label{section:solution-library}

Op een niveau hoger dan programmeertalen, bevinden we ons in het
toepassingsgebied. Hier komt vooral het ontwikkelingsproces en de mensen aan
bod. De problemen die hiermee gepaard gaan werden in de probleemstelling
opgesomd in secties \ref{section:problem-research},
\ref{section:problem-develop} en \ref{section:problem-operations}.

We moeten nu op zoek gaan naar een oplossing die enerzijds voldoet aan de eisen
van hardware, elementaire software en de programmeertaal, maar tevens tracht
een antwoord te bieden aan de problemen die de spelers in het
ontwikkelingsproces ervaren. We trachten dit te doen, door stapsgewijze
introductie van mogelijke oplossingen. De eerste stap is de introductie van een
softwarebibliotheek.
 
In sectie \ref{section:problem-develop} zagen we reeds dat \'e\'en van de
pijnpunten was dat de verschillende algoritmen typisch dezelfde acties
ondernemen. Indien de algoritmes als onafhankelijke modules zouden worden
ontwikkeld en sequentieel achter elkaar worden opgeroepen, zullen zij veel
dubbel werk leveren, die ten koste gaat van de energievoorziening en dus de
levensduur van de knoop.

Dit is een klassiek softwareprobleem en dit wordt normaal opgevangen door het
gebruik van een softwarebibliotheek. Deze biedt een verzameling aan functies
die toelaten om gemeenschappelijk functionaliteit slechts \'e\'enmaal te
defini\"eren en vervolgens vanuit specifieke toepassingen, of algoritmes aan te
spreken. In dit geval zou de gemeenschappelijk functionaliteit kunnen bestaan
uit een \emph{parser} voor het verwerken van binnenkomende berichten, het
accumuleren van uitgaande berichten en het plannen van repetitieve taken.

Het voordeel van een softwarebibliotheek is legio: door hergebruik
gemeenschappelijke functionaliteit wordt er bespaart op opslagruimte voor
programmacode, wordt de \mcu ontlast van herhaaldelijk dezelfde handelingen en
wordt het gebruik van de draadloze radio geoptimaliseerd.

We kunnen deze softwarebibliotheek beschouwen als een uitbreiding van de
elementaire software. Hierdoor zal ze deels afhankelijk zijn van de gebruikte
elementaire software, maar mits goed organisatie zal zelfs over verschillende
elementaire software heen, er hergebruik kunnen bewerkstelligd worden. In ieder
geval zal de softwarebibliotheek steeds afhankelijk zijn van de gebruikte
programmeertaal en indien meerdere talen moeten ondersteund worden, zal de
ontwikkeling van de bibliotheek steeds opnieuw uitgevoerd moeten worden.

De introductie van een softwarebibliotheek lost op zich veel van de technische
problemen op, maar zal nog maar weinig dekking geven voor de overige aspecten
van het probleem. Zo moeten bij gebruik van een softwarebibliotheek de regels
van de softwarebibliotheek ook gevolgd worden, door alle spelers. Het gebruik
van een softwarebibliotheek is slechts een hulpmiddel voor de ontwikkelaar, om
redundant werk te vermijden. De analyse van onderzoeksdocumenten en het mappen
naar de gebruikte softwarebibliotheek blijft nog steeds een taak waar enerzijds
veel werk in dient ge\"investeerd te worden en anderzijds waar nog steeds
fouten kunnen optreden.

Door louter introductie van een softwarebibliotheek, bereiken we tevens slechts
maar een klein gedeelte van de probleemcontext. Het stelt ontwikkelaars in
staat om de implementatie op zijn minst te optimaliseren, maar er is nog steeds
geen link met de analyse in onderzoeksmiddens. De domein experts zijn nog niet
in controle.

Zelfs indien de resultaten van het onderzoek zouden gebruik maken van \'e\'en
zelfde softwarebibliotheek, dan zou dit niet platform-onafhankelijk zijn, zou
de taal vastliggen en zou ook bv. het netwerkprotocol of zelfs de sensorknoop
vastliggen. Buiten het feit dat de ontwikkeling makkelijker en beter zou kunnen
gebeuren, levert het geen enkel voordeel op voor de onderzoekers en zou het
resultaat van hun werk misschien nog restrictiever zijn dan nu.

\section{Domeinspecifieke taal}
\label{section:solution-dsl}

Indien de programmeertaal, C, en een softwarebibliotheek nog te veel vrijheid
laten en geen integratie van onderzoek en ontwikkeling kunnen bewerkstelligen,
kan er gekeken worden naar een andere gemeenschappelijke taal om de algoritmen
in uit te drukken. In klassieke softwareontwerpmethodologie\"en wordt hiervoor
bv. UML \citep{url:uml} of een andere functionele analyse taal gebruikt. Het
gebruik van zulke formele taal laat toe om platform-onafhankelijk een
beschrijving te maken van de beoogde functionaliteit. Typisch wordt hierbij ook
gebruik gemaakt van een basisinfrastructuur die vergelijkbaar is met de
softwarebibliotheek, maar echter wel eerder inhoudelijk functioneel gericht is
en minder focust op technische optimalisatie.

In het geval van inbraakdetectie in DSN, zou de keuze kunnen vallen op een
domeinspecifieke taal (in het Engels: \emph{Domain Specific Language})
(DSL)\citep{van2000domain, mernik2005and, fowler2010domain}. Het domein,
inbraakdetectie in DSN, is duidelijk afgelijnd en is beperkt in
functionaliteit. Knopen kunnen berichten ontvangen en versturen en kunnen
metingen doen aan de hand van hun sensoren. In het kader van het domein is dit
laatste zelfs een nevenactiviteit.

Domeinspecifieke talen kunnen veel vormen aannemen, zo zijn er inwendige talen
die binnen een bestaande programmeertaal kunnen gebruikt worden. Ze gebruiken
de mogelijkheden van de omringende taal en omgeving om een bijkomende taal te
realiseren die voordelen biedt over het gebruik van de basistaal. Elementaire
voorbeelden van inwendige talen maken gebruik van het nesten van
functie-oproepen of het koppelen van methoden om een klassieke applicatie
programmatie interface (API) te verfraaien. Codevoorbeelds \ref{lst:internal-dsl-api}
en \ref{lst:internal-dsl-dsl} tonen de werking aan de hand van een voorbeeld
uit \citep{fowler2010domain}.

\begin{minted}[fontsize=\footnotesize,fontfamily=courier]{java}
Processor p = new Processor(2, 2500, Processor.Type.i386);
Disk d1 = new Disk(150, Disk.UNKNOWN_SPEED, null);
Disk d2 = new Disk(75, 7200, Disk.Interface.SATA);
return new Computer(p, d1, d2);
\end{minted}
\captionof{listing}{Klassiek gebruik van een Java API}
\label{lst:internal-dsl-api}

\begin{minted}[fontsize=\footnotesize,fontfamily=courier]{java}
computer()
  .processor()
    .cores(2)
    .speed(2500)
    .i386()
  .disk()
    .size(150)
  .disk()
    .size(75)
    .speed(7200)
    .sata()
  .end();
\end{minted}
\captionof{listing}{DSL met gekoppelde methodeoproepen}
\label{lst:internal-dsl-dsl}

Het is duidelijk dat de keuze voor een inwendige DSL samengaat met de
mogelijkheden van de overkoepelende programmeertaal. De implementatie van een
inwendige DSL in C is mogelijk, maar zal slechts beperkte mogelijkheden bieden.

Daartegenover staat trouwens dat we met een inwendige DSL geen beperkingen
kunnen opleggen aan de omkaderende taal. Alle mogelijkheden van C zouden ter
beschikking blijven en ongewenste constructies zouden nog steeds kunnen
binnendringen.

Een tweede groep van talen zijn de zgn. uitwendige talen. Dit zijn talen die
onafhankelijk van een volledige programmeertaal opgebouwd worden en dus in alle
vrijheid de nodige restricties of mogelijkheden kunnen bepalen. Dit zijn
letterlijke kleine programmeertalen, maar dan met een syntax en semantiek die
volledig gericht is op het probleemdomein.

Een uitwendige DSL kan in het geval van inbraakdetectie in DSN een goede
oplossing zijn om de beschrijving van detectiealgoritmen te formaliseren. Zo'n
beschrijving kan perfect platform-onafhankelijk zijn, waardoor de intrinsieke
waarde van het onderzoek er naar verder gebruik sterk op vooruitgaat. Indien de
DSL vervolgens tevens nauw aansluit bij bv. de C programmeertaal, zal deze ook
een vlotte adoptie mogelijk maken bij zowel onderzoekers als ontwikkelaars.

Naast de vertrouwde syntax van C - althans een subset ervan - kunnen vervolgens
constructies toegevoegd worden die toelaten om de algoritmes makkelijker
functioneel te beschrijven. In het geval van dit domein is bv. het kunnen
reageren op gebeurtenissen, zoals het ontvangen of versturen van een bericht
nuttige aanknopingspunten om functionele logica aan te verbinden.

Ook het toegankelijker maken van het ondervragen van gegevensstructuren zoals
lijsten \dots zal een toegevoegde waarde opleveren. Inspiratie kan bij andere
talen gevonden worden die zaken als detecteren van patronen ingebouwd hebben of
met annotaties functionaliteit toevoegen aan andere constructies.

De DSL kan verder hand in hand gaan met overeenkomstige implementaties van
gemeenschappelijke functionaliteit, in de vorm van een softwarebibliotheek. De
risico's op analysefouten worden zo verder afgebouwd en de verwerkingstijd kan
opnieuw gereduceerd worden.

\section{Code generatie}
\label{section:solution-codegen}

Eens men beschikt over een formele beschrijving van een algoritme, is de stap
naar code generatie niet meer groot. Code generatie is geen noodzakelijke stap,
maar blijkt in dit kader toch een grote toegevoegde meerwaarde te kunnen
leveren.

Indien immers een uitbater van een DSN in staat is om zelf de software voor
zijn netwerk te combineren met een door hem samengestelde selectie aan
inbraakdetectiealgoritmen en hiervoor niet telkens opnieuw via de
ontwikkelingsfase moet passeren, zal hij in staat zijn om op regelmatige basis
en met verschillende configuraties voor verschillende knopen deze van een
up-to-date IDS te voorzien.

Het is uiteraard een groot economisch voordeel indien er voor de ontwikkeling
geen tussenkomst nodig is van een (manuele) ontwikkelaar.

Een dergelijk platform kan nog verder uitgebouwd worden. Zo kan de code
generator in principe ook ingezet worden om ook de softewarebibliotheek te
genereren. Dit zou enerzijds betekenen dat ook hier niet langer extra werk
bestaat in het opstellen en onderhouden ervan. Maar dit biedt nog een
opportuniteit.

\'E\'en van de eigenschappen van de generator moet tevens zijn het
optimaliseren van de werking van de programmacode. Elementaire voorbeelden zijn
het zgn. \emph{inline}'en van programmacode, waarbij functieoproepen
uitgespaard worden en de opgeroepen code in de plaats van de oproep gekopieerd
wordt. Dit vraagt soms enkele bytes meer programmacode, maar het uitsparen van
de functieoproep kan er voor zorgen dat de \mcu veel minder lang moet werken.

Dat deze toepassing valabel is, wordt aangetoond door TinyOS
\citep{levis2005tinyos}. De oorspronkelijke nesC code wordt immers herschreven
tot standaard C code, welke vervolgens gecompileerd wordt. Bij deze omzetting
worden nagenoeg alle kleine functieoproepen \emph{inline} geplaatst
\citep{gay2007software}.

Een functionele code generator kan hierin echter veel verder gaan en niet
louter op syntactisch/technisch vlak aan \emph{inlining} doen, maar ook op
functioneel vlak. Indien dus de softwarebibliotheek ook, net zoals de DSL code
als invoer van de code generator wordt beschouwd, zal deze code inherent mee
geoptimaliseerd worden en afgestemd zijn op de eigenlijke algoritmes.

\section{Conclusie}
\label{section:solution-conclusion}

Met een DSL en code generatie is de keten ook volledig rond: onderzoekers
kunnen zich focussen op de functionele kant van hun algoritmen door gebruik te
maken van de platform-onafhankelijke DSL.

Deze formele beschrijving kan door ontwikkelaars en zelfs uitbaters van een DSN
direct gebruikt worden door middel van code generatie. Tijdens deze code
generatie wordt de platform-onafhankelijke beschrijving omgezet in een
platfom-specifieke implementatie en wordt ze samengevoegd met een
softwarebibliotheek die overeenkomst met het platform die die optimale
implementaties levert voor het desbetreffende platform.

Maar de mogelijkheden gaan breder: onderzoekers kunnen met de code generatie
tevens een gemeenschappelijk simulatieplatform bouwen, waar nieuwe algoritmen
kunnen samengebracht worden en getest worden op een objectieve en vooral
uniforme manier. Het wordt nu mogelijk om een centrale verzameling van
algoritmes aan te leggen, die uitbaters kunnen raadplegen.

De mogelijkheden die bv. door SNORT worden geboden kunnen op deze manier ook
gerealiseerd worden voor een gedecentraliseerde IDS.
