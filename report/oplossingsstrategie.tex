%!TEX root=masterproef.tex

\chapter{Oplossingsstrategie}
\label{chapter:oplossingsstrategie}

Uit de probleemstelling in voorgaand hoofdstuk moeten we spijtig genoeg
concluderen dat er weinig in het voordeel van het bouwen van een IDS voor een
WSN spreekt. Op elk niveau, van de hardware van de knoop en de elementaire
software die het netwerk vormgeeft, tot de onderzoeker, ontwikkelaar en
uitbater van het netwerk, zijn er obstakels te identificeren. Dit hoofdstuk
volgt opnieuw de verschillende stappen in de keten en tracht een antwoord aan
te bieden dat tegemoet komt aan de ge\"identificeerde problemen.

Sectie \ref{section:solution-node-wsn} stelt dat het ontbreken van een dominant
hardwareplatform er toe leidt dat de oplossing flexibel moet zijn t.o.v.
verschillende platformen.

Sectie \ref{section:solution-software} veralgemeent deze eis naar het niveau
van de software, omdat er bv. ook geen overheersend besturingssysteem
beschikbaar is.

Sectie \ref{section:solution-proglang} evalueert de mogelijkheid om de C
programmeertaal te hanteren als gemeenschappelijke en gestandaardiseerde taal
om detectiealgoritmen te beschrijven.

Sectie \ref{section:solution-library} introduceert softwarebibliotheken als een
oplossing voor gemeenschappelijke logica en het centraliseren van iteratieve
processen.

Sectie \ref{section:solution-dsl} argumenteert dat een domeinspecifieke taal
een afdoende oplossing kan zijn om aan de hand van een formele en
platformonafhankelijke beschrijving van algoritmen een brug te slagen tussen
onderzoek en ontwikkeling.

Sectie \ref{section:solution-codegen} stelt tot slotte dat code generatie een
geformaliseerd proces kan automatiseren en kan tegemoet komen aan de overige
problemen in het kader van het ontwikkelingsproces.

\vspace{-3mm}

\section{Hardware en netwerk}
\label{section:solution-node-wsn}

Het feit dat sensorknopen, en embedded systemen in het algemeen, fysiek
toegankelijk zijn en daarom inherent vatbaar zijn voor fysieke aanvallen is een
probleem waar op zich weinig aan kan gedaan worden. Zeker zolang de
ontwikkeling van sensorknopen in zijn kinderschoenen staat en de focus nog op
de basisbehoeften ligt (energiebesparing, grootte,...), zolang er geen echte
standaard voor deze hardware bestaat\footnote{Standaarden zullen zonder twijfel
ontstaan en de eerste schuchtere pogingen zijn reeds te zien in de vorm van bv.
het Zigduino platform (\url{http://www.logos-electro.com/store/zigduino-r2}) of
de Waspmote van Libellum (\url{http://www.libelium.com/products/waspmote/}).},
zal het nog even duren voor de focus verschuift naar de beveiliging van de
hardware.

Vanuit het oogpunt van de hardware en het netwerk, houden we daarom een
belangrijk criterium voor ogen: de oplossing moet flexibel genoeg zijn om met
verschillende soorten hardware, netwerkprotocollen \dots om te gaan en mag dus
niet inherent afhankelijk zijn van beperkende veronderstellingen op dit niveau.

\vspace{-3mm}

\section{Elementaire software}
\label{section:solution-software}

Onder elementaire software verstaan we elke vorm van software die nodig is om
de basisbehoeften van het systeem te vervullen. In eerste plaats denken we
daarbij natuurlijk aan een besturingssysteem, maar dat verschilt in het geval
van een sensorknoop een beetje van de klassieke opvatting, op zijn minst al in
grootte en mogelijkheden. Maar het kan ook ze eenvoudig zijn als een
\emph{event loop}.

Een besturingssysteem voor een sensorknoop biedt een dunne abstractielaag die
de harde realiteit van de programmatie van een naakte \mcu verlicht door de
introductie van processen, componenten \dots Gegeven de beperkte voorzieningen
van een sensorknoop, focust de ontwikkeling van zulke systemen zich op het
beperken van de eigen impact en zal daarom zoveel mogelijk belast trachten te
vermijden.

De meeste van deze systemen zijn open-bron-software en integratie is mogelijk
door diepgaande aanpassingen. Zulke aanpassingen zijn dan wel nodig voor elk
besturingssysteem en net zoals bij de hardware, is er nog geen de facto
standaard. TinyOS en Contiki zijn zonder twijfel sterke spelers, maar de
\emph{Windows} onder de besturingssystemen voor sensorknopen moet nog opstaan.

\vspace{-3mm}

\section{Programmeertalen}
\label{section:solution-proglang}

De keuze van een programmeertaal hangt in het geval van een DSN vooral vast aan
de taal die gebruikt wordt door het besturingssysteem. Omdat dit systeem
typisch niet op zich staat en de toepassing samen met het systeem tot een
geheel verwerkt wordt, is de integratie tussen de twee zeer sterk.

De ontwikkelaar van de hardware levert bij zijn sensorknopen, of in
elementairdere zin de \mcu, typisch \'e\'en compiler en derhalve ook \'e\'en
taal. Het aanbod is uitermate beperkt en bestaat in grote lijnen uit ``C''. Er
bestaan alternatieven, maar die zijn schaars. Zo is er bv. de SunSpot van
Oracle, die Java aanbiedt.

De C programmeertaal staat zeer dicht bij de hardware en is duidelijk de eerste
stap in de ontplooiing van softwareontwikkeling op embedded systemen. De
voorzichtige pogingen om nieuwe programmeertalen voor te stellen voor deze
omgeving tonen aan dat er een evolutie op dit vlak in ontwikkeling is, maar
opnieuw zal dit nog tijd vragen vooraleer er een dominante taal C van de troon
zal stoten.

Op dit ogenblik is C de lingua franca voor alle spelers in dit segment: zowel
onderzoek als ontwikkelaar hanteren deze taal, ook al moet de kanttekening
gemaakt worden dat in veel gevallen onderzoekers zich nog beroepen op
simulaties waarbij C ontweken wordt.

\vspace{-3mm}

\section{Softwarebibliotheek}
\label{section:solution-library}

In sectie \ref{section:problem-develop} zagen we reeds dat \'e\'en van de
pijnpunten was dat de verschillende algoritmen typisch dezelfde acties
ondernemen. Indien de algoritmen als onafhankelijke modules zouden worden
ontwikkeld en sequentieel achter elkaar worden opgeroepen, zullen zij veel
dubbel werk leveren, die ten koste gaat van de energievoorziening en dus de
levensduur van de knoop.

Dit is een klassiek softwareprobleem en dit wordt normaal opgevangen door het
gebruik van een softwarebibliotheek. Deze biedt een verzameling aan functies
die toelaten om gemeenschappelijk functionaliteit slechts \'e\'enmaal te
defini\"eren en vervolgens vanuit specifieke toepassingen, of algoritmen aan te
spreken.

De introductie van een softwarebibliotheek lost op zich een aantal van de
technische problemen op, maar zal nog maar weinig dekking geven voor de overige
aspecten van het probleem. Zo moeten bij gebruik van een softwarebibliotheek de
regels van de softwarebibliotheek ook effectief gevolgd worden door alle
partijen. De analyse van onderzoeksdocumenten en het koppelen aan de gebruikte
softwarebibliotheek blijft nog steeds een kostelijke en foutgevoelige taak.

Zelfs men in onderzoek gebruik zou maken van dezelfde softwarebibliotheek, dan
zou dit nog niet platform-onafhankelijk zijn, zou de taal vastliggen en zou ook
bv. het netwerkprotocol of zelfs de sensorknoop vastliggen. Buiten het feit dat
de ontwikkeling makkelijker en beter zou kunnen gebeuren, levert het geen enkel
voordeel op voor de onderzoekers.

\vspace{-3mm}

\section{Domeinspecifieke taal}
\label{section:solution-dsl}

Indien de programmeertaal en een softwarebibliotheek nog te veel vrijheid laten
en geen integratie van onderzoek en ontwikkeling kunnen bewerkstelligen, kan er
gekeken worden naar een andere gemeenschappelijke taal om de algoritmen in uit
te drukken. In klassieke softwareontwerpmethodologie\"en wordt hiervoor bv. UML
\citep{url:uml} of een andere analysetaal gebruikt.

In het geval van inbraakdetectie in DSN, zou de keuze kunnen vallen op een
domeinspecifieke taal (\emph{Domain Specific Language})
(DSL)\citep{van2000domain, mernik2005and, fowler2010domain}. Het domein,
inbraakdetectie in DSN, is immers duidelijk afgelijnd en is beperkt in
functionaliteit: knopen kunnen berichten ontvangen en versturen en kunnen
metingen doen aan de hand van hun sensoren.

Domeinspecifieke talen kunnen veel vormen aannemen, zo zijn er inwendige talen
die binnen een bestaande programmeertaal kunnen gebruikt worden. Ze gebruiken
de mogelijkheden van de omringende taal en omgeving om een bijkomende taal te
realiseren die voordelen biedt over het gebruik van de basistaal. Het is
duidelijk dat de keuze voor een inwendige DSL samengaat met de mogelijkheden
van de overkoepelende programmeertaal. De implementatie van een inwendige DSL
in C is mogelijk, maar zal slechts beperkte mogelijkheden bieden. Daartegenover
staat trouwens dat we met een inwendige DSL geen beperkingen kunnen opleggen
aan de omkaderende taal. Alle mogelijkheden van C zouden ter beschikking
blijven en ongewenste constructies zouden nog steeds kunnen binnendringen.

Een tweede groep van talen zijn de zgn. uitwendige talen. Dit zijn talen die
onafhankelijk van een programmeertaal opgebouwd worden en dus volledig vrij
zijn in het bepalen van restricties en mogelijkheden. Een uitwendige DSL kan in
het geval van inbraakdetectie in DSN een oplossing zijn om de beschrijving van
detectiealgoritmen te formaliseren. Zo'n beschrijving kan platformonafhankelijk
zijn, waardoor de resultaten van onderzoek direct herbruikbaar en breed
toepasbaar worden.

Door de DSL te laten aansluiten bij C en uit te breiden met constructies die
toelaten om op een hoger niveau met gegevensstructuren om te gaan, ontstaat een
omgeving die zeer dicht aansluit bij zowel onderzoekers als ontwikkelaars.

\vspace{-3mm}

\section{Code generatie}
\label{section:solution-codegen}

Eens men beschikt over een formele beschrijving van een algoritme, is de stap
naar code generatie niet meer groot. Code generatie is geen noodzakelijke stap,
maar blijkt in dit kader toch een grote toegevoegde meerwaarde te kunnen
leveren.

Zo kan een uitbater van een DSN zelf de software voor zijn netwerk combineren
met een door hem samengestelde selectie aan inbraakdetectiealgoritmen, zonder
beroep te moeten doen op een ontwikkelaar.

De generator staat verder ook in voor het genereren van geoptimaliseerde
technische code. Een voorbeeld hiervan is het zgn. \emph{inline}'en van
programmacode, waarbij opgeroepen code uit een functie in de plaats van de
eigenlijk oproep gekopieerd wordt. Dit vraagt soms enkele bytes meer
programmacode, maar het uitsparen van de functieoproep kan de \mcu ontlasten.

Dat deze toepassing valabel is, wordt aangetoond door TinyOS
\citep{levis2005tinyos}. De oorspronkelijke nesC code wordt immers herschreven
tot standaard C code, welke vervolgens gecompileerd wordt. Bij deze omzetting
worden nagenoeg alle kleine functieoproepen \emph{inline} geplaatst
\citep{gay2007software}.

Een functionele code generator kan hierin echter veel verder gaan en niet
louter op syntactisch/technisch vlak aan \emph{inlining} doen, maar ook op
functioneel vlak. Indien dus de softwarebibliotheek ook, net zoals de DSL code
als invoer van de code generator wordt beschouwd, zal deze code inherent mee
geoptimaliseerd worden en afgestemd zijn op de eigenlijke algoritmen.

\vspace{-3mm}

\section{Dekking probleemdefinitie}
\label{section:problem-coverage}

Met deze oplossingsstrategie dekken we de in sectie
\ref{section:problem-definition} beschreven probleem-definitie af. Door het
mogelijk te maken om m\'e\'er algoritmen met dezelfde middelen te kunnen
implementeren, wordt de noodzaak tot detectie ingelost. Geen gebruik maken van
een standaardplatform zorgt ervoor dat de oplossing vrij blijft van externe
afhankelijkheden. De domeinspecifieke taal biedt de laat toe om op formele
wijze detectiealgoritmen te beschrijven. Samen met de codegeneratie kunnen de
kosten van de ontwikkeling geminimaliseerd worden en kan men effectief op
economisch verantwoorde wijze een IDS toevoegen aan een DSN. Tot slot biedt de
volledig geautomatiseerde oplossing een flexibele oplossing die kan inspelen op
wijzigende situaties.
