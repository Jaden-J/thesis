%!TEX root=masterproef.tex

\section{FOO-lib}
\label{section:devel-foo-lib}

Een laatste aspect van de implementatie bestaat uit de softwarebibliotheek die
de gegenereerde code aanvult. Deze bibliotheek biedt typisch twee types van
ondersteunende modules aan:

\begin{description}

\item[Algemene en abstraherende modules] Deze kunnen vergeleken worden met de
standaard-bibliotheek die tegenwoordig bij veel programmeertalen gevoegd wordt
en veelal de kracht ervan bepaalt. In het geval van deze eerste implementatie
betreft het hier de \ttt{crypto}-module met functies om cryptografische
sleutels te verwerken, een functionaliteit die typisch voorkomt bij
implementatie van detectiealgoritmen. De \ttt{time}-module biedt toegang tot
een klok en zorgt hiermee voor een abstractielaag naar het onderliggende
systeem.

\item[Knoopdomein-geori\"enteerde modules] Het uitgangspunt van deze
masterproef is dat het mogelijk is om gemeenschappelijke logica betreffende
sensorknopen te centraliseren, om z\'o een meer optimale werking te bekomen.
Het is dan ook niet verwonderlijk dat FOO-lib deze functionaliteit herbergt.
Het betreft hier een gemeenschappelijke parser, een planningsmodule voor
functieoproepen\dots

\end{description}

Vandaag dient er voor elk platform en programmeertaal een implementatie van
FOO-lib te gebeuren. Indien deze code verder zou omgezet worden in een vorm die
rechtstreeks behandeld kan worden de generator, zou deze logica mee verwerkt
kunnen worden en zou bv. \emph{inlining} kunnen toegepast worden, waardoor nog
verdere optimalisatie kan bereikt worden. Hierbij zou dan ook het onderhouden
van verschillende implementaties van FOO-lib sterk beperkt worden tot een lager
niveau van ondersteuning van de platformen en de talen.
