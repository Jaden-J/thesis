%!TEX root=masterproef.tex

\section{FOO-lib}
\label{section:devel-foo-lib}

Een laatste aspect van de implementatie bestaat uit de softwarebibliotheek die
de generatie vervolledigt. Deze bibliotheek biedt typisch twee types van
ondersteunende modules aan:

\begin{description}

\item[Algemene modules] - Dit kan vergeleken worden met de standaard
bibliotheek die tegenwoordig bij veel programmeertalen komt en veelal de kracht
er van bepaald. In het geval van deze eerste implementatie betreft het hier de
\ttt{crypto} module met functies om cryptografische sleutels te bepalen. Dit is
functionaliteit die te verwachten is bij implementatie van detectiealgoritmen.
De \ttt{time} module biedt toegang een klok en zorgt hiermee voor een
abstractielaag naar het onderliggende systeem.

\item[Knoopdomein-geori\"enteerde modules] - Het uitgangspunt van deze
masterproef is dat het mogelijk is om gemeenschappelijke logica betreffende
sensorknopen te centraliseren, om ze een meer optimale werking te bekomen. Het
is dan ook niet verwonderlijke dat FOO-lib de plaats is waar we deze
functionaliteit terugvinden. Het betreft hier de gemeenschappelijke parser voor
binnenkomende berichten, de planningsmodule voor planbare functieoproepen\dots

\end{description}

Vandaag dient er voor elk platform en programmeertaal een implementatie van
FOO-lib te gebeuren. Echter, indien deze code verder zou omgezet worden in een
vorm die rechtstreeks bruikbaar is voor de generator, zou deze logica mee
verwerkt kunnen worden door de generator, zou bv. \emph{inlining} kunnen
toegepast worden en zou nog verdere optimalisatie kunnen bereikt worden.

Misschien nog belangrijk is dat het onderhouden van verschillende
implementaties van FOO-lib sterk beperkt zou worden tot een nog lager niveau
van ondersteuning van de platformen en de talen.
