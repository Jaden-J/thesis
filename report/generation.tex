%!TEX root=masterproef.tex

\section{Generatie}
\label{section:generation}

Alle voorgaande structurele componenten zorgen samen voor het generatieproces.
Dit proces accepteert FOO-lang broncode, importeert die in het SM,
transformeert die in een CM en uiteindelijk wordt dat CM uitgevoerd als
programmacode.

Bijlage \ref{appendix:hello-srcs} bevat de belangrijkste gegenereerde code voor
het elementaire voorbeeld, \ttt{hello.foo}. We overlopen enkele aspecten van
deze laatste stap in het generatieproces en tonen hoe de verschillende
componenten bijdragen tot sommige constructies.

\subsection{main.c}

De platformimplementatie van het prototype gaat uit van een eenvoudige opzet,
met een expliciete \emph{event loop} (regels 18 tot 24). Verder voorziet de
generatie met commentaar instructies voor de gebruiker om applicatiespecifieke
code toe te voegen of om het gebruikte onderliggende raamwerk te initialiseren
(zie respectievelijk regels 9 en 4 in de broncode van \ttt{main.c}).

We merken vooral de functie-oproepen op naar functies met een
``\ttt{nodes\_}''-prefix. Deze gaan naar de \ttt{nodes}-module die deel
uitmaakt van de FOO-lib softwarebibliotheek. Op regel 28 zien we bv. de
registratie van de uitvoerstrategie die na het verstrijken van een intervaltijd
telkens de functie \ttt{step} zal oproepen.

Die \ttt{interval}-constante is gedefinieerd in het \ttt{constants.h} bestand,
samen met eventuele, andere constanten. De generator tracht om functioneel
verwante zaken bij elkaar te plaatsen. Zo worden alle constanten in \'e\'en
bestand verzameld, maar ook het importeren van functionaliteit wordt
gecentraliseerd om de te genereren code eenvoudiger te maken. Zo worden bv.
alle nodige \emph{include}-instructies ook in \'e\'en \ttt{includes.h} bestand
samengebracht.

\subsection{node\_t.h en node\_t.c}

H\'et centrale gegeven is natuurlijk de voorstelling van een sensorknoop. De
generator zal de informatie uit verschillende FOO-lang-modules trachten samen
te brengen tot \'e\'en gemeenschappelijke voorstelling.

Standaard heeft een knoop twee eigenschappen, nodig voor de interne werking van
de \ttt{nodes}-module: een unieke, interne identificatie (\ttt{id}) en het
netwerkadres van de knoop (\ttt{address}). Daarnaast voegt de generator alle
extra eigenschappen toe, maakt van het geheel een structuur en definieert een
\ttt{node\_t}-type.

Naast de declaratie van het type, wordt eveneens een functie voorzien om een
nieuwe knoop te initialiseren. Deze functie bevat opdrachten die overeenkomen
met de gedefinieerde initi\"ele waarden van de eigenschappen.

\subsection{nodes-\emph{module}.h en nodes-\emph{module}.c}

Tot slot zal de generator het \ttt{nodes}-domein de kans bieden om voor elke
FOO-lang-module een CM module te maken. Hierin wordt de functionaliteit
ondergebracht die eigen is aan de module. Typisch vinden we hier de functies
terug die eerder gekoppeld werden aan een uitvoerstrategie. In het geval van
het voorbeeld is dit de \ttt{step}-functie.

\subsection{Communicatie}

Aan het elementaire voorbeeld ontbreekt \'e\'en belangrijk aspect:
communicatie. Er worden geen berichten verstuurd, noch ontvangen. De generator
zal het versturen van berichten delegeren naar de \ttt{nodes}-module en zal
voor het verwerken van ontvangen berichten functies registreren bij diezelfde
module. Deze worden opgeroepen indien een binnenkomend bericht voldoet aan de
eisen voor die specifieke verwerkingsfunctie.

Codevoorbeelden \ref{lst:comm.foo} en \ref{lst:comm.c} illustreren het typische
communicatiepatroon en de overeenkomstige generatie. Via een uitvoerstrategie
wordt een (anonieme) functie gekoppeld aan het ontvangen van een bericht door
een knoop. Vervolgens wordt gecontroleerd op verschillende mogelijke patronen.
Een bericht dat een \ttt{heartbeat}-atoom bevat, zal de volgende bytes koppelen
aan drie variabelen: \ttt{time}, \ttt{sequence} en \ttt{signature}.

\begin{listing}[ht]
  \begin{minted}[linenos,frame=lines,framesep=2mm,fontsize=\footnotesize]{javascript}
after nodes receive do function(me, sender, from, hop, to, payload) {
  // payload is a list of data. we can consider one or more cases
  case payload {
    // e.g. we can check if we find an atom and three variables after is
    contains( [ #heartbeat, time:timestamp, sequence, signature:byte[20] ] ) {
      ...
    }
  }
}
  \end{minted}
  \vspace{-5mm}
  \caption{Verwerking van een binnenkomend bericht in FOO-lang}
  \label{lst:comm.foo}
\end{listing}

\begin{listing}[ht]
  \begin{minted}[linenos,frame=lines,framesep=2mm,fontsize=\footnotesize]{c}
void init(void) {
  ...
  payload_parser_register(nodes_process_incoming_case_0, 2, 0x00, 0x01);
  ...
}
...
void nodes_process_incoming_case_0(node_t* me, node_t* sender, node_t* from,
                                   node_t* hop, node_t* to, payload_t* payload) {
  // extract variables from payload
  uint32_t time = payload_parser_consume_timestamp();
  uint8_t sequence = payload_parser_consume_byte();
  uint8_t* signature = payload_parser_consume_bytes(20);
  ...
}
  \end{minted}
  \vspace{-5mm}
  \caption{Gegenereerde code voor een binnenkomend bericht}
  \label{lst:comm.c}
\end{listing}

De generator zal deze behandeling van een binnenkomend bericht detecteren, uit
elkaar halen en structureren aan de hand van een functie en diens registratie.

Het atoom is hier omgezet in twee opeenvolgende bytes: \ttt{0x00, 0x01}.
Wanneer de algemene parser deze sequentie ontmoet, zal
\ttt{nodes\_process\_incoming\_case\_0} opgeroepen worden. De signatuur van
deze functie bevat alle informatie over het bericht ivm afzender,
bestemmeling\dots

De overige variabele componenten van het gezochte patroon vinden we hier ook
terug: de drie variabelen worden gedeclareerd en ge\"initialiseerd door middel
van een aantal functies die de volgende bytes uit het bericht zullen omzetten
naar het gewenste type.

\subsection{\emph{Tuples}, lijsten en meer patroonherkenning}

Net zoals het \ttt{node\_t}-type, worden voor \emph{tuples} eveneens structuren
aangemaakt en functies gegenereerd die de behandeling van het \emph{tuple}-type
toelaten. Alle declaraties van de types en de definities van de manipulerende
functies worden samengebracht in respectievelijk \ttt{tuples.h} en
\ttt{tuples.c}. Codevoorbeeld \ref{lst:tuples.h} toont de typische
module-interface van een gegenereerde \emph{tuple}.

\begin{listing}[ht]
  \begin{minted}[linenos,frame=lines,framesep=2mm,fontsize=\footnotesize]{c}
typedef struct tuple_0_t {
  uint32_t elem_0;
  payload_t* elem_1;
  struct tuple_0_t* next;
} tuple_0_t;
tuple_0_t* make_tuple_0_t(uint32_t elem_0, payload_t* elem_1);
void free_tuple_0_t(tuple_0_t* tuple);
tuple_0_t* copy_tuple_0_t(tuple_0_t* source);
  \end{minted}
  \vspace{-5mm}
  \caption{Gegenereerde code voor een \emph{tuple}}
  \label{lst:tuples.h}
\end{listing}

Aangezien een \emph{tuple} in FOO-lang gedefinieerd wordt op basis van types,
worden standaardnamen gebruikt voor de elementen. Tevens voorziet de structuur
een verwijzing naar zichzelf, om de constructie van lijsten toe te laten. Drie
functies bieden de mogelijkheid om een \emph{tuple} aan te maken, vrij te geven
of te kopi\"eren.

\emph{Tuples} worden meestal in combinatie met lijsten gebruikt. Codevoorbeeld
\ref{lst:lists.c} toont enkele fragmenten uit de generatie van functies om lijsten
te onderhouden.

\begin{listing}[ht]
  \begin{minted}[linenos,frame=lines,framesep=2mm,fontsize=\footnotesize]{c}
void list_of_tuple_0_ts_push(tuple_0_t** list, tuple_0_t* item) {
  item->next = *list;
  *list = item;
}
...
uint16_t list_of_tuple_0_ts_remove_match_lt_now(tuple_0_t** list) {
  ...
  while((iter != NULL)) {
    if((iter->elem_0 < now())) {
      ...
    }
  }
}
  \end{minted}
  \vspace{-5mm}
  \caption{Gegenereerde code voor manipulatie van lijsten}
  \label{lst:lists.c}
\end{listing}

De eerste functie voegt een instantie van het \emph{tuple}-type toe aan een
lijst van dat type. Hierbij wordt de eerder gedefinieerde verwijzing gebruikt.

De tweede functie is iets complexer. Het betreft een functie om een element uit
diezelfde lijst te verwijderen. De conditie om zo'n element te verwijderen is
verwerkt in de functie, zowel in de naam als in de logica. De oorsprong hiervan
kunnen we terugvinden in de overeenkomstige FOO-lang broncode:
\mint{javascript}| failures = node.queue.remove([ < now(), _ ]) |

Met deze opdracht wordt de \ttt{queue}-eigenschap van een knoop-object
geraadpleegd. De eigenschap is gedeclareerd als een lijst van het voorgaande
\emph{tuple}-type. Het argument van de \ttt{remove}-methode, is een patroon dat
bestaat uit een conditie \ttt{< now()} en een ``\_'' die syntactisch aanduidt
dat alle waarden op deze positie in het \emph{tuple} aanvaardbaar zijn.

De generator zal zo'n patroon analyseren en trachten om specifieke code te
genereren. Een alternatief zou zijn om generieke lijst-functies op te nemen in
FOO-lib. Vervolgens zouden de condities, ge\"implementeerd als kleine
gegenereerde functies, meegegeven kunnen worden aan de generieke functies. Deze
kleine functies zouden dan opgeroepen kunnen worden om de effectieve condities
te testen.

In het prototype is gekozen om een voorbeeld te introduceren van
\emph{inlining}. Het gebruiken van functieverwijzingen zou het genereren
eenvoudiger maken, doch de kost die gepaard gaat met het herhaaldelijk oproepen
van de verwijzing naar de functie met de conditie zou een serieuze belasting
voor de \mcu met zich meebrengen. In het kader van dit probleem werden testen
gedaan die uitwezen dat het verschil tussen het gebruik van een verwijzing en
het \emph{inline}'en van deze code kan oplopen tot een factor 3000.

\subsection{Conclusie}

Dankzij codegeneratie vanaf een functioneel niveau is het mogelijk om dit soort
van optimalisaties te realiseren. Dit illustreert de kracht en de
opportuniteiten die codegeneratie voor dit soort van problemen kan bieden.

De generatiepatronen die ge\"implementeerd werden in dit prototype zijn niet
volledig en niet geoptimaliseerd. Alleen hier al liggen grote verdere kansen
tot verbetering. De doelstelling was om code te genereren die vergelijkbaar was
met overeenkomstige manuele code. In een volgend hoofdstuk willen we immers
beide implementaties vergelijken en zien of de theoretische winst die de beter
georganiseerde gegenereerde code zou moeten opleveren, zich ook effectief in
praktijk manifesteert.
