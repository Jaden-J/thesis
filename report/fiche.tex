%!TEX root=masterproef.tex

\setup{
  title={Verlagen van de impact van inbraakdetectie
         in draadloze sensornetwerken
         door middel van een domeinspecifieke taal
         en codegeneratietechnieken},
  author={Christophe Van Ginneken},
  promotor={Prof.\,dr.\,ir.\ Wouter Joosen \and Prof.\,dr.\,ir.\ Christophe Huygens},
  assessor={\TODO \and \TODO},
  assistant={Drs.\,ir.\ Jef Maerien}
}

\setup{
  filingcard,
  translatedtitle={Lowering the Impact of Intrusion Detection
                   in Wireless Sensor Networks
                   using a Domain Specific Language
                   \& Code Generation Techniques},
  udc=621.3,
  shortabstract={
Het introduceren van inbraakdetectie in draadloze sensornetwerken resulteert al
snel in een gevecht om middelen: een draadloze sensorknoop beschikt over een
beperkte autonomie en moet zijn energie optimaal benutten. Bijkomende
niet-functionele inbraakbeveiliging vraagt veel van de beschikbare middelen en
bedreigt daarmee de kans om opgenomen te worden in het uiteindelijke ontwerp
van elk nieuwe draadloze sensorknoop. Omdat draadloze sensornetwerken met rasse
schreden onze persoonlijke levenssfeer binnentreden, moeten we echter een
degelijke beveiliging eisen. Preventie is een eerste stap, maar niet alle
inbraken kunnen verijdeld worden. Soms moeten we genoegen nemen met het kunnen
detecteren van inbraken om ons er in toekomstige situaties beter tegen te
wapenen. Indien het probleem niet kan vermeden worden, moeten we trachten het
draaglijker te maken. Deze masterproef wil zowel de druk op de middelen van de
sensorknopen verlichten als de bijkomende economische druk op de ontwikkeling,
die de introductie van inbraakdetectie met zich meebrengt, reduceren. Om dit te
realiseren wordt een domeinspecifieke taal voorgesteld die onderzoekers in
staat stelt om algoritmen voor inbraakdetectie op een formele en
platformonafhankelijke manier te defini\"eren. Deze eerste stap ontslaat
ontwikkelaars van nieuwe sensornetwerken van de taak om onderzoeksliteratuur te
doorworstelen en algoritmen uit deze teksten te puren.Een formele beschrijving
laat verder toe om deze op geautomatiseerde wijze te benaderen. Zo wordt het
mogelijk om door middel van codegeneratie de algoritmen automatisch om te
zetten in platformspecifieke programmacode, zo georganiseerd dat de middelen
van de sensorknoop zo optimaal mogelijk benut worden. Dankzij het vrijwaren van
de middelen van de sensorknoop en het reduceren van de economische kost, wordt
het mogelijk om meer inbraakpogingen te detecteren.
}}
