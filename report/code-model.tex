%!TEX root=masterproef.tex

\subsection{Code model}
\label{subsection:devel-code-model}

Het CM staat in feite volledig los van de generator en het SM. Het is dan ook
als een op zich staand project ontwikkeld, als generieke beschrijving van
uitvoerbare code, genaamd \emph{CodeCanvas}.

\subsubsection{CodeCanvas}

CodeCanvas biedt een API om hi\"erarchische structuren te bouwen. Deze kunnen
opgebouwd worden uit zelf te defini\"eren entiteiten. Standaard voorziet
CodeCanvas de concepten \emph{unit}, \emph{module}, \emph{sectie} en
\emph{code}.

\begin{description}[itemsep=1pt, topsep=2pt, partopsep=2pt]

  \item[Unit] is het hoogste niveau en verzamelt alle onderliggende
  \emph{modules}.
  
  \item[Module] komt overeen met een functioneel geheel en bestaat uit
  \emph{secties}.
  
  \item[Sectie] wordt functioneel ingevuld met \emph{code}instructies. Er
  worden standaard twee secties binnen een module aangemaakt: \'e\'en voor
  declaraties en \'e\'en voor definities.
  
  \item[Code] is een basisbouwsteen voor instructies. Standaard kan hier een
  tekstuele inhoud aan gegeven worden. Praktisch zal men overerven van deze
  klasse en er een taxonomie mee opbouwen.

\end{description}

\noindent Daarnaast biedt CodeCanvas de mogelijkheid om entiteiten te markeren
met een label (\emph{tag}) en onderliggende entiteiten te selecteren of zoeken
op basis van die labels.

Dankzij een \emph{vloeiende} (\emph{fluent}) interface laat CodeCanvas toe om
zeer leesbare operaties te formuleren op deze hi\"erarchische codestructuren.
Codevoorbeeld \ref{lst:codecanvas-hello} toont een eenvoudig voorbeeld dat de
belangrijkste functionaliteit van CodeCanvas toepast.

\begin{listing}[ht]
  \begin{minted}[linenos,frame=lines,framesep=2mm,fontsize=\footnotesize]{python}
from structure import Unit, Module
import instructions as code
import languages.C  as C

unit = Unit().append( Module("hello") )
main = unit.select("hello", "dec").append(code.Function(name="main"))
main.append(code.Print("Hello World\n"))

print str(unit)
print C.Emitter().emit(unit)
print str(unit)
  \end{minted}
  \vspace{-5mm}
  \caption{Werking van het \emph{CodeCanvas}}
  \label{lst:codecanvas-hello}
\end{listing}

Het voorbeeld construeert op regel 5 een \emph{unit} en voegt er een
\emph{module}, genaamd \ttt{hello}, aan toe. Achterliggend worden bij de
aanmaak van een module onmiddellijk 2 secties toegevoegd, genaamd \ttt{dec} en
\ttt{def}.

Op regel 6 wordt vanaf de \emph{unit} de \ttt{dec}-\emph{sectie} geselecteerd.
De \ttt{select}-methode laat toe om een opeenvolgende reeks van \emph{labels}
te defini\"eren die het pad vormen vanaf de startentiteit tot de te selecteren
entiteit. Een gelijkaardige methode, \ttt{find}, accepteert een variabele lijst
argumenten en zoekt vervolgens, vertrekkende van de startentiteit, naar
entiteiten die alle opgegeven \emph{labels} dragen.

Beide methodes kunnen lijsten van entiteiten teruggeven. Op deze lijsten kunnen
eveneens alle methoden opgeroepen worden zoals op een enkele entiteit. Dit
resulteert in een zeer transparante interface.

Aan de geselecteerde sectie wordt vervolgens een functie toegevoegd met de naam
\ttt{main}. Op regel 7 wordt aan deze functie een \ttt{print}-opdracht
toegevoegd. Tot slot wordt de \emph{unit} op twee manieren uitgevoerd: eerst
door er een tekstuele voorstelling van te maken en in tweede instantie door
gebruik te maken van een programmeertaal, in dit geval C.

De uitvoer van dit programma is weergegeven in \ref{lst:codecanvas-output} en
toont eerst de technische tekstuele voorstelling van de hi\"erarchie. Tussen
vierkante haakjes staan de \emph{labels} die aan een entiteit verbonden zijn.
Effectieve instructie-implementaties tonen hun parameters als een lijst van
namen en de waarden.

\begin{listing}[ht]
  \begin{minted}[linenos,frame=lines,framesep=2mm,fontsize=\footnotesize]{console}
  Module hello [hello]
    Section def [def]
    Section dec [dec]
      Function {'params': (), 'type': void, 'id': main}
        Print {'args': (), 'string': "Hello World\n"}

  void main(void);
  #import <stdio.h>
  void main(void) {
    printf("Hello World\n");
  }
  
  Module hello [hello]
    Section def [def]
      Prototype {'params': [], 'type': void, 'id': main}
    Section dec [dec]
      Import {'imported': '<stdio.h>'} [import_stdio] <sticky>
      Function {'params': [], 'type': void, 'id': main}
        Print {'args': (), 'string': "Hello World\n"}
  \end{minted}
  \vspace{-5mm}
  \caption{Uitvoer van voorbeeld werking van het \emph{CodeCanvas}}
  \label{lst:codecanvas-output}
\end{listing}

Het tweede deel van de uitvoer toont de overeenkomstige C-code voor deze
hi\"erarchie. We merken op dat op regel 7 een prototype en op regel 8 een
\emph{import}-opdracht verschijnen die initieel niet in de hi\"erarchie
voorkwamen. Wanneer we een tweede maal de \emph{unit} omvormen tot een
tekstuele voorstelling, zien we deze twee entiteiten wel opduiken.

De uitvoermodule voor de programmeertaal C werkt in meerdere stappen: tijdens
een eerste fase wordt de hi\"erarchie doorlopen en worden uitbreidingen gedaan.
In dit voorbeeld gebeuren er twee: wanneer een \emph{print}-opdracht gevonden
wordt, wordt een \emph{import}-opdracht toegevoegd die de declaraties van
\ttt{stdio.h} zal inladen. Een tweede transformatie zal voor elke
\emph{functie} een prototype aanmaken in de declaratiesectie van de module.

Deze fase zorgt ook voor het omzetten van constructies die niet standaard
ondersteund worden in constructies die wel mogelijk zijn in de beoogde taal.
Een voorbeeld hiervan zijn bv. \emph{tuples}. Deze worden door de transformatie
naar C herschreven aan de hand van structuren en functies om deze structuren te
behandelen.

In een tweede fase doorloopt de uitvoermodule opnieuw de volledige
hi\"erarchie, maar vormt nu elke entiteit om in de overeenkomstige tekstuele
C-syntax.

Beide fasen worden ge\"implementeerd door middel van \emph{visitors}. Deze zijn
ook beschikbaar van buitenaf en laten toe om andere transformaties te
implementeren.

\subsubsection{Filosofie}

De doelstelling van CodeCanvas is het aanbieden van een API die toelaat om te
werken zoals een programmeur denkt/werkt tijdens het programmeren, maar in dit
geval op basis van een abstracte programmeertaal die een superset aanbiedt van
constructies uit verschillende programmeertalen.

Enkele eigenschappen die deze doelstelling ondersteunen zijn:

\begin{description}

  \item[Functionele cross-referenties] Door middel van \emph{labels} en de
  \emph{selectie}- of \emph{zoek}-functionaliteit kan er op functionele wijze
  omgegaan worden met code. Zo kan een gebruiker aan de beschrijving van een
  functie een \emph{label} toekennen en hier later eenvoudig naar verwijzen om
  nog bijkomende logica toe te voegen.

  \item[Zoek-en-wijzig] Dankzij de functionele cross-referenties en de
  transparantie van een entiteit of een lijst van entiteiten is het eenvoudig
  om algemene aanpassingen door te voeren. Dit kan bv. gebruikt worden om aan
  het begin van alle declaratiesecties een blok commentaar te plaatsen of om
  systematisch aanpassingen met betrekking tot naamgeving door te voeren.

  \item[Automatisch vervolledigen] Het voorbeeld van het automatisch toevoegen
  van \ttt{import}-opdrachten of \ttt{prototypes} illustreet de mogelijkheid om
  de gebruiker te ontslaan van redundant en repetitief werk.

\end{description}
