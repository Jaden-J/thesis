%!TEX root=masterproef.tex

\subsection{Code model}
\label{subsection:devel-code-model}

Het CM staat in feite volledig los van de generator en het SM en is ook als een
op zich staand project ontwikkeld als generieke beschrijving van uitvoerbare
code, genaamd \emph{CodeCanvas}.

\subsubsection{CodeCanvas}

CodeCanvas biedt een API om hi\"erarchische structuren te bouwen. Deze kunnen
opgebouwd worden uit zelf te defini\"eren entiteiten bestaan. Standaard
voorziet CodeCanvas de concepten \emph{unit}, \emph{module} en \emph{sectie}.

Een \emph{unit} is het hoogste niveau en verzamelt alle onderliggende
\emph{modules}. Een \emph{module} komt overeen met een functioneel geheel en
bestaat standaard uit twee \emph{secties}, \'e\'en voor declaraties en \'e\'en
voor definities. Een \emph{sectie} kan verder ingevuld worden met \emph{code
instructies}.

Daarnaast biedt CodeCanvas de mogelijkheid om entiteiten te markeren met een
label (\emph{tag}) en onderliggende entiteiten te selecteren of zoeken
op basis van die labels.

Dankzij een \emph{vloeiende} (\emph{fluent}) interface laat CodeCanvas
toe om zeer leesbare operaties te formuleren op deze hi\"erarchische
codestructuren.

Codevoorbeeld \ref{lst:codecanvas-hello} toont een eenvoudig voorbeeld dat de
meeste functionaliteit toepast. 

\begin{listing}[ht]
  \begin{minted}[linenos,frame=lines,framesep=2mm,fontsize=\footnotesize]{python}
from structure import Unit, Module
import instructions as code
import languages.C  as C

unit = Unit().append( Module("hello") )
main = unit.select("hello", "dec").append(code.Function(name="main"))
main.append(code.Print("Hello World\n"))

print str(unit)
print C.Emitter().emit(unit)
print str(unit)
  \end{minted}
  \vspace{-5mm}
  \caption{Werking van het \emph{CodeCanvas}}
  \label{lst:codecanvas-hello}
\end{listing}

Het voorbeeld construeert op regel 5 een \emph{unit} aan en voegt er een
\emph{module} genaamd \ttt{hello} aan toe. Achterliggend worden bij de aanmaak
van een module onmiddellijk 2 \emph{secties} toegevoegd, genaamd \ttt{dec} en
\ttt{def}.

Op regel 6 wordt vertrekkende van de \emph{unit} de \ttt{dec} \emph{sectie}
geselecteerd. De \ttt{select} methode laat toe om een opeenvolgende reeks van
\emph{labels} te defini\"eren die het pad vormen vanaf de startentiteit tot de
te selecteren entiteit.

Een gelijkaardige methode, \ttt{find}, accepteert ook een variabele lijst
argumenten en zoekt vervolgens, vertrekkende van de startentiteit, naar
entiteiten die alle opgegeven \emph{labels} dragen.

Beide methodes kunnen lijsten van entiteiten teruggeven. Op deze lijsten kunnen
evengoed alle methoden opgeroepen worden als op een enkele entiteit. Dit
resulteert in een zeer transparante interface.

De geselecteerde sectie wordt vervolgens een functie toegevoegd met de naam
\ttt{main}. Op regel 7 wordt aan deze functie een \ttt{print} opdracht
toegevoegd. Tot slot wordt de \emph{unit} op twee manieren uitgevoerd: eerst
door er een tekstuele voorstelling van te maken en in tweede instantie door
gebruik te maken van een programmeertaal, in dit geval C.

De uitvoer van dit programma is weergegeven in \ref{lst:codecanvas-output} en
toont eerst de technische tekstuele voorstelling van de hi\"erarchie. Tussen
vierkante haakjes staan de \emph{labels} die aan een entiteit verbonden zijn.
Effectieve instructie-implementaties tonen hun parameters als een verzameling
van de naam en de waarde.

\begin{listing}[ht]
  \begin{minted}[linenos,frame=lines,framesep=2mm,fontsize=\footnotesize]{console}
  Module hello [hello]
    Section def [def]
    Section dec [dec]
      Function {'params': (), 'type': void, 'id': main}
        Print {'args': (), 'string': "Hello World\n"}

  void main(void);
  #import <stdio.h>
  void main(void) {
    printf("Hello World\n");
  }
  
  Module hello [hello]
    Section def [def]
      Prototype {'params': [], 'type': void, 'id': main}
    Section dec [dec]
      Import {'imported': '<stdio.h>'} [import_stdio] <sticky>
      Function {'params': [], 'type': void, 'id': main}
        Print {'args': (), 'string': "Hello World\n"}
  \end{minted}
  \vspace{-5mm}
  \caption{Uitvoer van voorbeeld werking van het \emph{CodeCanvas}}
  \label{lst:codecanvas-output}
\end{listing}

Het tweede deel van de uitvoer toont de overeenkomstige C code voor deze
hi\"erarchie. We merken hier op dat op regel 7 een prototype en op regel 8 een
\emph{import} opdracht verschijnen die niet in de hi\"erarchie voorkwamen.

Wanneer we een tweede maal de \emph{unit} omvormen tot een tekstuele
voorstelling, zien we deze twee entiteiten wel opduiken. De uitvoermodule voor
de C programmeertaal werkt in meerdere stappen. 

Tijdens een eerste fase wordt de hi\"erarchie doorlopen en worden uitbreidingen
gedaan. In dit voorbeeld gebeuren er twee: wanneer een \emph{print} opdracht
gevonden wordt, wordt een \emph{import} opdracht toegevoegd die de declaraties
van \ttt{stdio.h} zal inladen. Een tweede transformatie zal voor elke
\emph{functie} een prototype aanmaken in de declaratie sectie van de module.

Deze fase zorgt ook voor het omzetten van constructies die niet standaard
ondersteund worden naar uitwerkingen met constructies die wel bestaan bij de
beoogde taal. Een voorbeeld hiervan zijn bv. \emph{tuples}. Deze worden door de
omvormer van C herschreven door middel van structuren en functies om deze
structuren te benaderen en onderhouden.

In een tweede fase doorloopt de uitvoermodule opnieuw de volledige
hi\"erarchie, maar vormt nu elke entiteit om in de overeenkomstige tekstuele C
syntax.

Beide fasen worden ge\"implementeerd door middel van \emph{visitors}. Deze
\emph{visitor} is tevens beschikbaar van buitenaf en laat toe om andere
transformaties te implementeren.

\subsubsection{Filosofie}

De doelstelling van CodeCanvas is het aanbieden van een API die toelaat om te
werken zoals zoals programmeur denkt/werkt tijdens programmeren, maar nu op
basis van een abstracte programmeertaal die een superset aanbiedt van
constructies uit verschillende programmeertalen.

Enkele typische eigenschappen die deze doelstelling ondersteunen zijn:

\begin{description}

  \item[Functionele cross-referenties] - Door middel van \emph{labels} en de
  \emph{selectie} en \emph{zoek} functionaliteit kan er op functionele wijze
  omgegaan worden met code. Zo kan een gebruiker een in de beschrijving van een
  functie een \emph{label} toevoegen en hier later eenvoudig naar verwijzen om
  nog bijkomende logica toe te voegen.

  \item[Zoek-en-wijzig] - Dankzij de functionele cross-referenties en de
  transparantie van een enkele entiteit of een lijst van entiteiten is het heel
  eenvoudig om algemene aanpassingen door te voeren. Dit kan bv. gebruikt
  worden om aan het begin van alle declaratie-secties een standaard blok
  commentaar te plaatsen of om systematisch aanpassingen met betrekking tot
  naamgeving door te voeren.

  \item[Automatische vervolledigen] - Het voorbeeld van het automatisch
  toevoegen van \ttt{import} opdrachten of \ttt{prototypes} bieden een
  krachtige manier om de gebruiker te ontslaan van redundant en repetitief werk.

\end{description}
