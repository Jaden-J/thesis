%!TEX root=masterproef.tex

\subsection{Transformaties}
\label{subsection:devel-transformations}

Uit de gedetailleerde voorgaande bespreking van het SM en het CM leren we dat
de kracht van de oplossing niet zozeer zit in de statische taxonomie die beide
modellen aanbieden, maar wel in de transformaties.

Het idee hiervoor werd ontleend aan een analyse- en ontwikkelparadigma dat
gebruik maakt van zgn. model-gedreven architectuur (\emph{Model Driven
Architecture}) (MDA) \citep{soley2000model,kleppe2003mda}. MDA focust op UML,
maar de principes zijn overdraagbaar naar andere modelvoorstellingen.

Het principe vertrekt van een hoog-niveau en zeer abstracte beschrijving van
een probleemdomein in de vorm van een platformonafhankelijk model
(\emph{Platform Independent Model}) (PIM) en evolueert door een opeenvolging
van (model) transformaties (MT) stapsgewijs tot een platformspecifiek model
(\emph{Platform Specific Model}) (PSM).

In bijlage \ref{appendix:visitor} wordt het \emph{visitor}-patroon voorgesteld.
Dit is de basis voor de implementatie van transformaties in de generator. Naast
het theoretische patroon wordt ook ingegaan op de technische uitwerking in
Python.
