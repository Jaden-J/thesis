%!TEX root=masterproef.tex

\subsection{Transformaties}
\label{subsection:devel-transformations}

Uit de gedetailleerde voorgaande bespreking van het SM en het CM leren we dat
de kracht van de oplossing niet zozeer in de statische taxonomie zit die beide
modellen aanbieden, maar wel in de flexibiliteit die mogelijk gemaakt wordt
door de transformaties.

Het idee hiervoor werd ontleend aan een analyse- en ontwikkel-paradigma dat
gebruik maakt van zgn. model-gedreven architectuur (\emph{Model Driven
Architecture}) (MDA) \citep{soley2000model,kleppe2003mda}. Oorspronkelijk was
MDA gefocust op UML, maar de principes zijn overdraagbaar naar generieke
modelvoorstellingen.

Het principe vertrekt van een hoog-niveau en zeer abstracte beschrijving van
een probleemdomein in de vorm van een platformonafhankelijk model
(\emph{Platform Independent Model}) (PIM) en wordt door een opeenvolging van
(model) transformaties (MT) stapsgewijs naar een platform-specifiek model
(\emph{Platform Specific Model}) (PSM) ge\"evolueerd.

In bijlage \ref{appendix:visitor} wordt het \emph{visitor}-patroon voorgesteld.
Dit is de basis voor de implementatie van transformaties in de generator. Naast
het theoretische patroon wordt ook ingegaan op de technische uitwerking in
Python.
