%!TEX root=masterproef.tex

\chapter{Reputatie}
\label{appendix:reputation}

In deze bijlage belichten we de mathematische onderbouw van een door
\citep{ganeriwal2008reputation} beschreven architectuur voor het opbouwen van
een reputatie en vertrouwen met betrekking tot een sensorknoop in het netwerk.

Gegeven knopen $i$ en $j$, met $\alpha_j$, het aantal observaties van acties
van knoop $j$ dat als co\"operatief werd beschouwd, en $\beta_j$, het aantal
niet co\"operatieve acties, toont men aan dat de reputatie van knoop $j$ wordt
weergegeven door een beta-distributie van $\alpha_j$ en $\beta_j$:

\begin{equation} \label{eq:reputation-beta}
R_{ij} \sim Beta(\alpha_j+1, \beta_j+1)
\end{equation}

Van deze reputatie kan vervolgens een vertrouwen bepaald worden van knoop $i$
ten opzichte van knoop $j$ als volgt: 

\begin{equation} \label{eq:reputation-trust}
\begin{array}{rcl}
T_{ij} & = & E(R_{ij}) \\
       & = & E(Beta(\alpha_j+1, \beta_j+1)) \\
       & = & \frac{\alpha_j+1}{\alpha_j+\beta_j+2} \\
\end{array}
\end{equation}

$\alpha_j$ en $\beta_j$ evolueren doorheen de tijd. Hierbij dienen enerzijds
nieuwe observaties binnen afzonderlijke tijdspannes beschouwd te worden, maar
moet ook een wegingsfactor toegepast worden op de oude waarden, om er voor te
zorgen dat een historisch opgebouwd beeld niet dominant blijft, en nieuwe
wijzigingen in het gedrag overstemt. Gegeven $r$ het aantal co\"operatieve
observaties in een bepaalde tijdspanne en $s$ het aantal niet-co\"operatieve
observaties in diezelfde tijdspanne worden de nieuwe waarden voor $\alpha_j$ en
$\beta_j$ gegeven door:

\begin{equation} \label{eq:reputation-update-direct}
\begin{array}{rcl}
\alpha^{new}_j & = & (w_{age} \times \alpha_j) + r \\
\beta^{new}_j  & = & (w_{age} \times \beta_j) + s \\
\end{array}
\end{equation}

Hierbij is $w_{age}$ een factor ($< 1$) die zorgt voor een afname van de
belangrijkheid van de oudere informatie.

Naast deze eigen directe observaties kunnen ook indirecte observaties door
naburige knopen in beschouwing genomen worden. Voor zo'n naburige knoop, $k$,
zal een knoop $i$ eveneens een vertrouwen $T_{ik}$ kunnen bepalen op basis van
$\alpha_k$ en $\beta_k$. Knoop $k$ kan vervolgens zijn eigen informatie met
betrekking tot de reputatie van knoop $j$ kenbaar maken als $\alpha^k_j$ en
$\beta^k_j$. Knoop $i$ kan vervolgens zijn parameters bijwerken als volgt:

\begin{equation} \label{eq:reputation-update-indirect}
\begin{array}{rrcl}
& \alpha^{new}_j & = & \alpha_j + ( w^k_{rep} \times \alpha^k_j ) \\
& \beta^{new}_j  & = & \beta_j  + ( w^k_{rep} \times \beta^k_j )  \\
met \\
& w^k_{rep}      & = & \frac{2 \alpha_k}{(\beta_k+2) (\alpha^k_j+\beta^k_j+2)+2 \alpha_k} \\
\end{array}
\end{equation}

De factor $w^k_{rep}$ zorgt er voor dat de opname van indirecte informatie van
knoop $k$ in verhouding tot zijn reputatie zal gebeuren.

Enkele bijkomende regels beschermen tegen typische problemen gerelateerd aan
deze aanpak: een knoop accepteert slechts indirecte informatie van een andere,
indien deze knoop zelf als vertrouwenswaardig wordt beschouwd. Hierbij wordt
een drempelwaarde ($TH_{SHI}$) gehanteerd. Verder wordt enkel positieve
informatie uitgewisseld, om negatieve be\"invloeding te vermijden. Tot slot
wordt tevens alleen directe informatie uitgewisseld, om de onafhankelijkheid
van de informatie te garanderen.
