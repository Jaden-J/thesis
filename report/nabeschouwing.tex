%!TEX root=masterproef.tex

\chapter*{Nabeschouwing}
\label{nabeschouwing}

In de eerste paragrafen van dit verslag van deze masterproef ben ik de iets wat
ongewone situatie van mijn masterproef niet uit de weg gegaan. Deze laatste
paragrafen wil ik benutten om mijn werk te kaderen en te reflecteren over de
persoonlijke ervaring van de afgelopen maanden.

\begin{quote}
\emph{Het is niet iedereen gegeven om op 40 jarige leeftijd een masterproef te
mogen, of eerder kunnen, maken. De keuze om opnieuw te gaan studeren maak je
niet alleen en vraagt van veel mensen een bijzondere inspanning. Familie en
vrienden, maar ook professoren en assistenten worden plots uit hun vertrouwde
omgeving weggerukt en worden geconfronteerd met een ongewone situatie.}
\end{quote}

Deze ongewone situatie wordt vooral getekend door een voorgeschiedenis met een
reeds uitgebreide ervaring en opgebouwde specialisatie. In mijn geval betreft
het misschien eerder de capaciteit om uitermate generalistische oplossingen te
vinden voor een brede waaier aan informatica-gerelateerde problemen.

Een masterproef is het hoogtepunt van een ingenieursopleiding. Misschien nog
meer dan voor mijn jongere collega's, zou deze opdracht een ware uitdaging
worden om mijn generalistische geest te kaderen in een specialisatie binnen een
zeer sterk afgebakende context. Was ik, na drie jaar, in staat om alle nieuwe
verworven kennis op de juiste manier toe te passen en mij inderdaad master in
het onderwerp te mogen noemen.

Bij aanvang van zowel de voorafgaande studiejaren als de masterproef zelf, had
ik duidelijke een doelstelling om niet zomaar te vervallen in oude gewoontes en
te berusten op mijn ervaring. Mijn masterproef moest inderdaad een duidelijke
ommekeer tonen in mijn manier van denken en moest zonder twijfel getuigen van
een correcte onderzoeksattitude. Daarnaast belichaamde deze hernieuwde
studieperiode de ultieme kans om een herori\"entering van mijn vakgebied te
realiseren.

Met open geest vertrok ik op zoek naar wat mij echt boeide en waar ik mij
effectief meester over wilde maken. Elk vak en elke topic werden onderworpen
aan een diepgaande evaluatie. Elk gesprek met professoren en assistenten peilde
naar mogelijkheden en perspectieven. Elke taak werd beschouwd als een test die
moest uitwijzen of ik er mijn nieuwe toekomst in kon vinden.

Na ongeveer twee jaar was het \emph{moment supr\`eme} aangebroken en moest ik
een keuze maken. Het onderwerp van de masterproef moest bepaald worden. Tot
mijn ontzetting voelde ik mij er nog helemaal niet klaar voor. Na een
twintigtal vakken en enorm veel informele gesprekken had ik de graal nog niet
gevonden. Geen van de voorgestelde topics konden mijn ogenschijnlijk bekoren en
het was duidelijk dat ik op \'e\'en of andere manier zelf een onderwerp moest
voorstellen.

\'E\'en vakgebied had zich echter van de andere onderscheiden: digitale
electronica en embedded systemen. Dankzij de kennismaking met Professor Hughes
was de wereld van draadloze sensorknopen aan mij ge\"introduceerd en het was
liefde op het eerste gezicht. Misschien was het zelfs geen nieuwe liefde, maar
een liefde die al bestond sinds de vroege jaren tachtig. In een tijdperk waarin
nog amper sprake was van Personal Computers en informatica nog heel dicht stond
bij de electronica die ze implementeerde, zette ik mijn eerste stappen. Reeds
toen voelde ik een grote affiniteit met de zeer elementaire bouwstenen van
computers en de talen die ze konden besturen. De liefde voor compilers is nooit
ver zoek geweest.

Enkele jaren later werd dit, hoofdzakelijk door de intrede van het internet en
niet op zijn minst door de opportuniteit om in \'e\'en van de eerste
internetaanbieders in Antwerpen te mogen helpen, uitgebreid met de mogelijkheid
om systemen met elkaar te verbinden via netwerken. Nog enkele jaren later zou
dit zich nog verder concretiseren tijdens mijn studies aan de KHLeuven en mijn
eerste bedrijfservaringen die volledig in het teken stonden van netwerken en de
beveiliging ervan.

Ofschoon ik mij had voorgenomen om nieuwe terreinen te verkennen, begon alles
toch te wijzen op een enorme samengang van oude en nieuwe liefdes. Met
inbraakdetectie in draadloze sensornetwerken zag ik het huwelijk van
verschillende persoonlijk interesses voltrokken worden.

Net zoals een koppel evolueert na die eerste huwelijksdagen, was mijn
masterproef evenzeer een sterk evoluerende ervaring - zij het dan op een veel
kortere tijdspanne, maar daardoor zeker zo intens. De initi\"ele verwachtingen
maakten snel plaats voor ontzetting en twijfel. Had ik wel de juiste keuze
gemaakt? Had ik wel voldoende vooronderzoek gedaan? Wat als nu deze masterproef
helemaal niet uitdraaide zoals ik mij enkele jaren voordien had voorgenomen?
Had ik ondertussen wel de nodige capaciteiten opgebouwd en was ik wel de
specialist die een masterproef vereist? Of was ik nog steeds dezelfde
generalist, echter nu met wat extra gereedschap in mijn spreekwoordelijke
koffer?

Een antwoord vond ik in de de eigen woorden van de faculteit:
\begin{quote}
\emph{De masterproef wordt beschouwd als een heel belangrijk leermoment in de
ingenieursopleidingen, omdat je als student op een actieve wijze betrokken
wordt bij het onderzoek in het domein waarvoor je gekozen hebt.}
\end{quote}

Het kernwoord voor mij was \emph{onderzoek}; het zoeken betreft hier niet
louter het zoeken naar antwoorden, maar evenzeer naar hypothesen, het zoeken
naar de onderliggende beweegredenen en het durven aanwijzen van problemen.

Ofschoon ik een domein gekozen had en binnen dit domein een deelgebied had
geselecteerd was niet het einde van het defini\"eren van mijn masterproef. Het
onderzoeken van dit domein moest in eerste plaats uitwijzen wat de specifieke
onderzoeksvraag zou zijn die ik op mij durfde te nemen. De ontzetting en angst
die tijdens de eerste weken van literatuurstudie zich had meester gemaakt van
mij, was feitelijk het ontluiken van mijn eigen visie en drijfveer. Plots
kwamen dan ook mijn oude gewoonten opnieuw opzetten.

Net zoals zo vaak tijdens mijn voorgaande professionele carri\`ere was de
ontdekking van de onverwachte chaos en wantoestanden de bron om mijn stoute
schoenen aan te trekken en al mijn ervaring uit de kast te halen om orde op
zaken te scheppen. Het werd snel duidelijk dat het dogmatisch afzweren van mijn
voorgeschiedenis een grove fout zou geweest zijn. Het huwelijk van oude en
nieuwe liefdes, van elementaire informatica en draadloze sensoren werd al snel
ondersteund door software architectuur en code generatie, twee sterke getuigen
uit bijna vervlogen gewaande tijden.

Op dat ogenblik besefte ik dat ik mijn buitengewone situatie dringend moest
omarmen en gebruik moest maken van de unieke kans om mijn oude ervaring en mijn
nieuw verworven kennis te bundelen om een onderwerp te tackelen dat misschien
zelfs iets te groot was voor een \emph{gewone} masterproef.

Het resultaat staat op deze voorgaande bladzijden vereeuwigd. Of het potentieel
dat er in zit ooit ten volle mijn visie zal bereiken zal de toekomst moeten
uitwijzen. Voor mij persoonlijk is deze masterproef een onverdeeld succes. In
het verloop van \'e\'en jaar heb ik al mijn oude ervaring en kennis kunnen en
moeten aanscherpen en mogen verrijken met een enorme nieuwe bagage aan
wetenschappelijke technieken en methoden. Alles wat de afgelopen 30 jaar voor
mij belangrijk is geweest zit vervat in deze masterproef: programmeertalen,
electronica, code generatie, software architectuur, functionele analyse en
eindgebruiker-orientatie \dots zijn in 75 bladzijden samengebracht tot een
resultaat waar ik trots op ben en waar ik hopelijk een nieuw hoofdstuk en nog
vele daarna zal over kunnen schrijven in de nabije toekomst.

\bigskip

Dank aan allen die dit mogelijk hebben gemaakt.\\
Ja, jullie allemaal.
